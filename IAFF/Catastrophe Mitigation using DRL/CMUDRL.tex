%&latex
\documentclass[a4paper]{article}

\usepackage[a4paper,margin=1in]{geometry}
\usepackage[affil-it]{authblk}
\usepackage{cite}
\usepackage[unicode]{hyperref}
\usepackage[utf8]{inputenc}
\usepackage[english]{babel}
\usepackage{csquotes}
\usepackage{amsmath,amssymb,amsthm}
\usepackage{enumerate}
\usepackage{commath}

\newcommand{\Comment}[1]{}

\newcommand{\Bool}{\{0,1\}}
\newcommand{\Words}{{\Bool^*}}

% operators that are separated from the operand by a space
\DeclareMathOperator{\Sgn}{sgn}
\DeclareMathOperator{\Supp}{supp}
\DeclareMathOperator{\Stab}{stab}
\DeclareMathOperator{\Img}{Im}
\DeclareMathOperator{\Dom}{dom}

% autosize deliminaters
\newcommand{\AP}[1]{\left(#1\right)}
\newcommand{\AB}[1]{\left[#1\right]}

% operators that require brackets
\newcommand{\Pa}[2]{\underset{#1}{\operatorname{Pr}}\AB{#2}}
\newcommand{\PP}[2]{\underset{\substack{#1 \\ #2}}{\operatorname{Pr}}}
\newcommand{\PPP}[3]{\underset{\substack{#1 \\ #2 \\ #3}}{\operatorname{Pr}}}
\newcommand{\E}[1]{\underset{#1}{\operatorname{E}}}
\newcommand{\Ea}[2]{\underset{#1}{\operatorname{E}}\AB{#2}}
\newcommand{\EE}[2]{\underset{\substack{#1 \\ #2}}{\operatorname{E}}}
\newcommand{\EEE}[3]{\underset{\substack{#1 \\ #2 \\ #3}}{\operatorname{E}}}
\newcommand{\I}[1]{\underset{#1}{\operatorname{I}}}
\newcommand{\Ia}[2]{\underset{#1}{\operatorname{I}}\AB{#2}}
\newcommand{\II}[2]{\underset{\substack{#1 \\ #2}}{\operatorname{I}}}
\newcommand{\III}[3]{\underset{\substack{#1 \\ #2 \\ #3}}{\operatorname{I}}}
\newcommand{\Var}{\operatorname{Var}}

% operators that require parentheses
\newcommand{\En}{\operatorname{H}}
\newcommand{\Ena}[1]{\operatorname{H}\AP{#1}}
\newcommand{\Hom}{\operatorname{Hom}}
\newcommand{\End}{\operatorname{End}}
\newcommand{\Sym}{\operatorname{Sym}}

\newcommand{\Prj}{\operatorname{pr}}

\newcommand{\D}{\mathrm{d}}
\newcommand{\KL}[2]{\operatorname{D}_{\mathrm{KL}}(#1 \| #2)}
\newcommand{\Dtv}{\operatorname{d}_{\text{tv}}}
\newcommand{\Dtva}[1]{\operatorname{d}_{\text{tv}}\AP{#1}}

\newcommand{\Argmin}[1]{\underset{#1}{\operatorname{arg\,min}}\,}
\newcommand{\Argmax}[1]{\underset{#1}{\operatorname{arg\,max}}\,}

\newcommand{\Nats}{\mathbb{N}}
\newcommand{\Ints}{\mathbb{Z}}
\newcommand{\Rats}{\mathbb{Q}}
\newcommand{\Reals}{\mathbb{R}}
\newcommand{\Coms}{\mathbb{C}}

\newcommand{\Sq}[2]{\{#1\}_{#2 \in \Nats}}
\newcommand{\Sqn}[1]{\Sq{#1}{n}}

\newcommand{\Estr}{\boldsymbol{\lambda}}

\newcommand{\Lim}[1]{\lim_{#1 \rightarrow \infty}}
\newcommand{\LimInf}[1]{\liminf_{#1 \rightarrow \infty}}
\newcommand{\LimSup}[1]{\limsup_{#1 \rightarrow \infty}}

\newcommand{\Abs}[1]{\lvert #1 \rvert}
\newcommand{\Norm}[1]{\lVert #1 \rVert}
\newcommand{\Floor}[1]{\lfloor #1 \rfloor}
\newcommand{\Ceil}[1]{\lceil #1 \rceil}
\newcommand{\Chev}[1]{\langle #1 \rangle}
\newcommand{\Quote}[1]{\ulcorner #1 \urcorner}

\newcommand{\Alg}{\xrightarrow{\textnormal{alg}}}
\newcommand{\M}{\xrightarrow{\textnormal{k}}}
\newcommand{\PF}{\xrightarrow{\circ}}

% Paper specific

\newcommand{\Ob}{\mathcal{O}}
\newcommand{\A}{\mathcal{A}}
\newcommand{\St}{\mathcal{S}}
\newcommand{\T}{\mathcal{T}}
\newcommand{\R}{\mathcal{R}}
\newcommand{\In}{\mathcal{I}}
\newcommand{\FH}{(\A \times \Ob)^*}
\newcommand{\IH}{(\A \times \Ob)^\omega}
\newcommand{\Ado}{\bar{\Ob}}
\newcommand{\Ada}{\bar{\A}}
\newcommand{\Adi}{{\bar{\In}}}
\newcommand{\Adao}{\overline{\A \times \Ob}}
\newcommand{\Adfh}{\Adao^*}
\newcommand{\Adih}{\Adao^\omega}
\DeclareMathOperator{\HD}{hdom}
\newcommand{\Hy}{\mathcal{H}}
\newcommand{\UC}{\mathcal{U}}
\newcommand{\SF}{\St^{\text{F}}}
\newcommand{\SD}{\St^{\text{D}}}
\newcommand{\SC}{\St^{\text{C}}}
\newcommand{\MF}{M^{\text{F}}}

\newcommand{\Ut}{\operatorname{U}}
\newcommand{\V}{\operatorname{V}}
\newcommand{\Q}{\operatorname{Q}}
\newcommand{\EU}{\operatorname{EU}}

\newcommand{\Dl}{\mathcal{D}}
\newcommand{\Do}{\mathfrak{D}}
\newcommand{\F}{\mathcal{F}}
\newcommand{\B}{\mathcal{B}}
\newcommand{\Z}{Z}
\newcommand{\J}{J}

\begin{document}

[Previously](https://agentfoundations.org/item?id=1656) we derived a regret bound for DRL which assumed the advisor is "locally sane." Such an advisor can only take actions that don't lose any value in the long term. In particular, if the environment contains a latent *catastrophe* that manifests with a certain rate (such as the possibility of an UFAI), a locally sane advisor has to take the optimal course of action to mitigate it, since every delay yields a positive probability of the catastrophe manifesting and leading to permanent loss of value. This state of affairs is unsatisfactory, since we would like to have performance guarantees for an AI that can mitigate catastrophes that the human operator cannot mitigate on their own. To address this problem, we introduce a new form of DRL where in every hypothetical environment the set of uncorrupted states is divided into "dangerous" (impending catastrophe) and "safe" (catastrophe was mitigated). The advisor is then only required to be locally sane in safe states, whereas in dangerous states certain "leaking" of long-term value is allowed. We derive a regret bound in this setting as a function of the time discount factor, the expected value and standard deviation of the catastrophe mitigation time for the optimal policy, and the "value leak" rate (i.e. essentially the rate of catastrophe occurrence). The form of this regret bound implies that in certain asymptotic regimes, the agent attains near-optimal expected utility (and in particular mitigates the catastrophe with probability close to 1), whereas the advisor on its own *fails* to mitigate the catastrophe with probability close to 1. Thus, this formalism can be regarded as a simple model of aligned superintelligence: the agent is *aligned* since near-optimal utility is achieved despite the presence of corrupted states (which are treated more or less the same as before) and it is *super*intelligent since its performance is vastly better than the performance of the advisor.

***

Appendix A contains the proofs...

\section{Notation}

Whatever...

\section{Results}

We start by formalising the concepts of a "catastrophe" and "catastrophe mitigation" in the language of MDPs.

\#Definition 1

Fix $\delta \in (0,1)$. A *$\delta$-catastrophe MDP* is an MDP $M$ together with a partition of $\St_M$ into subsets $\St_M:=\SF_M \sqcup \SD_M \sqcup \SC_M$ (safe, dangerous and corrupt states respectively) s.t. there is some $\pi^2: \St_M \rightarrow \A_M$ with the properties

i. $\pi^2$ is 2-optimal.

ii. For any $s \in \SF_M$, $\T_{M\pi^2}\AP{\SF_M \mid s}=1$.

iii. For any $s \in \SD_M$, $\T_{M\pi^2}\AP{\SC_M \mid s} < \delta$.

\#Definition 2

Fix $\delta\in(0,1)$ and a $\delta$-catastrophe MDP $M$. Consider some $\pi^1: \St_M \rightarrow \A_M$. $\pi^1$ is said to be a *mitigation policy for $M$* when

i. $\pi^1$ is 1-optimal.

ii. For any $s \in \SF_M$, $\T_{M\pi^1}\AP{\SF_M \mid s}=1$.

iii. For any $s \in \SD_M$, $\T_{M\pi^1}\AP{\SC_M \mid s} < \delta$.

Fix $\tau_1,\tau_2 \in (0,\infty)$ and $s_0 \in \SD_M$. $\pi^1$ is said to have *mitigation time moments $\tau_1,\tau_2$ at $s_0$* when

iv. $$\sum_{n=0}^\infty n \AP{\T_{M\pi^1}^{n+1}\AP{\SF_M \mid s_0}-\T_{M\pi^1}^{n}\AP{\SF_M \mid s_0}} \leq \tau_1$$

v. $$\sum_{n=0}^\infty n^2 \AP{\T_{M\pi^1}^{n+1}\AP{\SF_M \mid s_0}-\T_{M\pi^1}^{n}\AP{\SF_M \mid s_0}} \leq \tau_2$$

***

Next, we introduce the notion of an MDP perturbation. We will use it by considering perturbations of a catastrophe MDP which "eliminate the catastrophe."

\#Definition 3

Fix $\delta,\delta'\in(0,1)$ and consider a $\delta'$-catastrophe MDP $M$. An MDP $\tilde{M}$ is said to be a *$\delta$-perturbation of $M$* when

i. $\St_{\tilde{M}} = \St_M$

ii. $\A_{\tilde{M}} = \A_M$

iii. $\R_{\tilde{M}}=\R_M$

iv. For any $s \in \SF_M$ and $a \in \A_M$, $\T_{\tilde{M}}\AP{s,a}=\T_{M}\AP{s,a}$

v. For any $s \in \SD_M$ and $a \in \A_M$, $\Dtva{\T_{\tilde{M}}\AP{s,a},\T_{M}\AP{s,a}} < \delta$

***

Similarly, we can consider perturbations of a policy.

\#Definition 4

Fix $\delta,\delta'\in(0,1)$ and consider a $\delta'$-catastrophe MDP $M$. Given $\pi: \St_M \M \A_M$ and $\tilde{\pi}: \St_M \M \A_M$, $\tilde{\pi}$ is said to be a *$\delta$-perturbation of $\pi$* when

i. For any $s \in \SF_M$, $\tilde{\pi}(s) = \pi(s)$.

ii. For any $s \in \SD_M$, $\Dtva{\tilde{\pi}(s),\pi(s)} < \delta$.

***

We will also need to introduce policy-specific value functions, Q-functions and relatively $k$-optimal actions.

\#Definition 5

Fix an MDP $M$ and $\pi: \St_M \M \A_M$. We define $\V_{M\pi}: \St_M \times (0,1) \rightarrow [0,1]$ and $\Q_{M\pi}: \St_M \times \A_M \times (0,1) \rightarrow [0,1]$ by

$$\V_{M\pi}(s,\gamma) := (1-\gamma) \sum_{n=0}^\infty \gamma^n \Ea{\T_{M\pi}^n(s)}{\R_M}$$

$$\Q_{M\pi}(s,a,\gamma) := (1-\gamma) \R_M(s) + \gamma \Ea{t \sim \T_{M\pi}(s)}{\V_{M\pi}(t,\gamma)}$$

For each $k \in \Nats$, we define $\V_{M\pi}^k: \St_M \rightarrow [0,1]$, $\Q_{M\pi}^k: \St_M \times \A_M \rightarrow [0,1]$ and $\A_{M\pi}^k: \St_M \rightarrow 2^{\A_M}$ by

$$\V_{M\pi}^k(s) := \frac{\D^k \V_{M\pi}(s,\gamma)}{\D\gamma^k}\bigg\vert_{\gamma=1}$$

$$\Q_{M\pi}^k(s,a) := \frac{\D^k \Q_{M\pi}(s,a,\gamma)}{\D\gamma^k}\bigg\vert_{\gamma=1}$$

$$\A_{M\pi}^0(s) := \{a \in \A_M \mid \Q_{M\pi}^0(s,a) \geq \V_{M\pi}^0(s)\}$$

$$\A_{M\pi}^{k+1}(s) := \{a \in \A_{M\pi}^k(s) \mid \Q_{M\pi}^{k+1}(s,a) \geq \V_{M\pi}^{k+1}(s) \text{ or } \exists j \leq k: \Q_{M\pi}^{j}(s,a) > \V_{M\pi}^{j}(s)\}$$

***

Now we give the new (weaker) condition on the advisor policy. For notational simplicity, we assume the policy is stationary. It is easy to generalize these results to non-stationary advisor policies and to policies that depend on irrelevant additional information (i.e. policies for universes that are *realizations* of the MDP).

\#Definition 6

Fix $\epsilon,\delta \in (0,1)$. Consider a $\delta$-catastrophe MDP $M$ with mitigation policy $\pi^1: \St_M\ \rightarrow A_M$. A policy $\pi: \St_M \M \A_M$ is said to be *locally $(\epsilon,\delta)$-sane* for $M$ when there exists a $\delta$-perturbation $\tilde{M}$ of $M$ and a $\delta$-perturbation $\tilde{\pi}$ of $\pi$ s.t.

i. For any $s \in \SF_M$: $$\T_{M\pi}\AP{\SF_M \mid s} = 1$$

ii. For any $s \in \SD_M$: $$\T_{\tilde{M}\tilde{\pi}}\AP{\SC_M \mid s} = 0$$

iii. For any $s \in \St_M \setminus \SC_M$: $$\Supp{\tilde{\pi}(s)} \subseteq \A_{\tilde{M}\pi^1}^0(s)$$

iv. For any $s \in \St_M \setminus \SC_M$: $$\tilde{\pi}\AP{\pi^1(s) \mid s} > \epsilon$$

\Comment{For $M$ an arbitrary MDP, $s_0 \in \St_M$ and $\tau_1,\tau_2 \in (0,\infty)$, we say $\pi$ is *locally $(\epsilon,\delta)$-sane for $\AP{M,s_0}$ with moments $\tau_1,\tau_2$* when *either* $\pi$ is locally $\epsilon$-sane for $\AP{M,s_0}$ *or* there is *some* way to view $M$ as a $\delta$-catastrophe MDP with $s_0 \in \SD_M$ such that the corresponding mitigation policy $\pi^1$ has moments $\tau_1,\tau_2$ at $s_0$ and $\pi$ is locally $(\epsilon,\delta)$-sane for $M$ in the corresponding sense.}

***

Note that a locally $(\epsilon,\delta)$-sane policy still has to be $0$-optimal in $\SF_M$. This requirement seems reasonably realistic, since, roughly speaking, it only means that there is *some* way to "rearrange the universe" that the agent can achieve, and that would be "endorsed" by the advisor, s.t this rearrangement doesn't destroy substantially much value and s.t. after this rearrangement, there is no "impending catastrophe" that the agent has to prevent and the advisor wouldn't be able to prevent in its place. In particular, this rearrangement may involve creating some subagents inside the environment and *destroying the original agent*, in which case any policy on $\SF_M$ is "vacuously" optimal (since all actions have no effect).

We can now formulate the main result.

\#Theorem 1

Fix an interface $\In=(\A,\Ob)$, $N \in \Nats$, $\epsilon \in (0,1)$ and for each $k \in [N]$, an MDP $\MF_k$ s.t. $\A_{\MF_k} = \A$. Now consider for each $k \in [N]$, an $\In$-universe $\upsilon^k=(\mu^k,r^k)$ which is an $\Ob$-realization of a catastrophe MDP $M_k$ with state function $S^k$ s.t.

i. $\SF_{M_k} = \St_{\MF_k}$

ii. For each $s \in \St_{\MF_k}$ and $a \in \A$, $\T_{M_k}(s,a) \mid \St_{\MF_k} = \T_{\MF_k}(s,a)$.

iii. For each $s \in \St_{\MF_k}$, $\R_{M_k}(s)=\R_{\MF_k}(s)$.

iv. Given $k,j \in [N]$ and $h \in \HD{\mu^k} \cap \HD{\mu^j}$, if $S^k(h) \in \St_{M^k} \setminus \SC_{M^k}$ and $S^j(h) \in \St_{M^j} \setminus \SC_{M^j}$, then $r^k(h)=r^j(h)$.

Consider also $\alpha,\delta\in(0,1)$, $\tau_{1,2} \in (0,\infty)$ and $\sigma^k$ a locally $(\epsilon,\delta)$-sane policy for $M_k$. Assume $M_k$ has a mitigation policy with mitigation time moments $\tau_{1,2}$. Then, there exists an $\Adi$-policy $\pi^*$ s.t. for any $k \in [N]$

$$\EU_{\upsilon^k}^*(1-\alpha) - \EU_{\bar{\upsilon}^k\AB{\sigma^k}}(1-\alpha) = O\AP{\tau_2 \alpha + (\tau_1 \alpha)^{1/4} + \delta \AP{\tau_2 + \tau_1 \alpha^{-3/4}}}$$

Here, $\epsilon$ and the $\MF_k$ are regarded as *fixed* and we don't explicitly examine their effect on regret, whereas $\alpha$, $\delta$, $\tau_{1,2}$ and the $M_k$ are regarded as variable with the asymptotics $\alpha,\delta \rightarrow 0$, $\tau_{1,2} \rightarrow \infty$.

***

% Why fixing \MF_k is not a strong requirement...
% Discussion of asymptotics when advisor does poorly and agent does well...
% Implications for "structural" case...

Bar

\section{Appendix A}

\#Proposition A.1

Hahaha

\#Proof of Proposition A.1

Mwhahaha

\end{document}



