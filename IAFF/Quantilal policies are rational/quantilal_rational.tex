%&latex
\documentclass[a4paper]{article}

\usepackage[a4paper,margin=1in]{geometry}
\usepackage[affil-it]{authblk}
\usepackage{cite}
\usepackage[unicode]{hyperref}
\usepackage[utf8]{inputenc}
\usepackage[english]{babel}
\usepackage{csquotes}
\usepackage{amsmath,amssymb,amsthm}
\usepackage{enumerate}
\usepackage{commath}

\newcommand{\Comment}[1]{}

% operators that are separated from the operand by a space
\DeclareMathOperator{\Sgn}{sgn}
\DeclareMathOperator{\Supp}{supp}
\DeclareMathOperator{\Dom}{dom}
\DeclareMathOperator{\Img}{Im}

% autosize delimiters
\newcommand{\AP}[1]{\left(#1\right)}
\newcommand{\AB}[1]{\left[#1\right]}
\newcommand{\AC}[1]{\left\{#1\right\}}
\newcommand{\APM}[2]{\left(#1\;\middle\vert\;#2\right)}
\newcommand{\ABM}[2]{\left[#1\;\middle\vert\;#2\right]}
\newcommand{\ACM}[2]{\left\{#1\;\middle\vert\;#2\right\}}

% operators that require brackets
\newcommand{\Pa}[2]{\underset{#1}{\operatorname{Pr}}\AB{#2}}
\newcommand{\CP}[3]{\underset{#1}{\operatorname{Pr}}\ABM{#2}{#3}}
\newcommand{\PP}[2]{\underset{\substack{#1 \\ #2}}{\operatorname{Pr}}}
\newcommand{\PPP}[3]{\underset{\substack{#1 \\ #2 \\ #3}}{\operatorname{Pr}}}
\newcommand{\E}[1]{\underset{#1}{\operatorname{E}}}
\newcommand{\Ea}[2]{\underset{#1}{\operatorname{E}}\AB{#2}}
\newcommand{\CE}[3]{\underset{#1}{\operatorname{E}}\ABM{#2}{#3}}
\newcommand{\EE}[2]{\underset{\substack{#1 \\ #2}}{\operatorname{E}}}
\newcommand{\EEE}[3]{\underset{\substack{#1 \\ #2 \\ #3}}{\operatorname{E}}}
\newcommand{\Var}{\operatorname{Var}}
\newcommand{\I}[1]{\underset{#1}{\operatorname{I}}}
\newcommand{\CI}[3]{\underset{#1}{\operatorname{I}}\ABM{#2}{#3}}
\newcommand{\Ia}[2]{\underset{#1}{\operatorname{I}}\AB{#2}}
\newcommand{\II}[2]{\underset{\substack{#1 \\ #2}}{\operatorname{I}}}
\newcommand{\III}[3]{\underset{\substack{#1 \\ #2 \\ #3}}{\operatorname{I}}}

% operators that require parentheses
\newcommand{\En}{\operatorname{H}}
\newcommand{\Ena}[1]{\operatorname{H}\AP{#1}}
\newcommand{\PS}[1]{\mathcal{P}\AP{#1}}

\newcommand{\D}{\mathrm{d}}
\newcommand{\KL}[2]{\operatorname{D}_{\mathrm{KL}}\AP{#1\middle\vert\middle\vert#2}}
\newcommand{\RD}[3]{\operatorname{D}_{#1}\AP{#2\middle\vert\middle\vert#3}}
\newcommand{\Dtv}{\operatorname{d}_{\textnormal{tv}}}
\newcommand{\Dtva}[1]{\operatorname{d}_{\textnormal{tv}}\AP{#1}}

\newcommand{\Argmin}[1]{\underset{#1}{\operatorname{arg\,min}}\,}
\newcommand{\Argmax}[1]{\underset{#1}{\operatorname{arg\,max}}\,}

\newcommand{\Nats}{\mathbb{N}}
\newcommand{\Ints}{\mathbb{Z}}
\newcommand{\Rats}{\mathbb{Q}}
\newcommand{\Reals}{\mathbb{R}}
\newcommand{\Coms}{\mathbb{C}}

\newcommand{\Estr}{\boldsymbol{\lambda}}

\newcommand{\Lim}[1]{\lim_{#1 \rightarrow \infty}}
\newcommand{\LimInf}[1]{\liminf_{#1 \rightarrow \infty}}
\newcommand{\LimSup}[1]{\limsup_{#1 \rightarrow \infty}}

\newcommand{\Abs}[1]{\left\vert #1 \right\vert}
\newcommand{\Norm}[1]{\left\Vert #1 \right\Vert}
\newcommand{\Floor}[1]{\left\lfloor #1 \right\rfloor}
\newcommand{\Ceil}[1]{\left\lceil #1 \right\rceil}
\newcommand{\Chev}[1]{\left\langle #1 \right\rangle}
\newcommand{\Quote}[1]{\left\ulcorner #1 \right\urcorner}

\newcommand{\K}{\xrightarrow{\textnormal{k}}}
\newcommand{\PF}{\xrightarrow{\circ}}

% Paper specific

\newcommand{\A}{\mathcal{A}}
\newcommand{\St}{\mathcal{S}}
\newcommand{\T}{\mathcal{T}}
\newcommand{\R}{\mathcal{R}}
\newcommand{\Pe}{P}

\newcommand{\Ut}{\operatorname{U}}
\newcommand{\Co}{C}
\newcommand{\V}{\operatorname{V}}
\newcommand{\QV}{\operatorname{QV}}
\DeclareMathOperator{\Hi}{H}
\DeclareMathOperator{\Z}{Z}

\begin{document}

[Earlier](https://agentfoundations.org/item?id=1785) we established that the quantilal policy can be computed in polynomial time to any given approximation (see "Proposition 5"). Now we show that an *exact* quantilal policy can be computed in polynomial time (in particular there is always a rational quantilal policy).

We assume geometric time discount throughout.

\#Lemma

Consider $\xi \in \Delta\St$. Define the linear operators $E: \Reals^{\St \times \A} \rightarrow \Reals^\St$ and $T: \Reals^{\St \times \A} \rightarrow \Reals^\St$ by

$$E_{t,sa} := [[t = s]]$$

$$T_{t,sa} := \T\APM{t}{s,a}$$

(Note that this $T$ is the *transpose* of the $T$ defined in "Proposition A.3" of the previous essay.)

Then, $\xi \in \Img{\Z}$ if and only if there is $\phi \in \Delta(\St\times\A)$ s.t.

$$E\phi = \xi$$

$$(E-\lambda T) \phi = (1-\lambda)\zeta_0$$

\#Proof of Lemma

(This is actually well known, but we spell out the proof to be self-contained.)

Suppose that $\xi \in \Img{\Z}$. We already know that this implies that there is a *stationary* policy $\pi:\St\K\A$ s.t. $\Z\pi = \xi$ (we abuse notation in the obvious way): see the proofs of "Proposition 2" and "Proposition 3". Define the linear operator $T^\pi: \Reals^\St \rightarrow \Reals^\St$ by

$$T_{ts}^\pi := \Ea{a\sim\pi(s)}{\T\APM{t}{s,a}}$$

It follows that 

$$\xi = (1-\lambda)\sum_{n=0}^\infty \lambda^n T^{\pi n} \zeta_0  = (1-\lambda)\AP{\boldsymbol{1}-\lambda T^\pi}^{-1} \zeta_0$$

$$\AP{\boldsymbol{1}-\lambda T^\pi}\xi = (1-\lambda)\zeta_0$$

Define $\phi$ by

$$\phi(s,a) := \xi(s) \pi(a \mid s)$$

We have

$$T^\pi\xi = \sum_{s\in\St} \Ea{a\sim\pi(s)}{\T(s,a)} \xi(s) = \sum_{\substack{s\in\St\\a\in\A}}\pi(a \mid s)\T(s,a)\xi(s) = \sum_{\substack{s\in\St\\a\in\A}} \T(s,a) \phi(s,a) = T\phi$$

Also, obviously $E\phi = \xi$. We get

$$(E-\lambda T)\phi = \xi - \lambda T^\pi\xi = \AP{\boldsymbol{1}-\lambda T^\pi}\xi = (1-\lambda) \zeta_0$$

Conversely, suppose that $\phi$ is as above. Since $E\phi=\xi$, there is $\pi: \St \K \A$ s.t. for any $s\in\St$, if $\xi(s) \ne 0$ then

$$\pi(a \mid s) = \frac{\phi(s,a)}{\xi(s)}$$

Again, we have

$$\Z\pi = (1-\lambda)\AP{\boldsymbol{1} - \lambda T^\pi}^{-1} \zeta_0$$

Also, for the same reason as before

$$(E - \lambda T)\phi = \AP{\boldsymbol{1}-\lambda T^\pi}\xi$$

By the assumption, the left hand side equals $(1-\lambda) \zeta_0$. We conclude

$$\xi = (1-\lambda) \AP{\boldsymbol{1} - \lambda T^\pi}^{-1} \zeta_0 = \Z \pi$$

\#Theorem

Assuming all parameters are rational like before, there is a polynomial time algorithm that computes a quantilal policy.

\#Proof

The algorithm starts by solving the following linear program. The indeterminates are $\phi \in \Reals^{\St\times\A}$ and $\QV\in\Reals$. The goal is maximizing $\QV$. The constraints are

$$\forall s\in\St,a\in\A: \phi(s,a) \geq 0$$

$$\sum_{\substack{s\in\St \\ a\in\A}} \phi(s,a) = 1$$

$$(E - \lambda T) \phi = (1-\lambda) \zeta_0$$

$$\forall s \in \St\setminus\Supp{\Z\sigma},a\in\A: \phi(s,a) = 0$$

$$\forall s \in \Supp{\Z\sigma}: \QV \leq \sum_{t\in\St} \R(t)\sum_{a\in\A}\phi(t,a) - \frac{\eta}{\Z\sigma(s)} \sum_{a\in\A}\phi(s,a)$$

Then, the algorithm computes $\pi: \St \K \A$ s.t. for any $s\in\St$, if $\sum_{b\in\A}\phi(s,b) > 0$ then

$$\pi(a \mid s) := \frac{\phi(s,a)}{\sum_{b\in\A}\phi(s,b)}$$

For $s\in\St$ s.t. $\sum_{b\in\A}\phi(s,b) = 0$, $\pi(s)$ is arbitrary.

Now we need to explain why this algorithm is correct.

Observe that, the first 3 constraints mean that $\xi\in\Reals^\St$ defined by $\xi(s) := \sum_{b\in\A} \phi(s,b)$ lies in $\Img{\Z}$ (by Lemma 1) and, moreover, $\phi(s,a) = \xi(s)\pi(a \mid s)$ for $\pi:\St\K\A$ s.t. $\xi = \Z\pi$. It remains to show that the linear program amounts to maximizing $\Ea{}{\R} - \eta\exp\RD{\infty}{\xi}{\Z\sigma}$ inside $\Img\Z$. Indeed, the 4th constraint just means that $\RD{\infty}{\xi}{\Z\sigma} < \infty$. The last constraint implies that we are actually maximizing

$$\min_{s\in\Supp\Z\sigma} \AP{\sum_{t\in\St} \R(t)\xi(t) - \frac{\eta}{\Z\sigma(s)} \xi(t)}$$

The latter is indeed $\Ea{}{\R} - \eta\exp\RD{\infty}{\xi}{\Z\sigma}$, since every $s\in\Supp{\Z\sigma}$ corresponds to a pure strategy of the adversary in the corresponding zero-sum game: namely, this strategy is setting the penalty function $P: \St \rightarrow [0,\infty)$ to

$$P(t) = \frac{[[t=s]]}{\Z\sigma(s)}$$

(Strategies that place non-vanishing penalty on states outside of $\Supp{\Z\sigma}$ become irrelevant after imposing the 4th constraint. The remaining penalty functions form a simplex with vertices as above.)

\end{document}



