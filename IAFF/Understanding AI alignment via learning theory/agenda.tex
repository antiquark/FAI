%&latex
\documentclass[a4paper]{article}

\usepackage[a4paper,margin=1in]{geometry}
\usepackage[affil-it]{authblk}
\usepackage{cite}
\usepackage[unicode]{hyperref}
\usepackage[utf8]{inputenc}
\usepackage[english]{babel}
\usepackage{csquotes}
\usepackage{amsmath,amssymb,amsthm}
\usepackage{enumerate}
\usepackage{commath}

\newcommand{\Comment}[1]{}

% operators that are separated from the operand by a space
\DeclareMathOperator{\Sgn}{sgn}
\DeclareMathOperator{\Supp}{supp}
\DeclareMathOperator{\Dom}{dom}

% autosize delimiters
\newcommand{\AP}[1]{\left(#1\right)}
\newcommand{\AB}[1]{\left[#1\right]}
\newcommand{\AC}[1]{\left\{#1\right\}}
\newcommand{\APM}[2]{\left(#1\;\middle\vert\;#2\right)}
\newcommand{\ABM}[2]{\left[#1\;\middle\vert\;#2\right]}
\newcommand{\ACM}[2]{\left\{#1\;\middle\vert\;#2\right\}}

% operators that require brackets
\newcommand{\Pa}[2]{\underset{#1}{\operatorname{Pr}}\AB{#2}}
\newcommand{\CP}[3]{\underset{#1}{\operatorname{Pr}}\ABM{#2}{#3}}
\newcommand{\PP}[2]{\underset{\substack{#1 \\ #2}}{\operatorname{Pr}}}
\newcommand{\PPP}[3]{\underset{\substack{#1 \\ #2 \\ #3}}{\operatorname{Pr}}}
\newcommand{\E}[1]{\underset{#1}{\operatorname{E}}}
\newcommand{\Ea}[2]{\underset{#1}{\operatorname{E}}\AB{#2}}
\newcommand{\CE}[3]{\underset{#1}{\operatorname{E}}\ABM{#2}{#3}}
\newcommand{\EE}[2]{\underset{\substack{#1 \\ #2}}{\operatorname{E}}}
\newcommand{\EEE}[3]{\underset{\substack{#1 \\ #2 \\ #3}}{\operatorname{E}}}
\newcommand{\Var}{\operatorname{Var}}
\newcommand{\I}[1]{\underset{#1}{\operatorname{I}}}
\newcommand{\CI}[3]{\underset{#1}{\operatorname{I}}\ABM{#2}{#3}}
\newcommand{\Ia}[2]{\underset{#1}{\operatorname{I}}\AB{#2}}
\newcommand{\II}[2]{\underset{\substack{#1 \\ #2}}{\operatorname{I}}}
\newcommand{\III}[3]{\underset{\substack{#1 \\ #2 \\ #3}}{\operatorname{I}}}

% operators that require parentheses
\newcommand{\En}{\operatorname{H}}
\newcommand{\Ena}[1]{\operatorname{H}\AP{#1}}
\newcommand{\PS}[1]{\mathcal{P}\AP{#1}}

\newcommand{\D}{\mathrm{d}}
\newcommand{\KL}[2]{\operatorname{D}_{\mathrm{KL}}\AP{#1\middle\vert\middle\vert#2}}
\newcommand{\RD}[3]{\operatorname{D}_{#1}\AP{#2\middle\vert\middle\vert#3}}
\newcommand{\Dtv}{\operatorname{d}_{\textnormal{tv}}}
\newcommand{\Dtva}[1]{\operatorname{d}_{\textnormal{tv}}\AP{#1}}

\newcommand{\Argmin}[1]{\underset{#1}{\operatorname{arg\,min}}\,}
\newcommand{\Argmax}[1]{\underset{#1}{\operatorname{arg\,max}}\,}

\newcommand{\Nats}{\mathbb{N}}
\newcommand{\Ints}{\mathbb{Z}}
\newcommand{\Rats}{\mathbb{Q}}
\newcommand{\Reals}{\mathbb{R}}
\newcommand{\Coms}{\mathbb{C}}

\newcommand{\Estr}{\boldsymbol{\lambda}}

\newcommand{\Lim}[1]{\lim_{#1 \rightarrow \infty}}
\newcommand{\LimInf}[1]{\liminf_{#1 \rightarrow \infty}}
\newcommand{\LimSup}[1]{\limsup_{#1 \rightarrow \infty}}

\newcommand{\Abs}[1]{\left\vert #1 \right\vert}
\newcommand{\Norm}[1]{\left\Vert #1 \right\Vert}
\newcommand{\Floor}[1]{\left\lfloor #1 \right\rfloor}
\newcommand{\Ceil}[1]{\left\lceil #1 \right\rceil}
\newcommand{\Chev}[1]{\left\langle #1 \right\rangle}
\newcommand{\Quote}[1]{\left\ulcorner #1 \right\urcorner}

\newcommand{\K}{\xrightarrow{\textnormal{k}}}
\newcommand{\PF}{\xrightarrow{\circ}}

% Paper specific

% TBD

\begin{document}

In this essay I will try to explain the overall structure and motivation of my AI alignment research agenda. The discussion is informal and no new theorems are proved here. The main features of my research agenda, as I explain them here, are 

* Viewing AI alignment theory as part of a general abstract theory of intelligence

* Using desiderata and axiomatic definitions as starting points, rather than specific algorithms and constructions

* Formulating alignment problems in the language of learning theory

* Evaluating solutions by their formal mathematical properties, ultimately aiming at a quantitative theory of risk assessment

* Relying on the mathematical intuition derived from learning theory to pave the way to solving philosophical questions

\section{Philosophy}

In this section I explain the key principles and assumptions that motivate my research agenda.

\#The importance of rigor

I believe that the solution to AI alignment must rely on a rigorous mathematic theory. The algorithms that comprise the solution must be justified by formal mathematic properties. All mathematical assumptions should be either proved or at least backed by considerable evidence, like the prominent conjectures of computational complexity theory. This needs to be the case because:

* We might be facing one-shot success or failure. This means we will have little empirical backing for our assumptions.

* To the extent we have or will have empirical evidence about AI, without a rigorous underlying theory it is very hard to know how scalable and transferable the conclusions are.

* The enormity of the stakes demands designing a solution which is as reliable as possible, limited only by the time constraints imposed by competing unaligned projects.

That said, I do expect the ultimate solution to have aspects that are not entirely rigorous, specifically:

* The quantitative risk analysis will probably rely on some parameters that will be very hard to determine from first principles, because of the involvement of humans and our physical universe in the equation. These parameters might be estimated through (i) study of the evolution of intelligence (ii) study of human brains, (iii) experiments with weak AI and its interaction with humans (iv) our understanding of physics. Nevertheless, we should demand the solution to be highly reliable even given cautious error margins on these parameters.

* The ultimate solution will probably involve some heuristics.\ However, it should only involve heuristics that are designed to improve AI capabilities *without invalidating any of the assumptions underlying the risk analysis.* Thus, in the worst-case scenario these heuristics will fail and the AI will not take off but will not become unaligned.

* *In addition* to the theoretical analysis, we do want to include as much empirical testing as possible, to provide an additional layer of defense. At the least, it can be a last ditch protection in the (hopefully very unlikely) scenario that some error got through the analysis.

\#Metaphilosophy and the role of models

In order to use mathematics to solve a real-world problem, a mathematical *model* of the problem must be constructed. When the real-world problem can be defined in terms of data that is observable and measurable, the validity of the mathematical model can be ascertained using the empirical method. However, AI alignment touches on problems that are *philosophical* in nature, meaning that there is still no agreed-upon empirical or other criterion for evaluating an answer. Dealing with such problems requires a *metaphilosophy*: a way of evaluating answers to philosophical questions.

Although I do not claim a fully general solution to metaphilosophy, I think that, pragmatically, a quasiscientific approach is possible. In science, we prefer theories that are (i) simple (Occam's razor) and (ii) fit the empirical data. We also test theories by gathering further empirical data. In philosophy, we can likewise prefer theories that are (i) simple and (ii) fit intuition in situations where intuition feels reliable (i.e. situations that are simple, familiar or received considerable analysis and reflection). We can also test theories by applying them to new situations and trying to see whether the answer becomes intuitive after sufficient reflection.

Moreover, I expect progress on most problems to be achieved by the means of successive approximations. This means that we start with a model that is grossly oversimplified but that already captures some key aspects of the problems. Once we have a solution within this model, we can start to attack its assumptions and arrive at a new, more sophistical model. This process should repeat until we arrive at a model that (i) has no *obvious* shortcomings and that (ii) we seem unable to improve despite our best efforts. 

Like in science, we can never be certain that a theory is true. *Any* assumption or model can be questioned. This requires striking a balance between complacency and excessive skepticism. To avoid complacency, we need to keep working to find better theories. To avoid excessive skepticism, we should entertain hypotheses honestly and acknowledge when a theory is already capable of passing non-trivial quasiscientific tests. Reaching agreement is harder work (because our tests rely on intuition which may vary from individual to individual), but we should not despair of that goal.

\#Intelligence is understandable

It is possible question whether a mathematical theory of intelligence is possible at all. After all, we don't expect to have a tractable mathematical theory of Rococo architecture, or a simple equation describing the shape of the coastline of Africa in the year 2018.

The key difference is that intelligence is a *natural* concept. Intelligence, the way I use this word in the context of AI alignment, is the ability of an agent to make choices in a way that effectively promote its goals, in an environment that is not entirely known or even not entirely knowable. Arguing over the meaning of the word would be a distraction: this is the meaning relevant to AI alignment, because the entire concern of AI alignment is about agents that effectively pursue their goals, undermining the conflicting goals of the human species. Moreover, intelligence is (empirically) a key force in determining the evolution of the physical universe.

I conjecture that natural concepts have useful mathematical theories, and this conjecture seems to me supported by evidence in natural and computer science. It would be nice to have this conjecture itself follow from a mathematical theory, but this is outside of my current scope. Moreover, we already *have* some progress towards a mathematical theory of intelligence (I will discuss it in the next section).

A related question is, whether it is possible to design an algorithm for strong AI based on simple mathematical principles, or whether any strong AI will inevitable be an enormous kludge of heuristics designed by trial and error. I think that we have some empirical evidence support for the former, given that humans evolved to survive in a certain environment but succeeded to use their intelligence to solve problems in very different environments. That said, I am less confident about this than about the previous question. In any case, having a mathematical theory of intelligence should allow us to resolve this question too, whether positively or negatively.

\#Value alignment is understandable

The core of AI alignment is reliably transferring human values to a strong AI. However, the problem of defining what we mean by "human values" is a philosophical problem. A common and natural model of "values" is expected utility maximization: this is what we find in game theory and economics, and this is supported by VNM and Savage theorems. However, as often pointed out, humans are not perfectly rational, therefore it's not clear in what sense they can be said to maximize the expectation of a specific utility function.

Nevertheless, I believe that "values" is also a natural concept. Denying the concept of "values" altogether is paramount to nihilism, and in such a framework there is no reason to do anything at all, including saving yourself and everyone else from a murderous AI. Admitting the *general* concept of "values" as something complex and human specific (despite the focus on "values" rather than "human values") seems implausible, since intuitively we can easily imagine alien minds facing a similar AI alignment problem. Moreover, the concept of "values" is part and parcel of the concept of "intelligence", so if we believe that intelligence (due to its importance in shaping the physical world) is a natural concept, then so are values.

Therefore, I conjecture that there is a simple mathematical theory of imperfect rationality, within which the concept of "human values" is well-defined modulo the (observable, measurable) concept of "humans". Some speculation on what this theory looks like appears in the following sections.

Now, that doesn't mean that "human values" are *perfectly* well-defined, anymore than, for example, the center of mass of the sun is perfectly well-defined (which would require deciding exactly which particles are considered part of the sun). However, like the center of mass of the sun is sufficiently well-defined for many practical purposes in astrophysics, the concept of "human values" should be sufficiently well-defined for designing an aligned AI. To the extent alignment remains ambiguous, the resolution of these ambiguities doesn't have a moral significance.

\section{Foundations}

In this section I briefly explain the mathematical tools with which I set out to study AI alignment, and the outline of the mathematical theory of intelligence that these tools already painted.

\#Statistical Learning Theory

Statistical learning theory studies the information-theoretic constraints on various types of learning tasks, answering questions such as, when is a learning task solvable at all, and how much training data is required to solve the learning task within given accuracy (sample complexity). Learning tasks can be broadly divided into:

* Classifications tasks: The input is sampled from a fixed probability distribution, and the objective is assigning the correct label. The deployment phase (during which the performance of the algorithm is evaluated) is distinct from the training phase (during which the correct labels are revealed).

* Online learning / multi-armed bandits: There is no distinction between deployment and training. Instead, the algorithm's performance on each round is evaluated, but also the algorithm might receive some feedback on its performance. The behavior of the environment might change over time, possibly even respond to the algorithm's output. However, we only evaluate each output conditioned on the past history (we don't require the algorithm to plan ahead).

* Reinforcement learning: There is two-sided interaction between the algorithm and the environment, and the algorithm's performance is the aggregate of some reward function over time. The algorithm *is* required to plan ahead in order to achieve optimal performance. This might or might not assume "resets" (when the environment periodically returns to the initial state) or partition of time into "episodes"\ (when the performance of the algorithm is only evaluated conditioned on the previous episodes, so that it doesn't have to plan ahead more than one episode into the future).

It is the last type of learning tasks, in particular assuming no resets or episodes, that is the most relevant for studying intelligence in the relevant sense. Indeed, the abstract setting of reinforcement learning is a good formalization for the informal definition of intelligence we had before. Note that the name "reward" might be misleading: this is not necessarily a signal received from outside, but can just as easily be some formally specified mathematical function.

In online learning and reinforcement learning, the theory typically aims to derive upper and lower bounds on "regret": the difference between the expected utility received by the algorithm and the expected utility it *would* receive if the environment was known a priori. Such an upper bound is effectively a *performance guarantee* for the given algorithm. In particular, if the reward function is assumed to be "aligned" then this performance guarantee is, to some extent, an alignment guarantee. This observation is not vacuous, since the learning protocol might be such that the true reward function is not directly available to the algorithm, as exemplified by [DIRL](https://agentfoundations.org/item?id=1550) and [DRL](https://agentfoundations.org/item?id=1656). Thus, formally proving alignment guarantees takes the form of proving appropriate regret bounds.

\#Computational Learning Theory

In addition to information-theoretic considerations, we have to take into account considerations of computational complexity. Thus, after deriving information-theoretic regret bounds, we should continue to refine them by constraining our algorithms to be computationally feasible (which typically means running on polynomial time, but we may also need to consider stronger restrictions, such as restrictions on space complexity or parallelizability). If we consider *Bayesian* regret (i.e. the expected value of regret w.r.t. some prior on the environments), this effectively means we are dealing with *average-case* complexity. Note that, imposing computational constraints on the agent implies bounded reasoning / non-omniscience and constitutes departure from "perfect rationality" in a certain sense.

In fact, the current gap in our theoretical understanding of deep learning is mostly of complexity-theoretic nature. Indeed, results about expressiveness and (statistical) learnability of neural networks are well-known, however exact learning of neural networks is NP-complete in the general case. Understanding how this computational barrier is circumvented in practical problems is a key challenge in understanding deep learning. Such understanding would probably be a positive development in terms of AI risk (although it might also contribute to increasing AI\ capacity), but I don't think it's a high priority problem since it seems to already receive considerable attention in mainstream academia (i.e. it is not neglected).

I believe that the development of AI alignment theory should start focusing on the information-theoretical layer, for the most part proceeding to complexity-theoretic analysis only later. That said, we should keep the complexity-theoretic considerations in mind, and strive to devise solutions that at least seem feasible modulo "miracles" similar to deep learning (i.e. modulo intractable problems that are plausibly tractable in realistic special cases). Moreover, certain complexity-theoretic considerations are already implicit in the choice of the space of hypotheses for your learning problem. In particular, we should keep in mind that the hypotheses must be computationally simpler than the agent itself, whereas the universe must be computationally more complex than the agent itself. More on resolving this apparent paradox later.

\#Algorithmic Information Theory

The choice of hypothesis space plays a crucial role in any learning task, and the choice of prior plays a crucial role in Bayesian reinforcement learning. In *narrow* AI this choice is based entirely on the prior knowledge of the AI designers about the problem. On the other hand, *general* AI should be able to learn its environment with little prior knowledge, by noticing patterns and using Occam's razor. Indeed, the latter the basis of epistemic rationality to the best of understanding. The Solomonoff measure is an elegant formalization of this idea.

However, Solomonoff induction is incomputable, so a realistic agent would have to use some truncated form of it, for example by bounding the computational resources made available to the universal Turing machine. It thus becomes an important problem to find a natural prior such that:

* It allows for a (sufficiently good) sublinear regret bound with a computationally feasible algorithm.

* It ranks hypotheses by description complexity in some appropriate sense.

* It satisfies some universality properties analogous to the Solomonoff measure (but appropriately weaker).

\#Towards a rigorous definition of intelligence

The combination of perfect Bayesian reinforcement learning and the Solomonoff prior is known as AIXI. AIXI may be regarded as a model ideal intelligence, but there are several issues that were argued to be flaws in this concept:

* Traps: AIXI doesn't satisfy any interesting regret bounds, because the environment might contain traps. In fact, the set of all computable environments is an *unlearnable* class of hypotheses: no agent has a sublinear regret bound w.r.t. this class.

* Cartesian duality: AIXI's "reasoning" (and RL in general) seems to assume the environment cannot influence the algorithm executed by the agent. This is unrealistic. For example if our agent is a robot, then it's perfectly possible to imagine some external force breaking into its computer and modifying its software.

* Irreflexivity: The Solomonoff measure contains only computable hypotheses but the agent itself is uncomputable. In particular, AIXI can satisfy no guarantees pertaining to environments that e.g. contain other AIXIs. An analogous problem persists with any simple attempt to modify the prior: the prior can only contain hypotheses simpler than the agent.

* Decision-theoretic paradoxes: AIXI seems to be similar to a Causal Decision\ Theorist so apparently it will fail on Newcomb-like problems.

The Cartesian duality problem and the traps problem are actually strongly related. Indeed, one can model any event that destroys the agent (including modifying its source code) as the transition of the environment into some inescapable state. Such a state should be assigned a reward that corresponds to the expected utility of the universe going on without the agent. However, it's not obvious how the agent can learn to anticipate such states, since observing it once eliminates any chance of using this knowledge later. DRL already partially addresses this problem: more discussion in the next section.

Solving Irreflexivity requires going beyond the Bayesian paradigm by including models that don't fully specify the environment. More details in the next section.

Finally, the decision-theoretic paradoxes are a more equivocal issue than it seems, because the usual philosophical way of thinking about decision theory assumes that the model of the environment is *given*, whereas in our way of thinking, the model is *learned*. This is important: for example, if AIXI is placed in a repeated Newcomb's problem, it will learn to one-box, since its model will predict that one-boxing *causes* the money to appear inside the box. In other words, AIXI might be regarded as a CDT, but the learned "causal" relationships are not the same as physical causality. Formalizing other Newcomb-like problems require solving Irreflexivity first, because the environment contains Omega which cannot be simulated by the agent. Therefore, my current working hypothesis is that decision theory will be mostly solved (or dissolved) by

* Solving Irreflexivity

* Value learning will automatically learn some aspects of the decision theory too.

* Allowing for self-modification, which should be possible after solving Irreflexivity + Cartesian duality (self-modification may be again be regarded as a terminal state)

To sum up, clarifying all of these issues should result in formulating a certain optimality condition (regret bound) which may be regarded as a rigorous definition of intelligence. This would also constitute progress towards defining "values" (having certain values means being intelligent w.r.t. these values), but the latter might require making the definition even more lax. More on that later.

\section{Research Programme Outline}

In this section I break down the research programme into different domains and subproblems. The list below is not intended to be a linear sequence. Indeed, many of the subproblems can be initially attacked in parallel, but also many of them are interconnected and progress in one subproblem can be leveraged to produce a more refined analysis of another. Any concrete plan I have regarding the order with which these questions should be addressed is liable to change significantly as progress is made. Moreover, I expect the entire breakdown to change as progress is made and new insights are available. However, I do believe that the high-level principles of the approach have a good chance of surviving, in some form, into the future.

\#Value learning protocols

At present, I conceive of the following possible basic mechanisms for value learning:

* Formal communication: Information about the values is communicated to the agent in a form with pre-defined formal semantics. Examples of this are, communicating a full formal specification of the utility function or manually producing a reward signal. Other possibilities are, communicating partial information about the reward signal, or evaluating particular hypothetical situations.

* Informal communication: Information about the values is communicated to the agent using natural language or other form whose semantics have to be learned somehow.

* Demonstration: The agent observes a human pursuing eir values and deduces the values from the behavior.

* Reverse engineering: The agent somehow acquires a full formal specification of a human (e.g. an uploaded brain) and deduces the values from this specification.

Formal communication is difficult because human values are complicated and describing them precisely is hard. A manual reward signal is more realistic than a full specification, but:

* Still difficult to produce, especially if this reward is support to reflect the true aggregate of the human's values as observed by em across the universe (which is what it would have to be in order to aim at the true human utility function).

* The reward signal will become erroneous if the human or just the communication channel between the human and the agent will be corrupted in some way. This is serious problem since, if not taken into account, it incentives the agent to produce such corruption.

* If the agent is supposed to aim at a very long-term goal, there might be little to no relevant information in the reward signal until the goal is attained.

Overall, it might be more realistic to rely on formal communication for tasks of limited scope (putting a strawberry on plate) rather than actually learning human values in full. However, it is also possible to combine several mechanisms in a single protocol, and formal communication might be only one of them.

The problem of corruption may be regarded as a special cases of the problem of traps (the latter was outlined in the previous section), if we assume that the agent is expected to achieve its goals without entering corrupt states. [Delegative Reinforcement Learning](https://agentfoundations.org/item?id=1656) aims to solve both problems by occasionally passing control to the human operator ("advisor"), and using it to learn which actions are safe. The analysis of DRL that I produced so far can and should be improved in multiple ways:

* Instead of considering only a finite or countable set of hypotheses, we should consider a space of hypotheses of finite "dimension" (for some appropriate notion of dimension; it is a common theme in statistical learning theory that different learning setups have different natural notions of dimensionality for hypothesis classes). We then should obtain a regret bounded depending on the dimension and the entropy of the prior. I believe that I already have significant progress on this point, with results expected soon.

* Merge the ideas of [quantilal control](https://agentfoundations.org/item?id=1785) and [catastrophe mitigation](https://agentfoundations.org/item?id=1715) to yield a setting where, corruption is a quantitative/gradual rather than boolean and there is a low but non-vanishing rate of corruption along the advisor policy. Successful catastrophe mitigation will be achieved if it is possible without high Renyi divergence from the advisor policy, in some sense.

* In the current form DRL requires the advisor to be ready to act instead of the agent at any given time moment. If the temporal rate at which actions are taken is high, this is an unrealistic demand. Naively, we can solve this by dividing time into intervals, and considering the policy on each interval as a single action. However, this would introduce an exponential penalty into the regret bound. Therefore, a more sophisticated way to manage the scheduling of control between the agent and the advisor is required, with a corresponding regret bound.

* Instead of only considering MDPs with a finite state space, we can consider e.g. Feller continuous MDPs with a compact state space, and/or POMDPs.

The demonstration mechanism avoids some of the difficulties with formal communication, but introduces its own drawbacks...

TBD

\#Universal reinforcement learning

TBD

\#Taming daemons

TBD

\#Recursive self-improvement

TBD

\section{Summary}

% Summarize the main points and relate them to the abstract

TBD

\end{document}



