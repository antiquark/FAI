%&latex
\documentclass[a4paper]{article}

\usepackage[a4paper,margin=1in]{geometry}
\usepackage[affil-it]{authblk}
\usepackage{cite}
\usepackage[unicode]{hyperref}
\usepackage[utf8]{inputenc}
\usepackage[english]{babel}
\usepackage{csquotes}
\usepackage{amsmath,amssymb,amsthm}
\usepackage{enumerate}
\usepackage{commath}

\newcommand{\Comment}[1]{}

% operators that are separated from the operand by a space
\DeclareMathOperator{\Sgn}{sgn}
\DeclareMathOperator{\Supp}{supp}
\DeclareMathOperator{\Dom}{dom}

% autosize delimiters
\newcommand{\AP}[1]{\left(#1\right)}
\newcommand{\AB}[1]{\left[#1\right]}
\newcommand{\AC}[1]{\left\{#1\right\}}
\newcommand{\APM}[2]{\left(#1\;\middle\vert\;#2\right)}
\newcommand{\ABM}[2]{\left[#1\;\middle\vert\;#2\right]}
\newcommand{\ACM}[2]{\left\{#1\;\middle\vert\;#2\right\}}

% operators that require brackets
\newcommand{\Pa}[2]{\underset{#1}{\operatorname{Pr}}\AB{#2}}
\newcommand{\CP}[3]{\underset{#1}{\operatorname{Pr}}\ABM{#2}{#3}}
\newcommand{\PP}[2]{\underset{\substack{#1 \\ #2}}{\operatorname{Pr}}}
\newcommand{\PPP}[3]{\underset{\substack{#1 \\ #2 \\ #3}}{\operatorname{Pr}}}
\newcommand{\E}[1]{\underset{#1}{\operatorname{E}}}
\newcommand{\Ea}[2]{\underset{#1}{\operatorname{E}}\AB{#2}}
\newcommand{\CE}[3]{\underset{#1}{\operatorname{E}}\ABM{#2}{#3}}
\newcommand{\EE}[2]{\underset{\substack{#1 \\ #2}}{\operatorname{E}}}
\newcommand{\EEE}[3]{\underset{\substack{#1 \\ #2 \\ #3}}{\operatorname{E}}}
\newcommand{\Var}{\operatorname{Var}}
\newcommand{\I}[1]{\underset{#1}{\operatorname{I}}}
\newcommand{\CI}[3]{\underset{#1}{\operatorname{I}}\ABM{#2}{#3}}
\newcommand{\Ia}[2]{\underset{#1}{\operatorname{I}}\AB{#2}}
\newcommand{\II}[2]{\underset{\substack{#1 \\ #2}}{\operatorname{I}}}
\newcommand{\III}[3]{\underset{\substack{#1 \\ #2 \\ #3}}{\operatorname{I}}}

% operators that require parentheses
\newcommand{\En}{\operatorname{H}}
\newcommand{\Ena}[1]{\operatorname{H}\AP{#1}}
\newcommand{\PS}[1]{\mathcal{P}\AP{#1}}

\newcommand{\D}{\mathrm{d}}
\newcommand{\KL}[2]{\operatorname{D}_{\mathrm{KL}}\AP{#1\middle\vert\middle\vert#2}}
\newcommand{\RD}[3]{\operatorname{D}_{#1}\AP{#2\middle\vert\middle\vert#3}}
\newcommand{\Dtv}{\operatorname{d}_{\textnormal{tv}}}
\newcommand{\Dtva}[1]{\operatorname{d}_{\textnormal{tv}}\AP{#1}}

\newcommand{\Argmin}[1]{\underset{#1}{\operatorname{arg\,min}}\,}
\newcommand{\Argmax}[1]{\underset{#1}{\operatorname{arg\,max}}\,}

\newcommand{\Nats}{\mathbb{N}}
\newcommand{\Ints}{\mathbb{Z}}
\newcommand{\Rats}{\mathbb{Q}}
\newcommand{\Reals}{\mathbb{R}}
\newcommand{\Coms}{\mathbb{C}}

\newcommand{\Estr}{\boldsymbol{\lambda}}

\newcommand{\Lim}[1]{\lim_{#1 \rightarrow \infty}}
\newcommand{\LimInf}[1]{\liminf_{#1 \rightarrow \infty}}
\newcommand{\LimSup}[1]{\limsup_{#1 \rightarrow \infty}}

\newcommand{\Abs}[1]{\left\vert #1 \right\vert}
\newcommand{\Norm}[1]{\left\Vert #1 \right\Vert}
\newcommand{\Floor}[1]{\left\lfloor #1 \right\rfloor}
\newcommand{\Ceil}[1]{\left\lceil #1 \right\rceil}
\newcommand{\Chev}[1]{\left\langle #1 \right\rangle}
\newcommand{\Quote}[1]{\left\ulcorner #1 \right\urcorner}

\newcommand{\K}{\xrightarrow{\textnormal{k}}}
\newcommand{\PF}{\xrightarrow{\circ}}

% Paper specific

% TBD

\begin{document}

In this essay I will try to explain the overall structure and motivation of my AI alignment research agenda. The discussion is informal and no new theorems are proved here. The main features of my research agenda, as I explain them here, are 

* Viewing AI alignment theory as part of a general abstract theory of intelligence

* Using desiderata and axiomatic definitions as starting points, rather than specific algorithms and constructions

* Formulating alignment problems in the language of learning theory

* Evaluating solutions by their formal mathematical properties, ultimately aiming at a quantitative theory of risk assessment

* Relying on the mathematical intuition derived from learning theory to pave the way to solving philosophical questions, to the extent these questions need to be addressed at all

\section{Philosophy}

\#The importance of rigor

I believe that the solution to AI alignment must rely on a rigorous mathematic theory. The algorithms that comprise the solution must be justified by formal mathematic properties. All mathematical assumptions should be either proved or at least backed by considerable evidence, like the prominent conjectures of computational complexity theory. This needs to be the case because:

* We might be facing one-shot success or failure. This means we will have little empirical backing for our assumptions.

* To the extent we have or will have empirical evidence about AI, without a rigorous underlying theory it is very hard to know how scalable and transferable the conclusions are.

* The enormity of the stakes demands designing a solution which is as reliable as possible, limited only by the time constraints imposed by competing unaligned projects.

That said, I do expect the ultimate solution to have aspects that are not entirely rigorous, specifically:

* The quantitative risk analysis will probably rely on some parameters that will be very hard to determine from first principles, because of the involvement of humans and our physical universe in the equation. These parameters might be estimated through (i) study of the evolution of intelligence (ii) study of human brains, (iii) restricted experiments with algorithms and AI and (iv) our understanding of physics. Nevertheless, we should demand the solution to be highly reliable even given cautious error margins on these parameters.

* The ultimate solution will probably involve some heuristics.\ However, it should only involve heuristics that are designed to improve AI capabilities *without invalidating any of the assumptions underlying the risk analysis.* Thus, in the worst-case scenario these heuristics will fail and the AI will not take off but will not become unaligned.

* *In addition* to the theoretical analysis, we do want to include as much empirical testing as possible, to provide an additional layer of defense. At the least, it can be a last ditch protection in the (hopefully very unlikely) scenario that some error got through the analysis.

\#Metaphilosophy and the role of models

TBD

\#Intelligence is understandable

TBD

\#Value alignment is understandable

TBD

\section{Foundation}

% Explain your current understanding of intelligence and alignment

TBD

\section{Research Programme Outline}

% The "work plan" for the future

TBD

\section{Summary}

% Summarize the main points and relate them to the abstract

TBD

\end{document}



