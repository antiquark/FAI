%&latex
\documentclass[a4paper]{article}

\usepackage[a4paper,margin=1in]{geometry}
\usepackage[affil-it]{authblk}
\usepackage{cite}
\usepackage[unicode]{hyperref}
\usepackage[utf8]{inputenc}
\usepackage[english]{babel}
\usepackage{csquotes}
\usepackage{amsmath,amssymb,amsthm}
\usepackage{enumerate}
\usepackage{commath}

\newcommand{\Comment}[1]{}

% operators that are separated from the operand by a space
\DeclareMathOperator{\Sgn}{sgn}
\DeclareMathOperator{\Supp}{supp}
\DeclareMathOperator{\Dom}{dom}

% autosize delimiters
\newcommand{\AP}[1]{\left(#1\right)}
\newcommand{\AB}[1]{\left[#1\right]}
\newcommand{\AC}[1]{\left\{#1\right\}}
\newcommand{\APM}[2]{\left(#1\;\middle\vert\;#2\right)}
\newcommand{\ABM}[2]{\left[#1\;\middle\vert\;#2\right]}
\newcommand{\ACM}[2]{\left\{#1\;\middle\vert\;#2\right\}}

% operators that require brackets
\newcommand{\Pa}[2]{\underset{#1}{\operatorname{Pr}}\AB{#2}}
\newcommand{\CP}[3]{\underset{#1}{\operatorname{Pr}}\ABM{#2}{#3}}
\newcommand{\PP}[2]{\underset{\substack{#1 \\ #2}}{\operatorname{Pr}}}
\newcommand{\PPP}[3]{\underset{\substack{#1 \\ #2 \\ #3}}{\operatorname{Pr}}}
\newcommand{\E}[1]{\underset{#1}{\operatorname{E}}}
\newcommand{\Ea}[2]{\underset{#1}{\operatorname{E}}\AB{#2}}
\newcommand{\CE}[3]{\underset{#1}{\operatorname{E}}\ABM{#2}{#3}}
\newcommand{\EE}[2]{\underset{\substack{#1 \\ #2}}{\operatorname{E}}}
\newcommand{\EEE}[3]{\underset{\substack{#1 \\ #2 \\ #3}}{\operatorname{E}}}
\newcommand{\Var}{\operatorname{Var}}
\newcommand{\I}[1]{\underset{#1}{\operatorname{I}}}
\newcommand{\CI}[3]{\underset{#1}{\operatorname{I}}\ABM{#2}{#3}}
\newcommand{\Ia}[2]{\underset{#1}{\operatorname{I}}\AB{#2}}
\newcommand{\II}[2]{\underset{\substack{#1 \\ #2}}{\operatorname{I}}}
\newcommand{\III}[3]{\underset{\substack{#1 \\ #2 \\ #3}}{\operatorname{I}}}

% operators that require parentheses
\newcommand{\En}{\operatorname{H}}
\newcommand{\Ena}[1]{\operatorname{H}\AP{#1}}
\newcommand{\PS}[1]{\mathcal{P}\AP{#1}}

\newcommand{\D}{\mathrm{d}}
\newcommand{\KL}[2]{\operatorname{D}_{\mathrm{KL}}\AP{#1\middle\vert\middle\vert#2}}
\newcommand{\RD}[3]{\operatorname{D}_{#1}\AP{#2\middle\vert\middle\vert#3}}
\newcommand{\Dtv}{\operatorname{d}_{\textnormal{tv}}}
\newcommand{\Dtva}[1]{\operatorname{d}_{\textnormal{tv}}\AP{#1}}

\newcommand{\Argmin}[1]{\underset{#1}{\operatorname{arg\,min}}\,}
\newcommand{\Argmax}[1]{\underset{#1}{\operatorname{arg\,max}}\,}

\newcommand{\Nats}{\mathbb{N}}
\newcommand{\Ints}{\mathbb{Z}}
\newcommand{\Rats}{\mathbb{Q}}
\newcommand{\Reals}{\mathbb{R}}
\newcommand{\Coms}{\mathbb{C}}

\newcommand{\Estr}{\boldsymbol{\lambda}}

\newcommand{\Lim}[1]{\lim_{#1 \rightarrow \infty}}
\newcommand{\LimInf}[1]{\liminf_{#1 \rightarrow \infty}}
\newcommand{\LimSup}[1]{\limsup_{#1 \rightarrow \infty}}

\newcommand{\Abs}[1]{\left\vert #1 \right\vert}
\newcommand{\Norm}[1]{\left\Vert #1 \right\Vert}
\newcommand{\Floor}[1]{\left\lfloor #1 \right\rfloor}
\newcommand{\Ceil}[1]{\left\lceil #1 \right\rceil}
\newcommand{\Chev}[1]{\left\langle #1 \right\rangle}
\newcommand{\Quote}[1]{\left\ulcorner #1 \right\urcorner}

\newcommand{\K}{\xrightarrow{\textnormal{k}}}
\newcommand{\PF}{\xrightarrow{\circ}}

% Paper specific

% TBD

\begin{document}

In this essay I will try to explain the overall structure and motivation of my AI alignment research agenda. The discussion is informal and no new theorems are proved here. The main features of my research agenda, as I explain them here, are 

* Viewing AI alignment theory as part of a general abstract theory of intelligence

* Using desiderata and axiomatic definitions as starting points, rather than specific algorithms and constructions

* Formulating alignment problems in the language of learning theory

* Evaluating solutions by their formal mathematical properties, ultimately aiming at a quantitative theory of risk assessment

* Relying on the mathematical intuition derived from learning theory to pave the way to solving philosophical questions, to the extent these questions need to be addressed at all

\section{Philosophy}

In this section I explain the key principles and assumptions that motivate my research agenda.

\#The importance of rigor

I believe that the solution to AI alignment must rely on a rigorous mathematic theory. The algorithms that comprise the solution must be justified by formal mathematic properties. All mathematical assumptions should be either proved or at least backed by considerable evidence, like the prominent conjectures of computational complexity theory. This needs to be the case because:

* We might be facing one-shot success or failure. This means we will have little empirical backing for our assumptions.

* To the extent we have or will have empirical evidence about AI, without a rigorous underlying theory it is very hard to know how scalable and transferable the conclusions are.

* The enormity of the stakes demands designing a solution which is as reliable as possible, limited only by the time constraints imposed by competing unaligned projects.

That said, I do expect the ultimate solution to have aspects that are not entirely rigorous, specifically:

* The quantitative risk analysis will probably rely on some parameters that will be very hard to determine from first principles, because of the involvement of humans and our physical universe in the equation. These parameters might be estimated through (i) study of the evolution of intelligence (ii) study of human brains, (iii) experiments with weak AI and its interaction with humans (iv) our understanding of physics. Nevertheless, we should demand the solution to be highly reliable even given cautious error margins on these parameters.

* The ultimate solution will probably involve some heuristics.\ However, it should only involve heuristics that are designed to improve AI capabilities *without invalidating any of the assumptions underlying the risk analysis.* Thus, in the worst-case scenario these heuristics will fail and the AI will not take off but will not become unaligned.

* *In addition* to the theoretical analysis, we do want to include as much empirical testing as possible, to provide an additional layer of defense. At the least, it can be a last ditch protection in the (hopefully very unlikely) scenario that some error got through the analysis.

\#Metaphilosophy and the role of models

In order to use mathematics to solve a real-world problem, a mathematical *model* of the problem must be constructed. When the real-world problem can be defined in terms of data that is observable and measurable, the validity of the mathematical model can be ascertained using the empirical method. However, AI alignment touches on problems that are *philosophical* in nature, meaning that there is still no agreed-upon empirical or other criterion for evaluating an answer. Dealing with such problems requires a *metaphilosophy*: a way of evaluating answers to philosophical questions.

Although I do not claim a fully general solution to metaphilosophy, I think that, pragmatically, a quasiscientific approach is possible. In science, we prefer theories that are (i) simple (Occam's razor) and (ii) fit the empirical data. We also test theories by gathering further empirical data. In philosophy, we can likewise prefer theories that are (i) simple and (ii) fit intuition in situations where intuition feels reliable (i.e. situations that are simple, familiar or received considerable analysis and reflection). We can also test theories by applying them to new situations and trying to see whether the answer becomes intuitive after sufficient reflection.

Moreover, I expect progress on most problems to be achieved by the means of successive approximations. This means that we start with a model that is grossly oversimplified but that already captures some key aspects of the problems. Once we have a solution within this model, we can start to attack its assumptions and arrive at a new, more sophistical model. This process should repeat until we arrive at a model that (i) has no *obvious* shortcomings and that (ii) we seem unable to improve despite our best efforts. 

Like in science, we can never be certain that a theory is true. *Any* assumption or model can be questioned. This requires striking a balance between complacency and excessive skepticism. To avoid complacency, we need to keep working to find better theories. To avoid excessive skepticism, we should entertain hypotheses honestly and acknowledge when a theory is already capable of passing non-trivial quasiscientific tests. Reaching agreement is harder work (because our tests rely on intuition which may vary from individual to individual), but we should not despair of that goal.

\#Intelligence is understandable

It is possible question whether a mathematical theory of intelligence is possible at all. After all, we don't expect to have a tractable mathematical theory of Rococo architecture, or a simple equation describing the shape of the coastline of Africa in the year 2018.

The key difference is that intelligence is a *natural* concept. Intelligence, the way I use this word in the context of AI alignment, is the ability of an agent to make choices in a way that effectively promote its goals, in an environment that is not entirely known or even not entirely knowable. Arguing over the meaning of the word would be a distraction: this is the meaning relevant to AI alignment, because the entire concern of AI alignment is about agents that effectively pursue their goals, undermining the conflicting goals of the human species. Moreover, intelligence is (empirically) a key force in determining the evolution of the physical universe.

I conjecture that natural concepts have useful mathematical theories, and this conjecture seems to me supported by evidence in natural and computer science. It would be nice to have this conjecture itself follow from a mathematical theory, but this is outside of my current scope. Moreover, we already *have* some progress towards a mathematical theory of intelligence (I will discuss it in the next section).

A related question is, whether it is possible to design an algorithm for strong AI based on simple mathematical principles, or whether any strong AI will inevitable be an enormous kludge of heuristics designed by trial and error. I think that we have some empirical evidence support for the former, given that humans evolved to survive in a certain environment but succeeded to use their intelligence to solve problems in very different environments. That said, I am less confident about this than about the previous question. In any case, having a mathematical theory of intelligence should allow us to resolve this question too, whether positively or negatively.

\#Value alignment is understandable

The core of AI alignment is reliably transferring human values to a strong AI. However, the problem of defining what we mean by "human values" is a philosophical problem. A common and natural model of "values" is expected utility maximization: this is what we find in game theory and economics, and this is supported by VNM and Savage theorems. However, as often pointed out, humans are not perfectly rational, therefore it's not clear in what sense they can be said to maximize the expectation of a specific utility function.

Nevertheless, I believe that "values" is also a natural concept. Denying the concept of "values" altogether is paramount to nihilism, and in such a framework there is no reason to do anything at all, including saving yourself and everyone else from a murderous AI. Admitting the *general* concept of "values" as something complex and human specific (despite the focus on "values" rather than "human values") seems implausible, since intuitively we can easily imagine alien minds facing a similar AI alignment problem. Moreover, the concept of "values" is part and parcel of the concept of "intelligence", so if we believe that intelligence (due to its importance in shaping the physical world) is a natural concept, then so are values.

Therefore, I conjecture that there is a simple mathematical theory of imperfect rationality, within which the concept of "human values" is well-defined modulo the (observable, measurable) concept of "humans". Some speculation on what this theory looks like appears in the following sections.

Now, that doesn't mean that "human values" are *perfectly* well-defined, anymore than, for example, the center of mass of the sun is perfectly well-defined (which would require deciding exactly which particles are considered part of the sun). However, like the center of mass of the sun is sufficiently well-defined for many practical purposes in astrophysics, the concept of "human values" should be sufficiently well-defined for designing an aligned AI. To the extent alignment remains ambiguous, the resolution of these ambiguities doesn't have a moral significance.

\section{Foundations}

In this section I briefly explain the mathematical tools with which I set out to study AI alignment, and the outline of the mathematical theory of intelligence that these tools already painted.

\#Statistical Learning Theory

Statistical learning theory studies the information-theoretic constraints on various types of learning tasks, answering questions such as, when is a learning task solvable at all, and how much data is required to solve the learning task within given accuracy (sample complexity).

TBD

\#Computational Learning Theory

TBD

\#Algorithmic Information Theory

TBD

\#Optimality conditions

TBD

\section{Research Programme Outline}

% The "work plan" for the future

TBD

\section{Summary}

% Summarize the main points and relate them to the abstract

TBD

\end{document}



