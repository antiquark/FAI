%&latex
\documentclass[a4paper]{article}

\usepackage[a4paper,margin=1in]{geometry}
\usepackage[affil-it]{authblk}
\usepackage{cite}
\usepackage[unicode]{hyperref}
\usepackage[utf8]{inputenc}
\usepackage[english]{babel}
\usepackage{csquotes}
\usepackage{amsmath,amssymb,amsthm}
\usepackage{enumerate}
\usepackage{commath}

\newcommand{\Comment}[1]{}

\newcommand{\Bool}{\{0,1\}}
\newcommand{\Words}{{\Bool^*}}

% operators that are separated from the operand by a space
\DeclareMathOperator{\Sgn}{sgn}
\DeclareMathOperator{\Supp}{supp}
\DeclareMathOperator{\Stab}{stab}
\DeclareMathOperator{\Img}{Im}
\DeclareMathOperator{\Dom}{dom}

% operators that require brackets
\newcommand{\PP}[2]{\underset{\substack{#1 \\ #2}}{\operatorname{Pr}}}
\newcommand{\PPP}[3]{\underset{\substack{#1 \\ #2 \\ #3}}{\operatorname{Pr}}}
\newcommand{\E}[1]{\underset{#1}{\operatorname{E}}}
\newcommand{\EE}[2]{\underset{\substack{#1 \\ #2}}{\operatorname{E}}}
\newcommand{\EEE}[3]{\underset{\substack{#1 \\ #2 \\ #3}}{\operatorname{E}}}
\newcommand{\Var}{\operatorname{Var}}

% operators that require parentheses
\newcommand{\Ent}{\operatorname{H}}
\newcommand{\Hom}{\operatorname{Hom}}
\newcommand{\End}{\operatorname{End}}
\newcommand{\Sym}{\operatorname{Sym}}

\newcommand{\Prj}{\operatorname{pr}}

\newcommand{\KL}[2]{\operatorname{D}_{\mathrm{KL}}(#1 \| #2)}
\newcommand{\Dtv}{\operatorname{d}_{\text{tv}}}

\newcommand{\Argmin}[1]{\underset{#1}{\operatorname{arg\,min}}\,}
\newcommand{\Argmax}[1]{\underset{#1}{\operatorname{arg\,max}}\,}

\newcommand{\Nats}{\mathbb{N}}
\newcommand{\Ints}{\mathbb{Z}}
\newcommand{\Rats}{\mathbb{Q}}
\newcommand{\Reals}{\mathbb{R}}
\newcommand{\Coms}{\mathbb{C}}

\newcommand{\Sq}[2]{\{#1\}_{#2 \in \Nats}}
\newcommand{\Sqn}[1]{\Sq{#1}{n}}

\newcommand{\Estr}{\boldsymbol{\lambda}}

\newcommand{\Lim}[1]{\lim_{#1 \rightarrow \infty}}
\newcommand{\LimInf}[1]{\liminf_{#1 \rightarrow \infty}}
\newcommand{\LimSup}[1]{\limsup_{#1 \rightarrow \infty}}

\newcommand{\Abs}[1]{\lvert #1 \rvert}
\newcommand{\Norm}[1]{\lVert #1 \rVert}
\newcommand{\Floor}[1]{\lfloor #1 \rfloor}
\newcommand{\Ceil}[1]{\lceil #1 \rceil}
\newcommand{\Chev}[1]{\langle #1 \rangle}
\newcommand{\Quote}[1]{\ulcorner #1 \urcorner}

\newcommand{\Alg}{\xrightarrow{\textnormal{alg}}}
\newcommand{\Markov}{\xrightarrow{\textnormal{k}}}
\newcommand{\PF}{\xrightarrow{\circ}}

% Paper specific

\newcommand{\Ob}{\mathcal{O}}
\newcommand{\A}{\mathcal{A}}
\newcommand{\I}{\mathcal{I}}
\newcommand{\FH}{(\A \times \Ob)^*}
\newcommand{\IH}{(\A \times \Ob)^\omega}
\newcommand{\Ado}{\bar{\Ob}}
\newcommand{\Ada}{\bar{\A}}
\newcommand{\Adi}{{\bar{\I}}}
\newcommand{\Adao}{\overline{\A \times \Ob}}
\newcommand{\Adfh}{\Adao^*}
\newcommand{\Adih}{\Adao^\omega}
\DeclareMathOperator{\HD}{hdom}
\newcommand{\Hy}{\mathcal{H}}
\newcommand{\Co}{\mathcal{C}}

\newcommand{\Ut}{\operatorname{U}}
\newcommand{\V}{\operatorname{V}}
\newcommand{\Q}{\operatorname{Q}}
\newcommand{\EU}{\operatorname{EU}}

\begin{document}

We give a treatment of advisor corruption in [DIRL](https://agentfoundations.org/item?id=1550), more elegant and general than our previous formalism.

The following definition replaces the original Definition 5.

\#Definition

Consider a meta-universe $\upsilon=(\mu,r)$ and $\beta:(0,\infty)\rightarrow(0,\infty)$. A metapolicy $\alpha$ is called *$\beta$-rational for $\upsilon$* (as opposed to before, we assume $\alpha$ is an $\I$-metapolicy rather than an $\Adi$-metapolicy; this is purely for notational convenience and it is straightforward to generalize the definition) when there exists $\{L^\alpha_t: \HD{\mu} \times \A \rightarrow [0,\infty]\}_{t \in (0,\infty)}$ s.t.

i. For any $h \in \HD{\mu}$ and $t \in (0,\infty)$, there is $a \in \A$ s.t. $L^\alpha_t(ha)=0$.

ii. $\alpha_t(h)(a)=\exp(-\beta(t)L^\alpha_t(ha)) \max_{a^* \in \A} \alpha_t(h)(a^*)$

iii. For any $\pi \in \Pi$ and $t \in (0,\infty)$

$$\lim_{t \rightarrow \infty}\min(\E{x\sim\mu_t\bowtie\pi}[\sum_{n=0}^\infty e^{-n/t} L^\alpha_t(x_{:n+1/2})]-\E{x\sim\mu_t\bowtie\pi}[\sum_{n=0}^\infty e^{-n/t}(\V^\upsilon_t(x_{:n})-\Q^\upsilon_t(x_{:n+1/2}))],0)=0$$



***

In condition ii, $\exp(-\infty)$ is understood to mean 0. Conditions i+ii can be seen as the *definition* of $L^\alpha_t$ given $\alpha_t$. A notable special case of condition iii is when for any $x \in \IH$

$$\sum_{n=0}^\infty e^{-n/t} L^\alpha_t(x_{:n+1/2}) \geq \sum_{n=0}^\infty e^{-n/t}(\V^\upsilon_t(x_{:n})-\Q^\upsilon_t(x_{:n+1/2}))$$

As a simple example, we can have a set of corrupt states $\{\Co_t \subseteq \FH\}_{t\in(0,\infty)}$ in which the behavior of the advisor becomes arbitrary, but for each $h \in \Co_t$ there is $g \in \FH \times \A$  s.t. $g \sqsubset h$ and $L^\alpha_t(g)=\infty$ (i.e., to corrupt the advisor one has to take an action that the advisor would never take). As opposed to before, this formalism can also account for *partial* corruption, e.g. if for each $h \not\in \Co_t$ and $a \in \A$, we have $L^\alpha_t(ha) \geq \V^\upsilon_t(h) - \Q^\upsilon_t(ha)$ (like in strict $\beta$-rationality) whereas for $h \in \Co_t$, we only have $L^\alpha_t(ha) \geq \V^\upsilon_t(h) - \Q^\upsilon_t(ha) - \delta$ for some constant $\delta > 0$, then to ensure $\beta$-rationality, it is sufficient that for each $h = a_0o_0a_1o_1 \ldots \in \Co_t$:

$$\sum_{n=0}^{\max\{m \mid h_{:m} \not\in\Co_t\}} e^{-n/t}(L^\alpha_t(h_{:n}a_n) - \V^\upsilon_t(h_{:n}) - \Q^\upsilon_t(h_{:n}a_n)) \geq \frac{\delta e^{-(\max\{m \mid h_{:m} \not\in\Co_t\}+1)/t}}{1-e^{-1/t}}$$

\#Theorem

Consider $\Hy = \{\upsilon^k\}_{k \in \Nats}$ a countable family of $\I$-meta-universes and $\beta: (0,\infty) \rightarrow (0,\infty)$ s.t. $\beta(t) = \omega(t^{2/3})$. Let $\{\alpha^k\}_{k \in \Nats}$ be a family of $\I$-metapolicies s.t. for every $k \in \Nats$, $\alpha^k$ is $\beta$-rational for $\upsilon^k$. Define $\bar{\Hy}:=\{\bar{\upsilon}^k[\alpha^k]\}_{k \in \Nats}$. Then, $\bar{\Hy}$ is learnable.

\#Proof of Theorem

We don't spell out the proof in detail, but only the modifications with respect to the original.

As in the proof of the original theorem, we can assume without loss of generality that $\Hy$ is finite. Define $\pi^*$ the same way as in Lemma A, but with $L_t$ redefined as

$$L_t(ha):=\E{k\sim\zeta_t(h)}[L^{\alpha^k}_t(ha)]$$

Similarly, define $\pi^!$ the same way as in the proof of Lemma A, but with $L_t$ redefined as

$$L_t(ha):=\E{k\sim\zeta^{!k}_t(h)}[L^{\alpha^k}_t(ha)]$$

As in the proof of Lemma A, we have

$$\frac{1}{N}\sum_{k < N}(\EU_{\bar{\upsilon}^k[\alpha^k]}^{*}(t) - \EU_{\bar{\upsilon}^k[\alpha^k]}^{\pi^{!k}}(t))=\sum_{n=0}^\infty e^{-n/t} \E{(k,x)\sim\rho^!_t}[\V^{\upsilon^k[\alpha^k]}_t(x_{:n})-\Q^{\upsilon^k[\alpha^k]}_t(x_{:n}\pi^{!k}(x_{:n}))]$$

Using condition iii in the Definition, we conclude that for some function $\delta:(0,\infty)\rightarrow[0,\infty)$ with $\lim_{t\rightarrow\infty}\delta(t)=0$

$$\frac{1}{N}\sum_{k < N}(\EU_{\bar{\upsilon}^k[\alpha^k]}^{*}(t) - \EU_{\bar{\upsilon}^k[\alpha^k]}^{\pi^{!k}}(t)) \leq \sum_{n=0}^\infty e^{-n/t} \E{(k,x)\sim\rho^!_t}[L^{\alpha^k}_t(\underline{x_{:n}}\pi^{!k}(x_{:n}))]+\delta(t)$$

We can now repeat the same arguments as in the proof of Lemma A to get

$$\frac{1}{N}\sum_{k < N}(\EU_{\bar{\upsilon}^k[\alpha^k]}^{*}(t) - \EU_{\bar{\upsilon}^k[\alpha^k]}^{\pi^{*}}(t)) \leq (\frac{1}{t}+1+\frac{8 \Abs{\A}^3 \ln{N}}{e(1-e^{-1})^2})\frac{t^{2/3}}{\beta(t)}+\frac{N-1}{t^{1/3}}+\delta(t)$$

The desired result follows.

\end{document}


%&latex
\documentclass[a4paper]{article}

\usepackage[a4paper,margin=1in]{geometry}
\usepackage[affil-it]{authblk}
\usepackage{cite}
\usepackage[unicode]{hyperref}
\usepackage[utf8]{inputenc}
\usepackage[english]{babel}
\usepackage{csquotes}
\usepackage{amsmath,amssymb,amsthm}
\usepackage{enumerate}
\usepackage{commath}

\newcommand{\Comment}[1]{}

\newcommand{\Bool}{\{0,1\}}
\newcommand{\Words}{{\Bool^*}}

% operators that are separated from the operand by a space
\DeclareMathOperator{\Sgn}{sgn}
\DeclareMathOperator{\Supp}{supp}
\DeclareMathOperator{\Stab}{stab}
\DeclareMathOperator{\Img}{Im}

% operators that require brackets
\DeclareMathOperator{\Prb}{Pr}
\DeclareMathOperator{\E}{E}
\newcommand{\EE}[2]{\operatorname{E}_{\substack{#1 \\ #2}}}
\newcommand{\EEE}[3]{\operatorname{E}_{\substack{#1 \\ #2 \\ #3}}}
\DeclareMathOperator{\Var}{Var}

% operators that require parentheses
\DeclareMathOperator{\Ent}{H}
\DeclareMathOperator{\Hom}{Hom}
\DeclareMathOperator{\End}{End}
\DeclareMathOperator{\Sym}{Sym}

\DeclareMathOperator{\Prj}{pr}

\newcommand{\KL}[2]{\operatorname{D}_{\mathrm{KL}}(#1 \| #2)}
\newcommand{\Dtv}{\operatorname{d}_{\text{tv}}}

\newcommand{\Argmin}[1]{\underset{#1}{\operatorname{arg\,min}}\,}
\newcommand{\Argmax}[1]{\underset{#1}{\operatorname{arg\,max}}\,}

\newcommand{\Nats}{\mathbb{N}}
\newcommand{\Ints}{\mathbb{Z}}
\newcommand{\Rats}{\mathbb{Q}}
\newcommand{\Reals}{\mathbb{R}}
\newcommand{\Coms}{\mathbb{C}}

\newcommand{\Sq}[2]{\{#1\}_{#2 \in \Nats}}
\newcommand{\Sqn}[1]{\Sq{#1}{n}}

\newcommand{\Estr}{\boldsymbol{\lambda}}

\newcommand{\Lim}[1]{\lim_{#1 \rightarrow \infty}}
\newcommand{\LimInf}[1]{\liminf_{#1 \rightarrow \infty}}
\newcommand{\LimSup}[1]{\limsup_{#1 \rightarrow \infty}}

\newcommand{\Abs}[1]{\lvert #1 \rvert}
\newcommand{\Norm}[1]{\lVert #1 \rVert}
\newcommand{\Floor}[1]{\lfloor #1 \rfloor}
\newcommand{\Ceil}[1]{\lceil #1 \rceil}
\newcommand{\Chev}[1]{\langle #1 \rangle}
\newcommand{\Quote}[1]{\ulcorner #1 \urcorner}

\newcommand{\Alg}{\xrightarrow{\textnormal{alg}}}
\newcommand{\Markov}{\xrightarrow{\textnormal{mk}}}

\newcommand{\Prob}{\mathcal{P}}

% Paper specific

\newcommand{\Ob}{\mathcal{O}}
\newcommand{\T}{\mathcal{T}}
\newcommand{\B}{\mathcal{B}}
\newcommand{\UM}{\mathcal{U}}
\newcommand{\W}{\operatorname{W}}
\newcommand{\SW}{\operatorname{\Sigma W}}
\newcommand{\I}{\operatorname{id}}
\newcommand{\Lip}{\operatorname{Lip}}
\newcommand{\NormL}[1]{\Norm{#1}_{\operatorname{Lip}}}
\newcommand{\Dkr}{\operatorname{d}_{\text{KR}}}
\newcommand{\Ball}{\operatorname{B}}
\newcommand{\F}{\mathcal{F}}

\begin{document}

*Blah blah blah*

Un deux trois.

***

Appendix A contains the proofs...

\section{Notation}

Whatever...

\section{Results}

Foo

\#Definition 1

Bar

\#Lemma 1

Foo

***

Bar

\section{Appendix A}

\#Proposition A.1

Hahaha

\#Proof of Proposition A.1

Mwhahaha

\end{document}



