%&latex
\documentclass[a4paper]{article}

\usepackage[a4paper,margin=1in]{geometry}
\usepackage[affil-it]{authblk}
\usepackage{cite}
\usepackage[unicode]{hyperref}
\usepackage[utf8]{inputenc}
\usepackage[english]{babel}
\usepackage{csquotes}
\usepackage{amsmath,amssymb,amsthm}
\usepackage{enumerate}
\usepackage{commath}

\newcommand{\Comment}[1]{}

\newcommand{\Bool}{\{0,1\}}
\newcommand{\Words}{{\Bool^*}}

% operators that are separated from the operand by a space
\DeclareMathOperator{\Sgn}{sgn}
\DeclareMathOperator{\Supp}{supp}
\DeclareMathOperator{\Stab}{stab}
\DeclareMathOperator{\Img}{Im}
\DeclareMathOperator{\Dom}{dom}

% autosize deliminaters
\newcommand{\AP}[1]{\left(#1\right)}
\newcommand{\AB}[1]{\left[#1\right]}
\newcommand{\AC}[1]{\left\{#1\right\}}

% operators that require brackets
\newcommand{\Pa}[2]{\underset{#1}{\operatorname{Pr}}\AB{#2}}
\newcommand{\PP}[2]{\underset{\substack{#1 \\ #2}}{\operatorname{Pr}}}
\newcommand{\PPP}[3]{\underset{\substack{#1 \\ #2 \\ #3}}{\operatorname{Pr}}}
\newcommand{\E}[1]{\underset{#1}{\operatorname{E}}}
\newcommand{\Ea}[2]{\underset{#1}{\operatorname{E}}\AB{#2}}
\newcommand{\EE}[2]{\underset{\substack{#1 \\ #2}}{\operatorname{E}}}
\newcommand{\EEE}[3]{\underset{\substack{#1 \\ #2 \\ #3}}{\operatorname{E}}}
\newcommand{\I}[1]{\underset{#1}{\operatorname{I}}}
\newcommand{\Ia}[2]{\underset{#1}{\operatorname{I}}\AB{#2}}
\newcommand{\II}[2]{\underset{\substack{#1 \\ #2}}{\operatorname{I}}}
\newcommand{\III}[3]{\underset{\substack{#1 \\ #2 \\ #3}}{\operatorname{I}}}
\newcommand{\Var}{\operatorname{Var}}

% operators that require parentheses
\newcommand{\En}{\operatorname{H}}
\newcommand{\Ena}[1]{\operatorname{H}\AP{#1}}
\newcommand{\Hom}{\operatorname{Hom}}
\newcommand{\End}{\operatorname{End}}
\newcommand{\Sym}{\operatorname{Sym}}

\newcommand{\Prj}{\operatorname{pr}}

\newcommand{\D}{\mathrm{d}}
\newcommand{\KL}[2]{\operatorname{D}_{\mathrm{KL}}(#1 \| #2)}
\newcommand{\Dtv}{\operatorname{d}_{\text{tv}}}
\newcommand{\Dtva}[1]{\operatorname{d}_{\text{tv}}\AP{#1}}

\newcommand{\Argmin}[1]{\underset{#1}{\operatorname{arg\,min}}\,}
\newcommand{\Argmax}[1]{\underset{#1}{\operatorname{arg\,max}}\,}

\newcommand{\Nats}{\mathbb{N}}
\newcommand{\Ints}{\mathbb{Z}}
\newcommand{\Rats}{\mathbb{Q}}
\newcommand{\Reals}{\mathbb{R}}
\newcommand{\Coms}{\mathbb{C}}

\newcommand{\Sq}[2]{\{#1\}_{#2 \in \Nats}}
\newcommand{\Sqn}[1]{\Sq{#1}{n}}

\newcommand{\Estr}{\boldsymbol{\lambda}}

\newcommand{\Lim}[1]{\lim_{#1 \rightarrow \infty}}
\newcommand{\LimInf}[1]{\liminf_{#1 \rightarrow \infty}}
\newcommand{\LimSup}[1]{\limsup_{#1 \rightarrow \infty}}

\newcommand{\Abs}[1]{\left\vert #1 \right\vert}
\newcommand{\Norm}[1]{\left\Vert #1 \right\Vert}
\newcommand{\Floor}[1]{\left\lfloor #1 \right\rfloor}
\newcommand{\Ceil}[1]{\left\lceil #1 \right\rceil}
\newcommand{\Chev}[1]{\left\langle #1 \right\rangle}
\newcommand{\Quote}[1]{\left\ulcorner #1 \right\urcorner}

\newcommand{\Alg}{\xrightarrow{\text{alg}}}
\newcommand{\M}{\xrightarrow{\text{k}}}
\newcommand{\PF}{\xrightarrow{\circ}}

% Paper specific

\newcommand{\Ob}{\mathcal{O}}
\newcommand{\A}{\mathcal{A}}
\newcommand{\St}{\mathcal{S}}
\newcommand{\T}{\mathcal{T}}
\newcommand{\R}{\mathcal{R}}
\newcommand{\In}{\mathcal{I}}
\newcommand{\FH}{(\A \times \Ob)^*}
\newcommand{\IH}{(\A \times \Ob)^\omega}
\newcommand{\Ado}{\bar{\Ob}}
\newcommand{\Ada}{\bar{\A}}
\newcommand{\Adi}{{\bar{\In}}}
\newcommand{\Adao}{\overline{\A \times \Ob}}
\newcommand{\Adfh}{\Adao^*}
\newcommand{\Adih}{\Adao^\omega}
\DeclareMathOperator{\HD}{hdom}
\newcommand{\Hy}{\mathcal{H}}
\newcommand{\UC}{\mathcal{U}}

\newcommand{\RMC}{\mathrm{C}}
\newcommand{\RMD}{\mathrm{D}}
\newcommand{\RME}{\mathrm{E}}
\newcommand{\RMF}{\mathrm{F}}

\newcommand{\SF}{\St^{\RMF}}
\newcommand{\SD}{\St^{\RMD}}
\newcommand{\SC}{\St^{\RMC}}
\newcommand{\MF}{M^{\RMF}}
\newcommand{\MD}{M^{\RMD}}
\newcommand{\ME}{M^{\RME}}
\newcommand{\TF}{\bar{\tau}^{\RMF}}
\newcommand{\PD}{\pi^{\RMD}}
\newcommand{\UD}{\upsilon^{\RMD}}

\newcommand{\Ut}{\operatorname{U}}
\newcommand{\V}{\operatorname{V}}
\newcommand{\Q}{\operatorname{Q}}
\newcommand{\EU}{\operatorname{EU}}

\newcommand{\Dl}{\mathcal{D}}
\newcommand{\Do}{\mathfrak{D}}
\newcommand{\F}{\mathcal{F}}
\newcommand{\B}{\mathcal{B}}
\newcommand{\Z}{Z}
\newcommand{\J}{J}

\begin{document}

Previously, I presented the theory of [DRL](https://agentfoundations.org/item?id=1656) for finite MDPs. Now, although I believe this theory can be generalized much beyond finite MDPs (at least to finite POMDPs and infinite POMDPs that satisfy some geometric and/or dynamical systems theoretic assumptions), it cannot work for arbitrary environments (without requiring the advisor to be more than "merely sane"). Constructing a counterexample is not difficult, but it seemed worthwhile to write it down.

Let $\A := \{a,b,c\}$ be the set of actions and $\Ob := \{o_-,o_+\}$ the set of observations. Define $\tau: \FH \rightarrow \Nats \sqcup \{\infty\}$ s.t. for any $n \in \Nats$ and $h \in (\A \times \Ob)^n$

$$\tau(h):=\begin{cases} \infty \text{ if } \forall m \in [n]:h_m \not\in c \times \Ob \\ \min\{m \in [n] \mid h_m \in c \times \Ob\} \text{ otherwise} \end{cases}$$

That is, $\tau(h)$ is the first time action $c$ (stands for "commit") appears in the history $h$. Now, consider the following environments $\mu_a$, $\mu_b$. 

$$\mu_a\AP{o_+ \mid h}:=\begin{cases} 0 \text{ if } \tau(h)=\infty \\ 0 \text{ if } \tau(h) \leq \frac{n}{2} \\ \frac{1}{\tau(h)}\Abs{\AC{m \in \AB{\tau(h)} \mid h_m \in a \times \Ob}} \text{ otherwise} \end{cases}$$

$$\mu_b\AP{o_+ \mid h}:=\begin{cases} 0 \text{ if } \tau(h)=\infty \\ 0 \text{ if } \tau(h) \leq \frac{n}{2} \\ \frac{1}{\tau(h)}\Abs{\AC{m \in \AB{\tau(h)} \mid h_m \in b \times \Ob}} \text{ otherwise} \end{cases}$$

Also, define the reward function $r$ by setting $r(h)=1$ for any history $h$ that ends with $o_+$ and $r(h) = 0$ for any other history. Denote $\upsilon_{a,b} := \AP{\mu_{a,b},r}$. That is, both universes count the number of actions $a$ and $b$ until time $\tau$: the first time action $c$ is taken. At times between $\tau$ and $2\tau$, they produce rewards with frequency that equals the relative fraction of the corresponding action in the count. Before $\tau$ and after $2\tau$, they produce no rewards.

Now, we haven't defined sane policies for arbitrary environments, but the spirit of the definition is that a sane policy is unlikely to take actions with irreversible long-term negative consequences and has to have some non-negligible probability of taking an optimal (or at least nearly optimal) action. For example, we might define it as follows

\#Definition

Consider $\beta\in(0,\infty)$, $\gamma,\epsilon\in(0,1)$ and a universe $\upsilon=(\mu,r)$. A policy $\sigma$ is called *$(\epsilon,\beta)$-sane* for $(\upsilon,\gamma)$ when for any $h\in\HD{\mu}$

i. $$\exists a^* \in \Argmax{a \in \A}{\Q^\upsilon_\gamma(h,a)} : \sigma\AP{a^* \mid h} \geq \epsilon$$

ii. $$\forall a \in \A: \sigma\AP{a \mid h} \leq \exp \AB{\beta\AP{\Q^\upsilon_\gamma(h,a) - \V^\upsilon_\gamma(h)}}$$

***

Here, we consider the asymptotics $\gamma \rightarrow 1$, $\beta \rightarrow \infty$ but should include the scenario $\beta = o\AP{\frac{1}{1-\gamma}}$ so that actions with *short-term* negative consequences are allowed (otherwise this would be some sort of "rationality" requirement rather than a mere "sanity" requirement).

Now, the optimal policy for $\upsilon_{a}$ (respectively $\upsilon_b$) is taking action $a$ (resp. $b$) until some time $\tau^*(\gamma) = \Theta\AP{\frac{1}{1-\gamma}}$ and taking action $c$ at this time (obviously, the following actions don't affect anything). Let $\pi^*_{a,b}$ be policies that are optimal *even off-policy* (so that their decision when to take action $c$ depends on the history of actions before). It is easy to see that the following policies $\sigma_{a,b}$ are $\AP{\frac{1}{2},\beta}$-sane for the respective universes and some "legitimate" choice of $\beta$:

$$\sigma_a(a \mid h) := \sigma_a(b \mid h) := \frac{1}{2}\pi^*_a(a \mid h)$$

$$\sigma_b(a \mid h) := \sigma_b(b \mid h) := \frac{1}{2}\pi^*_b(b \mid h)$$

That is, whenever $\pi^*$ suggests taking action $c$, $\sigma$ complies with it, and whenever $\pi^*$ suggests taking action $a$ or $b$, $\sigma$ flips a coin to decide which one to take. Indeed, in expectation, taking the wrong action out of $\{a,b\}$ loses one time moment of reward and is therefore equivalent to a "short-term" loss.

\end{document}



