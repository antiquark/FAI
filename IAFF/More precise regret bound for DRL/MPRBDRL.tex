%&latex
\documentclass[a4paper]{article}

\usepackage[a4paper,margin=1in]{geometry}
\usepackage[affil-it]{authblk}
\usepackage{cite}
\usepackage[unicode]{hyperref}
\usepackage[utf8]{inputenc}
\usepackage[english]{babel}
\usepackage{csquotes}
\usepackage{amsmath,amssymb,amsthm}
\usepackage{enumerate}
\usepackage{commath}

\newcommand{\Comment}[1]{}

\newcommand{\Bool}{\{0,1\}}
\newcommand{\Words}{{\Bool^*}}

% operators that are separated from the operand by a space
\DeclareMathOperator{\Sgn}{sgn}
\DeclareMathOperator{\Supp}{supp}
\DeclareMathOperator{\Stab}{stab}
\DeclareMathOperator{\Img}{Im}
\DeclareMathOperator{\Dom}{dom}

% autosize deliminaters
\newcommand{\AP}[1]{\left(#1\right)}
\newcommand{\AB}[1]{\left[#1\right]}
\newcommand{\AC}[1]{\left\{#1\right\}}

% operators that require brackets
\newcommand{\Pa}[2]{\underset{#1}{\operatorname{Pr}}\AB{#2}}
\newcommand{\PP}[2]{\underset{\substack{#1 \\ #2}}{\operatorname{Pr}}}
\newcommand{\PPP}[3]{\underset{\substack{#1 \\ #2 \\ #3}}{\operatorname{Pr}}}
\newcommand{\E}[1]{\underset{#1}{\operatorname{E}}}
\newcommand{\Ea}[2]{\underset{#1}{\operatorname{E}}\AB{#2}}
\newcommand{\EE}[2]{\underset{\substack{#1 \\ #2}}{\operatorname{E}}}
\newcommand{\EEE}[3]{\underset{\substack{#1 \\ #2 \\ #3}}{\operatorname{E}}}
\newcommand{\I}[1]{\underset{#1}{\operatorname{I}}}
\newcommand{\Ia}[2]{\underset{#1}{\operatorname{I}}\AB{#2}}
\newcommand{\II}[2]{\underset{\substack{#1 \\ #2}}{\operatorname{I}}}
\newcommand{\III}[3]{\underset{\substack{#1 \\ #2 \\ #3}}{\operatorname{I}}}
\newcommand{\Var}{\operatorname{Var}}

% operators that require parentheses
\newcommand{\En}{\operatorname{H}}
\newcommand{\Ena}[1]{\operatorname{H}\AP{#1}}
\newcommand{\Hom}{\operatorname{Hom}}
\newcommand{\End}{\operatorname{End}}
\newcommand{\Sym}{\operatorname{Sym}}

\newcommand{\Prj}{\operatorname{pr}}

\newcommand{\D}{\mathrm{d}}
\newcommand{\KL}[2]{\operatorname{D}_{\mathrm{KL}}(#1 \| #2)}
\newcommand{\Dtv}{\operatorname{d}_{\text{tv}}}
\newcommand{\Dtva}[1]{\operatorname{d}_{\text{tv}}\AP{#1}}

\newcommand{\Argmin}[1]{\underset{#1}{\operatorname{arg\,min}}\,}
\newcommand{\Argmax}[1]{\underset{#1}{\operatorname{arg\,max}}\,}

\newcommand{\Nats}{\mathbb{N}}
\newcommand{\Ints}{\mathbb{Z}}
\newcommand{\Rats}{\mathbb{Q}}
\newcommand{\Reals}{\mathbb{R}}
\newcommand{\Coms}{\mathbb{C}}

\newcommand{\Sq}[2]{\{#1\}_{#2 \in \Nats}}
\newcommand{\Sqn}[1]{\Sq{#1}{n}}

\newcommand{\Estr}{\boldsymbol{\lambda}}

\newcommand{\Lim}[1]{\lim_{#1 \rightarrow \infty}}
\newcommand{\LimInf}[1]{\liminf_{#1 \rightarrow \infty}}
\newcommand{\LimSup}[1]{\limsup_{#1 \rightarrow \infty}}

\newcommand{\Abs}[1]{\left\vert #1 \right\vert}
\newcommand{\Norm}[1]{\left\Vert #1 \right\Vert}
\newcommand{\Floor}[1]{\left\lfloor #1 \right\rfloor}
\newcommand{\Ceil}[1]{\left\lceil #1 \right\rceil}
\newcommand{\Chev}[1]{\left\langle #1 \right\rangle}
\newcommand{\Quote}[1]{\left\ulcorner #1 \right\urcorner}

\newcommand{\Alg}{\xrightarrow{\text{alg}}}
\newcommand{\M}{\xrightarrow{\text{k}}}
\newcommand{\PF}{\xrightarrow{\circ}}

% Paper specific

\newcommand{\Ob}{\mathcal{O}}
\newcommand{\A}{\mathcal{A}}
\newcommand{\St}{\mathcal{S}}
\newcommand{\T}{\mathcal{T}}
\newcommand{\R}{\mathcal{R}}
\newcommand{\In}{\mathcal{I}}
\newcommand{\FH}{(\A \times \Ob)^*}
\newcommand{\IH}{(\A \times \Ob)^\omega}
\newcommand{\Ado}{\bar{\Ob}}
\newcommand{\Ada}{\bar{\A}}
\newcommand{\Adi}{{\bar{\In}}}
\newcommand{\Adao}{\overline{\A \times \Ob}}
\newcommand{\Adfh}{\Adao^*}
\newcommand{\Adih}{\Adao^\omega}
\DeclareMathOperator{\HD}{hdom}
\newcommand{\Hy}{\mathcal{H}}
\newcommand{\UC}{\mathcal{U}}

\newcommand{\RMC}{\mathrm{C}}
\newcommand{\RMD}{\mathrm{D}}
\newcommand{\RME}{\mathrm{E}}
\newcommand{\RMF}{\mathrm{F}}

\newcommand{\SF}{\St^{\RMF}}
\newcommand{\SD}{\St^{\RMD}}
\newcommand{\SC}{\St^{\RMC}}
\newcommand{\MF}{M^{\RMF}}
\newcommand{\MD}{M^{\RMD}}
\newcommand{\ME}{M^{\RME}}
\newcommand{\TF}{\bar{\tau}^{\RMF}}
\newcommand{\PD}{\pi^{\RMD}}
\newcommand{\UD}{\upsilon^{\RMD}}

\newcommand{\Ut}{\operatorname{U}}
\newcommand{\V}{\operatorname{V}}
\newcommand{\Q}{\operatorname{Q}}
\newcommand{\EU}{\operatorname{EU}}

\newcommand{\Dl}{\mathcal{D}}
\newcommand{\Do}{\mathfrak{D}}
\newcommand{\F}{\mathcal{F}}
\newcommand{\B}{\mathcal{B}}
\newcommand{\Z}{Z}
\newcommand{\J}{J}

\newcommand{\Pd}{P}

\begin{document}

We derive a regret bound for [DRL](https://agentfoundations.org/item?id=1656) reflecting dependence on:

* Number of hypotheses

* Mixing time of MDP hypotheses

* The probability with which the advisor takes optimal actions

That is, the regret bound we get is fully explicit up to a multiplicative constant (which can also be made explicit). Currently we focus on plain (as opposed to [catastrophe](https://agentfoundations.org/item?id=1715)) and uniform (finite number of hypotheses, uniform prior) DRL, although this result can and should be extended to the catastrophe and/or non-uniform settings.

***

Appendix A contains the proofs, Appendix B recalls a few lemmas we need from previous essays.

\section{Notation}

Whatever...

\section{Results}

First, we briefly recall some properties of Markov chains.

\#Definition 1

Consider $\St$ a finite set and $\T: \St \M \St$. We say that $k \in \Nats^+$ is a *period of $\T$* when there is $s \in \St$ an essential state of $\T$ (that is, $\T^\infty(s \mid s) > 0$) s.t. $k$ is its period, i.e. $k = \gcd \AC{n \in \Nats^+ \mid \T^n(s \mid s) > 0}$. We denote $\Pd_\T$ the set of periods of $\T$.

***

The following is a corollary of the Perron-Frobenius theorem which we give without proof. *[I believe this is completely standard and would be grateful to get a source for this which treats the reducible case; of course I can produce the proof but it seems redundant.]*

\#Proposition 1

Consider $\St$ a finite set and $\T: \St \M \St$. Then, there are $F \in (0,\infty)$, $\lambda\in(0,1)$, $\zeta: \St \M \Pd_\T$ and $\xi: \St \times \Pd_\T \M \St$ s.t. for any $s \in \St_M$

$$\forall k \in \Pd_\T: \T^k \xi(s,k) = \xi(s,k)$$

$$\forall n \in \Nats: \Dtva{\T^n(s),\Ea{k \sim \zeta(s)}{\T^n \xi(s,k)}} \leq F \lambda^n$$

***

For the purpose of this essay, we will use a definition of local sanity slightly stronger than what [previously](https://agentfoundations.org/item?id=1656) appeared as "Definition 4." We think this strengthening is not substantial, but also the current analysis can be generalized to the weaker case by adding a term proportional to the 2nd derivative of $\V$ (or the 2nd moment of the mixing time). We leave out the details for the time being.

We will use the notation $\A_M^\omega(s):=\bigcap_{k\in\Nats} \A_M^k(s)$.

\#Definition 2

Let $\upsilon = (\mu,r)$ be a universe and $\epsilon \in (0,1)$. A policy $\sigma$ is said to be *locally $\epsilon$-sane for $\upsilon$* when there are $M$, $S$ and $\UC_M \subseteq \St_M$ (the set of uncorrupt states) s.t. $\upsilon$ is an $\Ob$-realization of $M$ with state function $S$, $S(\Estr) \in \UC_M$ and for any $h \in \HD{\mu}$, if $S(h) \in \UC_M$ then

i. $$\forall a \in \Supp{\sigma(h)}: \T_M\AP{\UC_M \mid S(h),a} = 1$$

ii. $$\Supp{\sigma(h)} \subseteq \A_M^0\left(S(h)\right)$$

iii. $$\exists a \in \A_M^\omega\AP{S(h)}: \sigma(a \mid h) > \epsilon$$

***

Recall that for any MPD $M$, there is $\gamma_M\in(0,1)$ s.t. for any $\gamma\in[\gamma_M,1)$, $a \in \A_M^\omega(s)$ if and only if $\Q_M(s,a,\gamma)=\V_M(s,\gamma)$.

We are now ready to state the regret bound.

\#Theorem 1

There is some $C \in (0,\infty)$ s.t. the following holds.

Consider $\In$ an interface, $\alpha,\epsilon \in (0,1)$, $\Hy = \{\upsilon^k = (\mu^k,r^k) \in \Upsilon_{\In}\}_{k \in [N]}$ for some $N \in \Nats$. For each $k \in [N]$, let $\sigma^k$ be a locally $\epsilon$-sane policy for $\upsilon^k$. For each $k \in [N]$, let $M^k$ be the corresponding MDP and $\UC^k \subseteq \St_{M^k}$ the corresponding set of uncorrupt states. Assume that 

i. For each $k \in [N]$, $\alpha \leq 1 - \gamma_{M^k}$.

ii. For any $k,j \in [N]$ and $h \in \HD{\mu^k} \cap \HD{\mu^j}$, if $S^k(h) \in \UC^k$ and $S^j(h) \in \UC^j$, then $r^k(h)=r^j(h)$. 

Then, there is an $\bar{\In}$-policy $\pi^*$ s.t. for any $k \in [N]$

$$\EU_{\upsilon^k}^*(1-\alpha) - \EU_{\bar{\upsilon}^k\left[\sigma^k\right]}^{\pi^*}(1-\alpha) \leq C\AP{\alpha N^5 \ln{N} \AP{\epsilon^{-1}+\Abs{\A}} \sum_{j = 0}^{N-1} \max_{s \in \UC^j} \sup_{\gamma \in [1-\alpha,1)} \Abs{\frac{\D{\V_{M^j}(s,\gamma)}}{\D\gamma}}}^{1/4}$$

***

The factor $\max_{s \in \UC^k} \sup_{\gamma \in [1-\alpha,1]} \Abs{\frac{\D{\V_{M^j}(s,\gamma)}}{\D\gamma}}$ might seem difficult to understand. However, it can be bounded as follows.

\#Proposition 2

Let $M$ be an MDP, $\pi$ a Blackwell optimal policy for $M$ and $F \in (0,\infty)$, $\lambda \in (0,1)$, $\Pd \subseteq \Nats^+$ as in Proposition 1 applied to the Markov chain $\T_{M\pi}$. Then

$$\max_{s \in \St_M} \sup_{\gamma \in [\gamma_M,1)} \Abs{\frac{\D{\V_M(s,\gamma)}}{\D\gamma}} \leq F\frac{1+\lambda}{1-\lambda}+\max{\Pd}$$

***

Theorem 1 and Proposition 2 immediately give the following:

\#Corollary 1

There is some $C' \in (0,\infty)$ s.t. the following holds.

Under the conditions of Theorem 1, let $\pi^k: \St_{M^k} \rightarrow \A$ be s.t. for any $s \in \UC^k$

i. $$\pi^k(s) \in \A_{M^k}^\omega(s)$$

ii. $$\T_{M^k \pi^k}\AP{\UC^k \mid s} = 1$$ 

Assuming w.l.o.g. that all uncorrupt states are reachable from $S^k(\Estr)$, $\pi^k$ is guaranteed to exist thanks to condition iii of Definition 2 (if some uncorrupt state is unreachable, we can consider it to be corrupt.) Let $F_k\in(0,\infty)$, $\lambda_k\in(0,1)$ and $\Pd_k \subseteq \Nats^+$ be as in Proposition 1, for the Markov chain $\T_{M^k\pi^k}: \UC^k \M \UC^k$. Then, there is an $\bar{\In}$-policy $\pi^*$ s.t. for any $k \in [N]$ 

$$\EU_{\upsilon^k}^*(1-\alpha) - \EU_{\bar{\upsilon}^k\left[\sigma^k\right]}^{\pi^*}(1-\alpha) \leq C'\AP{\alpha N^5 \ln{N} \AP{\epsilon^{-1}+\Abs{\A}} \sum_{j = 0}^{N-1} \AP{\frac{F_j}{1-\lambda_j}+\max{\Pd_j}}}^{1/4}$$

\section{Appendix A}

The proof of Theorem 1 mostly repeats the proof of the [previous](https://agentfoundations.org/item?id=1656) "Theorem 1", except that we keep track of the bounds more carefully.

\#Proof of Theorem 1

Fix $\eta\in\left(0,N^{-1}\right)$ and $T \in \Nats^+$. Denote $\nu^k:=\bar{\mu}^k\left[\sigma^k S^k\right]$. To avoid cumbersome notation, whenever $M^k$ should appear a subscript, we will replace it by $k$. Let $(\Omega,P \in \Delta\Omega)$ be a probability space\Comment{ and $\{\F_n \subseteq 2^\Omega\}_{n \in \Nats \sqcup \{-1\}}$ a filtration of $\Omega$}. Let $K: \Omega \rightarrow [N]$ be \Comment{measurable w.r.t. $\F_{-1}$}a random variable and the following be stochastic processes\Comment{ adapted to $\F$}

$$\Z_n,\tilde{\Z}_n: \Omega \rightarrow \Delta[N]$$

$$\J_n: \Omega \rightarrow [N]$$

$$\Psi_n: \Omega \rightarrow \A$$

$$A_n: \Omega \rightarrow \Ada$$

$$\Theta_n: \Omega \rightarrow \Ado$$

We also define $A\Theta_{:n}: \Omega \rightarrow \Adfh$ by

$$A\Theta_{:n}:= A_0\Theta_0A_1\Theta_1 \ldots A_{n-1}\Theta_{n-1}$$

(The following conditions on $A$ and $\Theta$ imply that the range of the above is indeed in $\Adfh$.) Let $\Dl$ and $\Dl^{!k}$ be as in Proposition B.N1 (the form of the bound we are proving allows assuming w.l.o.g. that $\epsilon < \frac{1}{\Abs{\A}}$). By condition iii of Definition 2, there is $\pi^k : \HD{\mu^k} \rightarrow \A$ s.t. for any $h \in \HD{\mu^k}$, if $S^k(h) \in \UC^k$ then

$$\sigma^k\AP{\pi^k(h) \mid h} > \epsilon$$

$$\pi^k(h) \in \A_k^\omega\AP{S^k(h)}$$

We construct $\Omega$\Comment{, $\F$}, $K$, $\Z$, $\tilde{\Z}$, $\J$, $\Psi$, $A$ and $\Theta$ s.t $K$ is uniformly distributed and for any $k \in [N]$, $l \in \Nats$, $m \in [T]$ and $o \in \Ob$, denoting $n = lT+m$

$$\tilde{\Z}_0(k)\equiv\frac{1}{N}$$

$$\Z_{n}(k) = \frac{\tilde{\Z}_{n}(k)[[\tilde{\Z}_{n}(k) \geq \eta]] }{\sum_{j = 0}^{N-1}\tilde{\Z}_{n}(j)[[\tilde{\Z}_{n}(j) \geq \eta]]}$$

$$\Pr\left[\J_{l} = k \mid Z_{lT}\right] = \Z_{lT}\left(k\right)$$

$$\Psi_{n} = \pi^{J_l}\AP{A\Theta_{:n}}$$

$$\Pr\left[\Theta_{n} = o \mid A\Theta_{:n}\right] = \nu^K\left(o \mid A\Theta_{:n}\right)$$

$$A_n = \Dl\left(A\Theta_{:n}, \Psi_n\right)$$

$$\tilde{\Z}_{n+1}(k)\sum_{j = 0}^{N-1} \Z_n(j) [[A_n = \Dl^{!j}\left(A\Theta_{:n}, \Psi_n\right)]] \nu^j(\Theta_n \mid A\Theta_{:n}A_n)=\Z_{n}(k) [[A_n = \Dl^{!k}\left(A\Theta_{:n}, \Psi_n\right)]] \nu^k\left(\Theta_{n} \mid A\Theta_{:n}A_{n}\right)$$

Note that the last equation has the form of a Bayesian update which is allowed to be arbitrary when update is on "impossible" information.

We now construct the $\Adi$-policy $\pi^*$ s.t. for any $n \in \Nats$, $h \in \Adfh$ s.t. $\Pr\left[A\Theta_{:n}=h\right] > 0$ and $a \in \Ada$

$$\pi^*(a \mid h):=\Pr\left[A_n = a \mid A\Theta_{:n} = h\right]$$

That is, we perform Thompson sampling at time intervals of size $T$, moderated by the delegation routine $\Dl$, and discard from our belief state hypotheses whose probability is below $\eta$ and hypotheses sampling which resulted in recommending "unsafe" actions i.e. actions that $\Dl$ refused to perform.

In order to prove $\pi^*$ has the desired property, we will define the stochastic processes $\Z^!$, $\tilde{\Z}^!$, $\J^!$, $\Psi^!$, $A^!$ and $\Theta^!$, each process of the same type as its shriekless counterpart (thus $\Omega$ is constructed to accommodate them). These processes are required to satisfy the following:

$$\tilde{\Z}^!_0(k)\equiv\frac{1}{N}$$

$$\Z_{n}^!(k) = \frac{\tilde{\Z}^!_{n}(k)[[\tilde{\Z}^!_{n}(k) \geq \eta]] }{\sum_{j = 0}^{N-1}\tilde{\Z}^!_{n}(j)[[\tilde{\Z}^!_{n}(j) \geq \eta]]}[[\tilde{\Z}^!_{n}(K) \geq \eta]] + [[K = k]]\cdot [[\tilde{\Z}^!_{n}(K) < \eta]]$$

$$\Pr\left[\J^!_{l} = k \mid Z^!_{lT}\right] = \Z^!_{lT}\left(k\right)$$

$$\Psi^!_{n} = \pi^{\J^!_l}\AP{A\Theta^!_{:n}}$$

$$\Pr\left[\Theta^!_{n} = o \mid A\Theta^!_{:n}\right] = \nu^K\left(o \mid A\Theta^!_{:n}\right)$$

$$A^!_n = \Dl^{!K}\left(A\Theta^!_{:n}, \Psi^!_n\right)$$

$$\tilde{\Z}^!_{n+1}(k)=\frac{\Z^!_{n}(k) [[A^!_n = \Dl^{!k}\left(A\Theta^!_{:n}, \Psi^!_n\right)]] \nu^k\left(\Theta^!_{n} \mid A\Theta^!_{:n}A^!_{n}\right)}{\sum_{j = 0}^{N-1} \Z^!_n(j) [[A^!_n = \Dl^{!j}\left(A\Theta^!_{:n}, \Psi^!_n\right)]] \nu^j(\Theta^!_n \mid A\Theta^!_{:n}A^!_n)}$$

For any $k \in [N]$, we construct the $\Adi$-policy $\pi^{?k}$ s.t. for any $n \in \Nats$, $h \in \Adfh$ s.t. $\Pr\left[A\Theta^!_{:n}=h,\ K = k\right] > 0$ and $a \in \Ada$

$$\pi^{?k}(a \mid h):=\Pr\left[A^!_n = a \mid A\Theta^!_{:n} = h,\ K = k\right]$$

Given any $\Adi$-policy $\pi$ and $\In$-policy $\sigma$ we define $\alpha_{\sigma\pi}: \FH \M \Adfh$ by

$$\alpha_{\sigma\pi} (g \mid h) := [[h = \underline{g}]]C_h\prod_{n = 0}^{\Abs{h}-1} \sum_{a \in \A}\left([[g_n \in \bot a\Ob]] \pi\left(\bot \mid g_{:n}\right)\sigma\left(a \mid h_{:n}\right)+[[g_n \in a\bot\Ob]]\pi\left(a \mid g_{:n}\right)\right)$$

Here, $C_h \in \Reals$ is a constant defined s.t. the probabilities sum to 1. We define the $\In$-policy $\left[\sigma\right]\underline{\pi}$ by

$$\left[\sigma\right]\underline{\pi}(a \mid h):=\Pr_{g \sim \alpha_{\sigma\pi}(h)}\left[\pi\left(g\right)=a \lor \left(\pi\left(g\right)=\bot \land \sigma(h)=a\right)\right]$$

Define $\tau^k \in (0,1)$ by

$$\tau^k := \max_{s \in \UC^k} \sup_{\gamma \in [1 - \alpha, 1)} \Abs{\frac{\D\V_k(s,\gamma)}{\D\gamma}}$$

Condition iii of Proposition B.N1 and condition ii of Definition 2 imply that for any $h \in \HD{\mu^k}$

$$\Supp{\AB{\sigma^k}\underline{\pi}^{?k}(h)} \subseteq \A^0_k\AP{S^k(h)}$$

This means we can apply Proposition B.N2 and get

$$\EU^*_{\upsilon^k}(1 - \alpha)-\EU^{\pi^{?k}}_{\bar{\upsilon}^k\AB{\sigma^k}}(1 - \alpha) \leq \alpha\sum_{n=0}^\infty \sum_{m=0}^{T-1} (1-\alpha)^{nT+m}\left(\E{x\sim\mu^k\bowtie\pi^{*k}_n}\left[r\left(x_{:nT+m}\right)\right]-\E{x\sim\nu^k\bowtie\pi^{?k}}\left[r\left(\underline{x}_{:nT+m}\right)\right]\right) + \frac{2\tau^k\alpha}{1-(1-\alpha)^T}$$

Here, the $\In$-policy $\pi^{*k}_n$ is defined as $\pi^*_n$ in Proposition B.N2, with $\pi^k$ in place of $\pi^*$. We also define the $\Adi$-policies $\pi^{!k}_n$ and $\pi^{!!k}_n$ by

$$\pi^{!k}_n(a \mid h):=\begin{cases} \pi^{?k}(a \mid h) \text{ if } \Abs{h} < nT \\ \Pr\left[A^!_{\Abs{h}} = a \mid A\Theta^!_{:{\Abs{h}}} = h,\ K = k,\ \J^!_n = k\right] \text{ otherwise} \end{cases}$$

$$\pi^{!!k}_n(a \mid h):=\begin{cases} \pi^{?k}(a \mid h) \text{ if } \Abs{h} < nT \\ \pi^{!k}_n(a \mid h) + \pi^{!k}_n(\bot \mid h) \cdot \pi^k_n\left(a \mid \underline{h}\right) \text{ if } \Abs{h} \geq nT \text{ and } a \ne \bot \\ 0 \text{ if } \Abs{h} \geq nT \text{ and } a = \bot \end{cases}$$

Denote 

$$\rho^{*k}_n:=\mu^k\bowtie\pi^{*k}_n$$

$$\rho^{!!k}_n:=\nu^k\bowtie\pi^{!!k}_n$$ 

$$\rho^{!k}_n:=\nu^k\bowtie\pi^{!k}_n$$ 

$$\rho^{?k}:=\nu^k\bowtie\pi^{?k}$$ 

$$R^{?k}:=\EU^{*}_{\upsilon^k}(1-\alpha)-\EU^{\pi^{?k}}_{\bar{\upsilon}^k\AB{\sigma^k}}(1-\alpha)$$

For each $n \in \Nats$, denote

$$\EU_n^{*k}:=\frac{\alpha}{1-(1-\alpha)^T}\sum_{m=0}^{T-1} (1-\alpha)^{m}\E{x\sim\rho^{*k}_n}\left[r\left(x_{:nT+m}\right)\right]$$

$$\EU_n^{!!k}:=\frac{\alpha}{1-(1-\alpha)^T}\sum_{m=0}^{T-1} (1-\alpha)^{m}\E{x\sim\rho^{!!k}_n}\left[r\left(\underline{x}_{:nT+m}\right)\right]$$

$$\EU_n^{!k}:=\frac{\alpha}{1-(1-\alpha)^T}\sum_{m=0}^{T-1} (1-\alpha)^{m}\E{x\sim\rho^{!k}_n}\left[r\left(\underline{x}_{:nT+m}\right)\right]$$

$$\EU_n^{?k}:=\frac{\alpha}{1-(1-\alpha)^T}\sum_{m=0}^{T-1} (1-\alpha)^{m}\E{x\sim\rho^{?k}}\left[r\left(\underline{x}_{:nT+m}\right)\right]$$

We have

$$R^{?k} \leq \AP{1-(1-\alpha)^T}sum_{n=0}^\infty (1-\alpha)^{nT} \left(\EU^{*k}_n-\EU^{?k}_n\right) + \frac{2\tau^k\alpha}{1-(1-\alpha)^T}$$

$$R^{?k} \leq\AP{1-(1-\alpha)^T}\sum_{n=0}^\infty (1-\alpha)^{nT} \left(\EU^{*k}_n-\EU^{!!k}_n+\EU^{!!k}_n-\EU^{!k}_n+\EU^{!k}_n-\EU^{?k}_n\right) + \frac{2\tau^k\alpha}{1-(1-\alpha)^T}$$

Condition iv of Proposition B.N1 implies that, given $h \in \HD{\nu^k}$ s.t. $\Abs{h} \geq nT$

$$\Supp{\pi^{!k}_n(h)} \subseteq \AC{\pi^k\AP{S^k(h)},\bot}$$

$$\pi^{!!k}_n\AP{\pi^k\AP{S^k(h)} \mid h} = 1$$

Therefore, $\pi^{!!k}_n = \pi^{*k}_n$, and we remain with

$$R^{?k} \leq\AP{1-(1-\alpha)^T}\sum_{n=0}^\infty (1-\alpha)^{nT} \left(\EU^{!!k}_n-\EU^{!k}_n+\EU^{!k}_n-\EU^{?k}_n\right) + \frac{2\tau^k\alpha}{1-(1-\alpha)^T}$$

We have

$$\Abs{\EU^{!!k}_n-\EU^{!k}_n} \leq \Pr_{x\sim\rho^{!k}_n}\left[\exists m \in [T]: x_{nT+m} \in \bot\Ado\right]$$

Since $\Z_{nT}^{!}(K) \geq \eta$, it follows that

$$\Abs{\EU^{!!k}_n-\EU^{!k}_n} \leq \frac{1}{\eta}\Pr_{x\sim\rho^{?k}}\left[\exists m \in [T]: x_{nT+m} \in \bot\Ado\right] \leq \frac{1}{\eta}\Ea{x\sim\rho^{?k}}{\Abs{\AC{m \in [T] \mid x_{nT+m} \in \bot\Ado}}}$$

$$\sum_{n=0}^\infty \Abs{\EU^{!!k}_n-\EU^{!k}_n} \leq \frac{1}{\eta}\Ea{x\sim\rho^{?k}}{\Abs{\AC{n \in \Nats \mid x_n \in \bot\Ado}}}$$

Using condition i of Proposition B.N1, we conclude

$$R^{?k} \leq\AP{1-(1-\alpha)^T}\sum_{n=0}^\infty (1-\alpha)^{nT} \left(\EU^{!k}_n-\EU^{?k}_n\right) + O\AP{\frac{\AP{1-(1-\alpha)^T}\ln N}{\eta^2\epsilon}+\frac{\tau^k\alpha}{1-(1-\alpha)^T}}$$

Define the random variables $\Sqn{U^!_n : \Omega \rightarrow [0,1]}$ by 

$$U^!_n:=\frac{\alpha}{1-(1-\alpha)^T}\sum_{m=0}^{T-1} (1-\alpha)^{m} r\left(A\Theta^!_{:nT+m}\right)$$

Denote 

$$\bar{\tau} := \frac{1}{N}\sum_{k=0}^{N-1} {\tau^k}$$

$$\beta:=\frac{\AP{1-(1-\alpha)^T}\ln N}{\eta^2\epsilon}+\frac{\bar{\tau}\alpha}{1-(1-\alpha)^T}$$

Averaging the previous inequality over $k$, we *get

$$\frac{1}{N}\sum_{k=0}^{N-1}R^{?k} \leq \AP{1-(1-\alpha)^T}\sum_{n=0}^\infty (1-\alpha)^{nT} \E{}\left[\E{}\left[U^!_n \mid \J^!_n = K,\ Z^!_{nT}\right]-\E{}\left[U^!_n \mid Z^!_{nT}\right]\right] + O\AP{\beta}$$

$$\frac{1}{N}\sum_{k=0}^{N-1}R^{?k} = \sqrt{\AP{1-(1-\alpha)^T}\sum_{n=0}^\infty (1-\alpha)^{nT} \E{}\left[\left(\E{}\left[U^!_n \mid \J^!_n = K,\ Z^!_{nT}\right]-\E{}\left[U^!_n \mid Z^!_{nT}\right]\right)^2\right]} +  O\AP{\beta}$$

We apply Proposition B.N3 to each term in the sum over $n$.

$$\frac{1}{N}\sum_{k=0}^{N-1}R^{?k} = \sqrt{\AP{1-(1-\alpha)^T}\sum_{n=0}^\infty (1-\alpha)^{nT} \E{}\left[\frac{1}{2\eta}\I{}\left[K;\J^!_n,U^!_n \mid Z^!_{nT}\right]\right]} +  O\AP{\beta}$$

$$\frac{1}{N}\sum_{k=0}^{N-1}R^{?k} = \sqrt{\frac{1-(1-\alpha)^T}{2\eta}\sum_{n=0}^\infty \E{}\left[\En\Big(Z^!_{nT}\Big)-\En\left(Z^!_{(n+1)T}\right)\right]} + O\AP{\beta}$$

$$\frac{1}{N}\sum_{k=0}^{N-1}R^{?k} = O\left(\sqrt{\frac{1-(1-\alpha)^T}{\eta}\ln N}  +\frac{\AP{1-(1-\alpha)^T}\ln N}{\eta^2\epsilon}+\frac{\bar{\tau}\alpha}{1-(1-\alpha)^T}\right)$$

Condition ii of Proposition B.N1 implies that

$$\Dtv\left(\frac{1}{N}\sum_{k=0}^{N-1}{\bar{\nu}^k\left[\sigma^k\right]\bowtie\pi^*},\ \frac{1}{N}\sum_{k=0}^{N-1}{\bar{\nu}^k\left[\sigma^k\right]\bowtie\pi^{?k}}\right) \leq 2(N-1)\eta$$

Here, the factor of 2 comes from the difference between the equations for $Z_n$ and $Z^!_n$ (we can construct and intermediate policy between $\pi^*$ and $\pi^{?k}$ and use the triangle inequality for $\Dtv$). We conclude

$$\frac{1}{N}\sum_{k=0}^{N-1}\AP{\EU^{*}_{\upsilon^k}(1-\alpha)-\EU^{\pi^{*}}_{\bar{\upsilon}^k\AB{\sigma^k}}(1-\alpha)} = O\left(N\eta +\sqrt{\frac{1-(1-\alpha)^T}{\eta}\ln N}  +\frac{\AP{1-(1-\alpha)^T}\ln N}{\eta^2\epsilon}+\frac{\bar{\tau}\alpha}{1-(1-\alpha)^T}\right)$$

For each $k \in [N]$, we get

$$\EU^{*}_{\upsilon^k}(1-\alpha)-\EU^{\pi^{*}}_{\bar{\upsilon}^k\AB{\sigma^k}}(1-\alpha) = N \cdot O\left(N\eta +\sqrt{\frac{1-(1-\alpha)^T}{\eta}\ln N}  +\frac{\AP{1-(1-\alpha)^T}\ln N}{\eta^2\epsilon}+\frac{\bar{\tau}\alpha}{1-(1-\alpha)^T}\right)$$

Now we set 

$$\eta:=\alpha^{1/4} N^{-1/2} \AP{\ln N}^{1/4} \epsilon^{-1/4} \bar{\tau}^{1/4}$$  

$$T:=\Ceil{\alpha^{-1/4}N^{-1/2} \AP{\ln N}^{-1/4} \epsilon^{1/4} \bar{\tau}^{3/4}}$$

Without loss of generality, we can assume this to be consistent with the assumption $\eta < \frac{1}{N}$ since the bound contains a term of $N^2\eta$ anyway. Similarly, the second term in the bound implies we can assume that $1 - (1-\alpha)^T \ll 1$ and therefore $1 - (1-\alpha)^T = O(T\alpha)$. We get

$$\EU^{*}_{\upsilon^k}(1-\alpha)-\EU^{\pi^{*}}_{\bar{\upsilon}^k\AB{\sigma^k}}(1-\alpha) = O\AP{\AP{\alpha N^6 \ln{N} \epsilon^{-1} \bar{\tau}}^{1/4}}$$

\#Proposition A.N1

Fix $n \in \Nats$, $m \in [n]$. Then, for $x \in (0,1)$ we have

$$\Abs{\frac{\D}{\D x}\AP{x^m\frac{1-x}{1-x^n}}} \leq 1$$

\#Proof of Proposition A.N1

$$\frac{\D}{\D x}\AP{x^m\frac{1-x}{1-x^n}} = mx^{m-1}\frac{1-x}{1-x^n}+x^m\frac{\D}{\D x}\AP{\frac{1-x}{1-x^n}}$$

We will establish the claim by showing that the first term is in $[0,1]$ and the second term is in $[-1,0]$.

For the first term, we have (assuming w.l.o.g. that $m \geq 1$)

$$0 < mx^{m-1}\frac{1-x}{1-x^n} = \frac{mx^{m-1}}{\sum_{k=0}^{n-1} x^k} \leq \frac{mx^{m-1}}{\sum_{k=0}^{m-1} x^k} \leq \frac{mx^{m-1}}{\sum_{k=0}^{m-1} x^{m-1}}=1$$

For the second term, we have

$$\frac{\D}{\D x}\AP{\frac{1-x}{1-x^n}} = \frac{(-1)\cdot\AP{1-x^n}-(1-x)\cdot\AP{-nx^{n-1}}}{\AP{1-x^n}^2}=\frac{-(n-1)x^n+nx^{n-1}-1}{\AP{1-x^n}^2}$$

Assume w.l.o.g. that $n \geq 2$. First, we show that the numerator (and hence the entire fraction) is negative. Denote $p(x):=-(n-1)x^n+nx^{n-1}-1$. We have

$$\frac{\D{p(x)}}{\D x}=n(n-1)\AP{x^{n-2}-x^{n-1}}$$

Therefore, $p(x)$ is increasing in $[0,1]$. Observing that $p(1)=0$, we conclude that $p$ is negative in $(0,1)$.

Now we show that the fraction is $\geq -1$. Equivalently, we need to show that $p(x) + \AP{1-x^n}^2 \geq 0$. We have

$$p(x) + \AP{1-x^n}^2 = -(n-1)x^n+nx^{n-1}-1 + 1 - 2x^n + x^{2n} = x^{2n} - (n+1)x^n + nx^{n-1} = x^{n-1}\AP{x^{n+1}-(n+1)x+n}$$

Denote $q(x):=x^{n+1}-(n+1)x+n$. We have

$$\frac{\D{q(x)}}{\D x} = (n+1)\AP{x^n - 1}$$

Therefore, $q(x)$ is decreasing in $[0,1]$. Observing that $q(1) = 0$ we conclude that $q$ is positive in $(0,1)$, establishing the desired result.

\#Proposition A.N2

For any $x,y\in(0,1)$:

$$\frac{1-x}{\AP{1-xy}^2} \leq \frac{1}{1-y}$$

\#Proof of Proposition A.N2

Consider $p(z):=z^3-3z+2$. We have

$$\frac{\D{p(z)}}{\D z} = 3z^2 - 3$$

$p(z)$ has a local minimum at $z=1$ and a local maximum at $z = -1$. Therefore, for $z \geq -1$, $p(z) \geq p(1) = 0$. For $z \geq 0$, we get

$$z^2 \AP{z^2 - 3} + 2z = z^4-3z^2+2z = zp(z) \geq 0$$

Taking $z := \frac{1}{2}(x+y)$, we get

$$\AP{\frac{x+y}{2}}^2 \AP{\AP{\frac{x+y}{2}}^2 - 3} + x + y \geq 0$$

Now, $\frac{1}{2}(x+y) \geq \sqrt{xy}$ and $t(t-3)$ is a decreasing function of $t$ for $t\in\AB{0,\frac{3}{2}}$. We conclude

$$xy \AP{xy - 3} + x + y \geq 0$$

$$x^2y^2 - 3xy + x + y \geq 0$$

$$x^2 y^2 - 2xy + 1 \geq xy - x - y + 1$$

$$\AP{1-xy}^2 \geq (1-x)(1-y)$$

$$\frac{1}{1-y} \geq \frac{1-x}{\AP{1-xy}^2}$$

\#Proof of Proposition 2

$$\V_M(s,\gamma) = (1-\gamma) \sum_{n=0}^\infty \gamma^n \sum_{t \in \St_M} \T_{M\pi}^n (t \mid s) \R_M(t)$$

By Proposition 1, we have

$$\sum_{t \in \St_M} \T_{M\pi}^n(t \mid s) \R_M(t) = \Ea{k \sim \zeta(s)}{\sum_{t \in \St_M} \T_{M\pi}^n \xi(t \mid s,k) \R_M(t)} + \delta_n$$

where, $\Abs{\delta_n} \leq F \lambda^n$. Denote

$$\V_M^{\text{per}}(s,\gamma) := (1-\gamma) \sum_{n=0}^\infty \gamma^n \delta_n$$

$$\V_{Mk}^{\text{cyc}}(s,\gamma) := (1-\gamma) \sum_{n=0}^\infty \gamma^n \sum_{t \in \St_M} \T_{M\pi}^n \xi(t \mid s,k) \R_M(t)$$

We get

$$\V_M(s,\gamma) = \Ea{k \sim \zeta(s)}{\V_{Mk}^{\text{cyc}}(s,\gamma)} + \V_M^{\text{per}}(s,\gamma) $$

$$\frac{\D{\V_M(s,\gamma)}}{\D\gamma} = \Ea{k \sim \zeta(s)}{\frac{\D{\V_{Mk}^{\text{cyc}}(s,\gamma)}}{\D\gamma}} + \frac{\D{\V_M^{\text{per}}(s,\gamma)}}{\D\gamma}$$

We now analyze each term separately.

Denote

$$r_{kn}:=\sum_{t \in \St_M} \T_{M\pi}^n \xi(t \mid s,k) \R_M(t)$$

Note that, due to the property of $\xi$, $r_{kn}$ is periodic in $n$ with period $k$. We have

$$\V_{Mk}^{\text{cyc}}(s,\gamma) := (1-\gamma) \sum_{n=0}^{\infty} \gamma^{n} r_{kn} = (1-\gamma) \sum_{i=0}^{k-1}\sum_{n=0}^\infty \gamma^{nk+i} r_{ki} = \sum_{i=0}^{k-1} \frac{(1-\gamma)\gamma^i}{1-\gamma^k}r_{ki}$$

$$\Abs{\frac{\D{\V_{Mk}^{\text{cyc}}(s,\gamma)}}{\D\gamma}} \leq \sum_{i=0}^{k-1} \Abs{\frac{\D}{\D\gamma}\AP{\frac{(1-\gamma)\gamma^i}{1-\gamma^k}}}r_{ki}$$

By Proposition A.N1, we get

$$\Abs{\frac{\D{\V_{Mk}^{\text{cyc}}(s,\gamma)}}{\D\gamma}} \leq \sum_{i=0}^{k-1} r_{ki} \leq k$$

Now we analyze the second term.

$$\V_M^{\text{per}}(s,\gamma) = \delta_0 + \sum_{n=0}^\infty \gamma^{n+1} \AP{\delta_{n+1} - \delta_n}$$

$$\frac{\D{\V_M^{\text{per}}(s,\gamma)}}{\D\gamma} = \sum_{n=0}^\infty (n+1)\gamma^n \AP{\delta_{n+1} - \delta_n}=\sum_{n=0}^\infty \AP{n\gamma^{n-1}-(n+1)\gamma^n} \delta_n = \sum_{n=0}^\infty \AP{\AP{1-\gamma}n\gamma^{n-1}-\gamma^n} \delta_n$$

$$\Abs{\frac{\D{\V_M^{\text{per}}(s,\gamma)}}{\D\gamma}} \leq F \sum_{n=0}^\infty \AP{\AP{1-\gamma}n\gamma^{n-1}+\gamma^n} \lambda^n=F\AP{(1-\gamma)\sum_{n=0}^\infty n \gamma^{n-1}\lambda^n+\frac{1}{1-\gamma\lambda}}$$

$$\Abs{\frac{\D{\V_M^{\text{per}}(s,\gamma)}}{\D\gamma}} \leq F\AP{(1-\gamma)\frac{\D}{\D\gamma}\sum_{n=0}^\infty \gamma^n\lambda^n+\frac{1}{1-\gamma\lambda}}=F\AP{(1-\gamma)\frac{\D}{\D\gamma}\AP{\frac{1}{1-\gamma\lambda}}+\frac{1}{1-\gamma\lambda}}$$

$$\Abs{\frac{\D{\V_M^{\text{per}}(s,\gamma)}}{\D\gamma}} \leq F\AP{\lambda\cdot\frac{1-\gamma}{\AP{1-\gamma\lambda}^2}+\frac{1}{1-\gamma\lambda}}$$

Applying Proposition A.N2 to the first term, we get

$$\Abs{\frac{\D{\V_M^{\text{per}}(s,\gamma)}}{\D\gamma}} \leq F\AP{\lambda\cdot\frac{1}{1-\lambda}+\frac{1}{1-\gamma\lambda}} \leq F\frac{1+\lambda}{1-\lambda}$$

TBD

\section{Appendix B}

The following appeared in a [previous essay](https://agentfoundations.org/item?id=1723) as "Proposition C.1".

\#Proposition B.N1

Fix an interface $\In=(\A,\Ob)$, $N \in \Nats$, $\epsilon \in (0,\frac{1}{\Abs{\A}})$, $\eta \in (0,\frac{1}{N})$. Consider some $\{\sigma^k: \FH \M \A\}_{k \in [N]}$. Then, there exist $\Dl: \Adfh \times \A \rightarrow \Ada$ and $\{\Dl^{!k}: \Adfh \times \A \rightarrow \Ada\}_{k \in [N]}$ with the following properties. Given $x \in \left(2^\A \times \Adao\right)^*$, we denote $\underline{x}$ its projection to $\Adfh$. Thus, $\underline{\underline{x}}\in\FH$.
Given  $\mu$ an $\In$-environment, $\pi: \HD{\mu} \M \A$, $\Dl': \Adfh \times \A \rightarrow \Ada$ and $k \in [N]$, we can define $\Xi\left[\mu,\sigma^k,\Dl',\pi\right]\in \Delta\left(\A \times \Adao\right)^\omega$ as follows
 
$$\Xi\left[\mu,\sigma^k,\Dl',\pi\right]\left(b,a,o \mid x\right):=\pi\left(b \mid \underline{\underline{x}}\right)\Dl'\left(a \mid \underline{x},b\right) \bar{\mu}[\sigma^k]\left(o \mid \underline{x}a\right)$$

We require that for every $\pi$, $\mu$ and $k$ as above, the following conditions hold

i. $$\E{x \sim\Xi\left[\mu,\sigma^k,\Dl^{!k},\pi\right]}\left[\Abs{\{n \in \Nats \mid x_n \in \A \times \bot \times \bar{\Ob}\}}\right] \leq \frac{\ln N}{\eta \ln\left(1 + \epsilon(1-\epsilon)^{(1-\epsilon)/\epsilon}\right)}=O\left(\frac{\ln N}{\eta \epsilon}\right)$$

ii. $\Dtv\left(\frac{1}{N}\sum_{j=0}^{N-1}{\Xi\left[\mu,\sigma^j,\Dl,\pi\right]},\frac{1}{N}\sum_{j=0}^{N-1}{\Xi\left[\mu,\sigma^j,\Dl^{!j},\pi\right]}\right) \leq (N-1)\eta$

iii. For all $x \in \HD{\bar{\mu}[\sigma^k]}$, if $\Dl^{!k}\left(x,\pi\left(\underline{x}\right)\right) \ne \bot$ then $\sigma^k\left(\Dl^{!k}\left(x,\pi\left(\underline{x}\right)\right) \mid \underline{x}\right) > 0$

iv. For all $x \in \HD{\bar{\mu}[\sigma^k]}$, if $\Dl^{!k}\left(x,\pi\left(\underline{x}\right)\right) \not\in \AC{\pi\left(\underline{x}\right), \bot}$ then $\sigma^k\left(\pi\left(\underline{x}\right) \mid \underline{x}\right) \leq \epsilon$

***

The following is a simple modification of what appeared there as "Proposition B.2" (the corresponding modification of the proof is trivial and we leave it out.)

\#Proposition B.N2

Consider some $\gamma\in(0,1)$, $\tau\in(0,\infty)$, $T\in\Nats^+$, a universe $\upsilon=(\mu,r)$ that is an $\Ob$-realization of $M$ with state function $S$, some $\pi^*: \HD{\mu} \rightarrow \A$ and some $\pi^0: \HD{\mu} \M \A$. Assume that $\gamma \geq \gamma_M$. For any $n \in \Nats$, let $\pi^*_n$ be an $\In$-policy s.t. for any $h \in \HD{\mu}$

$$\pi^*_n(h):=\begin{cases} \pi^0(h) \text{ if } \Abs{h} < nT \\ \pi^*(h) \text{ otherwise} \end{cases}$$

Assume that for any $h \in \HD{\mu}$

i. $$\pi^*(s) \in \A_{M}^\omega\AP{S(h)}$$

ii. $$\Supp{\pi^0(h)} \subseteq \A_{M}^0\AP{S(h)}$$

iii. For any $\theta\in(\gamma,1)$ $$\Abs{\frac{\D\V_{M}\AP{S(h),\theta}}{\D\theta}} \leq \tau$$

Then

$$\EU^{*}_\upsilon(\gamma)-\EU^{\pi^0}_\upsilon(\gamma) \leq (1-\gamma)\sum_{n=0}^\infty \sum_{m=0}^{T-1} \gamma^{nT+m}\left(\E{x\sim\mu\bowtie\pi^*_n}\left[r\left(x_{:nT+m}\right)\right]-\E{x\sim\mu\bowtie\pi^0}\left[r\left(x_{:nT+m}\right)\right]\right) + \frac{2\tau\gamma^T(1-\gamma)}{1-\gamma^T}$$

***

The following appeared [previously](https://agentfoundations.org/item?id=1656) as "Proposition A.1".

\#Proposition B.N3

Consider a probability space $(\Omega, P \in \Delta\Omega)$, $N \in \Nats$, $R \subseteq [0,1]$ a finite set and random variables $U: \Omega \rightarrow R$, $K: \Omega \rightarrow [N]$ and $\J: \Omega \rightarrow [N]$. Assume that $K_*P = J_*P = \zeta \in \Delta[N]$ and $\I{}[K;J] = 0$. Then

$$\I{}\left[K;J,U\right] \geq 2 \left(\min_{i \in [N]} {\zeta(i)}\right) \left(\E{}\left[U \mid J = K\right]-\E{}\left[U\right]\right)^2$$

\end{document}



