%&latex
\documentclass[a4paper]{article}

\usepackage[a4paper,margin=1in]{geometry}
\usepackage[affil-it]{authblk}
\usepackage{cite}
\usepackage[unicode]{hyperref}
\usepackage[utf8]{inputenc}
\usepackage[english]{babel}
\usepackage{csquotes}
\usepackage{amsmath,amssymb,amsthm}
\usepackage{enumerate}
\usepackage{commath}

\newcommand{\Comment}[1]{}

\newcommand{\Bool}{\{0,1\}}
\newcommand{\Words}{{\Bool^*}}

% operators that are separated from the operand by a space
\DeclareMathOperator{\Sgn}{sgn}
\DeclareMathOperator{\Supp}{supp}
\DeclareMathOperator{\Stab}{stab}
\DeclareMathOperator{\Img}{Im}
\DeclareMathOperator{\Dom}{dom}

% autosize deliminaters
\newcommand{\AP}[1]{\left(#1\right)}
\newcommand{\AB}[1]{\left[#1\right]}
\newcommand{\AC}[1]{\left\{#1\right\}}

% operators that require brackets
\newcommand{\Pa}[2]{\underset{#1}{\operatorname{Pr}}\AB{#2}}
\newcommand{\PP}[2]{\underset{\substack{#1 \\ #2}}{\operatorname{Pr}}}
\newcommand{\PPP}[3]{\underset{\substack{#1 \\ #2 \\ #3}}{\operatorname{Pr}}}
\newcommand{\E}[1]{\underset{#1}{\operatorname{E}}}
\newcommand{\Ea}[2]{\underset{#1}{\operatorname{E}}\AB{#2}}
\newcommand{\EE}[2]{\underset{\substack{#1 \\ #2}}{\operatorname{E}}}
\newcommand{\EEE}[3]{\underset{\substack{#1 \\ #2 \\ #3}}{\operatorname{E}}}
\newcommand{\I}[1]{\underset{#1}{\operatorname{I}}}
\newcommand{\Ia}[2]{\underset{#1}{\operatorname{I}}\AB{#2}}
\newcommand{\II}[2]{\underset{\substack{#1 \\ #2}}{\operatorname{I}}}
\newcommand{\III}[3]{\underset{\substack{#1 \\ #2 \\ #3}}{\operatorname{I}}}
\newcommand{\Var}{\operatorname{Var}}

% operators that require parentheses
\newcommand{\En}{\operatorname{H}}
\newcommand{\Ena}[1]{\operatorname{H}\AP{#1}}
\newcommand{\Hom}{\operatorname{Hom}}
\newcommand{\End}{\operatorname{End}}
\newcommand{\Sym}{\operatorname{Sym}}

\newcommand{\Prj}{\operatorname{pr}}

\newcommand{\D}{\mathrm{d}}
\newcommand{\KL}[2]{\operatorname{D}_{\mathrm{KL}}(#1 \| #2)}
\newcommand{\Dtv}{\operatorname{d}_{\text{tv}}}
\newcommand{\Dtva}[1]{\operatorname{d}_{\text{tv}}\AP{#1}}

\newcommand{\Argmin}[1]{\underset{#1}{\operatorname{arg\,min}}\,}
\newcommand{\Argmax}[1]{\underset{#1}{\operatorname{arg\,max}}\,}

\newcommand{\Nats}{\mathbb{N}}
\newcommand{\Ints}{\mathbb{Z}}
\newcommand{\Rats}{\mathbb{Q}}
\newcommand{\Reals}{\mathbb{R}}
\newcommand{\Coms}{\mathbb{C}}

\newcommand{\Sq}[2]{\{#1\}_{#2 \in \Nats}}
\newcommand{\Sqn}[1]{\Sq{#1}{n}}

\newcommand{\Estr}{\boldsymbol{\lambda}}

\newcommand{\Lim}[1]{\lim_{#1 \rightarrow \infty}}
\newcommand{\LimInf}[1]{\liminf_{#1 \rightarrow \infty}}
\newcommand{\LimSup}[1]{\limsup_{#1 \rightarrow \infty}}

\newcommand{\Abs}[1]{\left\vert #1 \right\vert}
\newcommand{\Norm}[1]{\left\Vert #1 \right\Vert}
\newcommand{\Floor}[1]{\left\lfloor #1 \right\rfloor}
\newcommand{\Ceil}[1]{\left\lceil #1 \right\rceil}
\newcommand{\Chev}[1]{\left\langle #1 \right\rangle}
\newcommand{\Quote}[1]{\left\ulcorner #1 \right\urcorner}

\newcommand{\Alg}{\xrightarrow{\text{alg}}}
\newcommand{\M}{\xrightarrow{\text{k}}}
\newcommand{\PF}{\xrightarrow{\circ}}

% Paper specific

\newcommand{\Ob}{\mathcal{O}}
\newcommand{\A}{\mathcal{A}}
\newcommand{\St}{\mathcal{S}}
\newcommand{\T}{\mathcal{T}}
\newcommand{\R}{\mathcal{R}}
\newcommand{\In}{\mathcal{I}}
\newcommand{\FH}{(\A \times \Ob)^*}
\newcommand{\IH}{(\A \times \Ob)^\omega}
\newcommand{\Ado}{\bar{\Ob}}
\newcommand{\Ada}{\bar{\A}}
\newcommand{\Adi}{{\bar{\In}}}
\newcommand{\Adao}{\overline{\A \times \Ob}}
\newcommand{\Adfh}{\Adao^*}
\newcommand{\Adih}{\Adao^\omega}
\DeclareMathOperator{\HD}{hdom}
\newcommand{\Hy}{\mathcal{H}}
\newcommand{\UC}{\mathcal{U}}

\newcommand{\RMC}{\mathrm{C}}
\newcommand{\RMD}{\mathrm{D}}
\newcommand{\RME}{\mathrm{E}}
\newcommand{\RMF}{\mathrm{F}}

\newcommand{\SF}{\St^{\RMF}}
\newcommand{\SD}{\St^{\RMD}}
\newcommand{\SC}{\St^{\RMC}}
\newcommand{\MF}{M^{\RMF}}
\newcommand{\MD}{M^{\RMD}}
\newcommand{\ME}{M^{\RME}}
\newcommand{\TF}{\bar{\tau}^{\RMF}}
\newcommand{\PD}{\pi^{\RMD}}
\newcommand{\UD}{\upsilon^{\RMD}}

\newcommand{\Ut}{\operatorname{U}}
\newcommand{\V}{\operatorname{V}}
\newcommand{\Q}{\operatorname{Q}}
\newcommand{\EU}{\operatorname{EU}}

\newcommand{\Dl}{\mathcal{D}}
\newcommand{\Do}{\mathfrak{D}}
\newcommand{\F}{\mathcal{F}}
\newcommand{\B}{\mathcal{B}}
\newcommand{\Z}{Z}
\newcommand{\J}{J}

\newcommand{\Pd}{P}

\begin{document}

We derive a regret bound for [DRL](https://agentfoundations.org/item?id=1656) reflecting dependence on:

* Number of hypotheses

* Mixing time of MDP hypotheses

* The probability with which the advisor takes optimal actions

That is, the regret bound we get is fully explicit up to a multiplicative constant (which can also be made explicit). Currently we focus on plain (as opposed to [catastrophe](https://agentfoundations.org/item?id=1715)) and uniform (finite number of hypotheses, uniform prior) DRL, although this result can and should be extended to the catastrophe and/or non-uniform settings.

***

Appendix A contains the proofs...

\section{Notation}

Whatever...

\section{Results}

First, we briefly recall some properties of Markov chains.

\#Definition 1

Consider $\St$ a finite set and $\T: \St \M \St$. We say that $k \in \Nats^+$ is a *period of $\T$* when there is $s \in \St$ an essential state of $\T$ (that is, $\T^\infty(s \mid s) > 0$) s.t. $k$ is its period, i.e. $k = \gcd \AC{n \in \Nats^+ \mid \T^n(s \mid s) > 0}$. We denote $\Pd_\T$ the set of periods of $\T$.

***

The following is a corollary of the Perron-Frobenius theorem which we give without proof. *[I believe this is completely standard and would be grateful to get a source for this which treats the reducible case; of course I can produce the proof but it seems redundant.]*

\#Proposition 1

Consider $\St$ a finite set and $\T: \St \M \St$. Then, there are $F_\T \in (0,\infty)$ and $\lambda_\T\in(0,1)$ s.t. for any $s \in \St$, there are some $\zeta_s \in \Delta\Pd_\T$ and $\xi_s: \Pd_\T \M \St$ s.t. for any $n \in \Nats$

$$\forall k \in \Pd_T: \T^k \xi_s(k) = \xi_s(k)$$

$$\Dtva{\T^n(s),\Ea{k \sim \zeta_s}{\T^n \xi_s(k)}} \leq F_\T \lambda_{\T}^{-n}$$

***

For the purpose of this write-up, we will use a definition of local sanity slightly stronger than what [previously](https://agentfoundations.org/item?id=1656) appeared as "Definition 4." We think this strengthening is not substantial, but also the current analysis can be generalized to the weaker case by adding a term proportional to the 2nd derivative of $\V$ (or the 2nd moment of the mixing time). We leave out the details for the time being.

We will use the notation $\A_M^\omega(s) := \bigcap_{k \in \Nats} \A_M^k(s)$ (this is always non-empty).

\#Definition 2

Let $\upsilon = (\mu,r)$ be a universe and $\epsilon > 0$. A policy $\pi$ is said to be *locally $\epsilon$-sane for $\upsilon$* when there are $M$, $S$ and $\UC_M \subseteq \St_M$ (the set of uncorrupt states) s.t. $\upsilon$ is an $\Ob$-realization of $M$ with state function $S$, $S(\Estr) \in \UC_M$ and for any $h \in \HD{\mu}$, if $S(h) \in \UC_M$ then

i. $$\forall a \in \Supp{\pi(h)}: \T_M\AP{\UC_M \mid S(h),a} = 1$$

ii. $$\Supp{\pi(h)} \subseteq \A_M^0\left(S(h)\right)$$

iii. $$\exists a \in \A_M^\omega\left(S(h)\right): \pi(a \mid h) > \epsilon$$

***

We obtain the following regret bound.

\#Theorem 1

There is some $C \in (0,\infty)$ s.t. the following holds.

Consider $\In$ an interface, $\epsilon \in (0,1)$ and $\Hy = \{\upsilon^k = (\mu^k,r^k) \in \Upsilon_{\In}\}_{k \in [N]}$ for some $N \in \Nats$. Assume that for each $k \in [N]$, $\sigma^k$ is locally $\epsilon$-sane for $\upsilon^k$. For each $k \in [N]$, let $M^k$ be the corresponding MDP and $\UC^k \subseteq \St_{M^k}$ the corresponding set of uncorrupt states. Assume further that for any $k,j \in \Nats$ and $h \in \HD{\mu^k} \cap \HD{\mu^j}$, if $S^k(h) \in \UC^k$ and $S^j(h) \in \UC^j$, then $r^k(h)=r^j(h)$. Then, there is an $\bar{\In}$-policy $\pi^*$ s.t. for any $k \in [N]$ and $\alpha \in (0,1)$, if $\alpha \ll 1$ then

$$\EU_{\upsilon^k}^*(1-\alpha) - \EU_{\bar{\upsilon}^k\left[\sigma^k\right]}^{\pi^*}(1-\alpha) \leq C\AP{\frac{\alpha N^5 \ln{N} \sum_{j = 0}^{N-1} \max_{s \in \UC_k} \Abs{\V^1_{M^k}(s)}}{\epsilon}}^{1/4}$$

***

Here, the assumption $\alpha \ll 1$ is used only to simplify Definition 2. If we replace $\A_M^\omega$ in condition iii by the set of actions optimal for the *particular value of $\alpha$*, then we can drop this assumption. 

The factor $\max_{s \in \UC_k} \Abs{\V^1_{M^k}(s)}$ might seem difficult to understand. However, it can be bounded as follows.

\#Proposition 2

Let $M$ be an MDP, $\pi$ a Blackwell optimal policy for $M$ and $F \in (0,\infty)$, $\lambda \in (0,1)$, $\Pd \subseteq \Nats^+$ as in Proposition 1 applied to the Markov chain $\T_{M\pi}$. Then

$$\max_{s \in \St_M} {\Abs{\V_M^1(s)}} = O\AP{\frac{F}{1-\lambda}+\max{\Pd}}$$

***

Theorem 1 and Proposition 2 immediately give the following:

\#Corollary 1

There is some $C' \in (0,\infty)$ s.t. the following holds.

Under the conditions of Theorem 1, let $\pi^k$ be a Blackwell optimal policy for $M^k$ s.t. 

$$\T_{M^k \pi^k}\AP{\UC^k \mid \UC^k} = 1$$ 

Assuming w.l.o.g. all uncorrupt states are reachable from $S^k(\Estr)$, $\pi^k$ is guaranteed to exist thanks to condition iii of Definition 2 (if some uncorrupt state is unreachable, we can consider it to be corrupt.) Let $F^k\in(0,\infty)$, $\lambda^k\in(0,1)$ and $\Pd^k \subseteq \Nats^+$ be as in Proposition 1, for the Markov chain $\T_{M^k\pi^k}: \UC^k \M \UC^k$. Then, there is an $\bar{\In}$-policy $\pi^*$ s.t. for any $k \in [N]$ and $\alpha \in (0,1)$, if $\alpha \ll 1$ then

$$\EU_{\upsilon^k}^*(1-\alpha) - \EU_{\bar{\upsilon}^k\left[\sigma^k\right]}^{\pi^*}(1-\alpha) \leq C'\AP{\frac{\alpha N^5 \ln{N} \sum_{j = 0}^{N-1} \AP{\frac{F^k}{1-\lambda^k}+\max{\Pd^k}}}{\epsilon}}^{1/4}$$

\section{Appendix A}

\#Proposition A.1

Hahaha

\#Proof of Proposition A.1

Mwhahaha

\end{document}



