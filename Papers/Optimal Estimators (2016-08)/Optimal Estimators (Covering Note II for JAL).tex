%&latex
\documentclass{letter}

\usepackage{csquotes}

%+Make Labels
\makelabels
%-Make Labels
\usepackage[normalem]{ulem}
\usepackage{color}

\begin{document}

% 04/10/03 name and address are moved out from preamble to enjoy TeXWord ...

%+Your Name
\name{Vadim Kosoy} % Your name, used for printing on the envelope together with the return address
%-Your Name

%+Return Address
\address{}
%-Return Address

\begin{letter}{To the Journal of Applied Logic} % Receiver Address

\opening{Dear editors and/or referees,}

Per your request, this is annotated version of the referee reports, explaining how I addressed each point. The relevant paragraphs in the reports are in red, my comments are in green.

\uline{\textbf{Report 1}}
\begin{verbatim}

EVALUATION

The paper gives complexity-theoretic definitions of optimal prediction 
and estimation, with the goal of recovering something resembling a 
theory of probabilistic belief under uncertainty, where the uncertainty 
is due to computational limitations as opposed to information-theoretic 
limitations. That is, there is an intuition that good reasoning under 
uncertainty due to insufficient information may share some structure 
with good reasoning under uncertainty due to insufficient computational 
resources; and this intuition is explored by studying the values 
assigned by optimal computationally bounded algorithms for estimating 
hard-to-compute functions, as though they were probabilities (or 
estimates of random variables).

I regard the main novelties of the paper to be the definition of optimal 
estimators and the non-triviality of the definition (existence); the 
observation that they satisfy basic properties of probability theory 
(Section 3); and the theorems of uniqueness and uniqueness of 
conditionals, which support the view that there are objectively correct 
probabilities to assign to the outcomes of some computations. These seem 
like technically non-trivial and philosophically interesting results, 
with room for significant further work. They also seem potentially 
useful for abstractly modeling intelligent systems; for example, if a 
problem class faced by some intelligent systems admits an optimal 
estimator, then uniqueness potentially gives ways to compare reasoning 
across those systems (because there is one true best set of beliefs to 
hold about that problem class).

\end{verbatim}
\color{red}
\begin{verbatim}

Some of the theorems are hard to parse; it may be helpful to add brief 
English translations of the theorem statements, even if they gloss over 
important details.

\end{verbatim}
\color[rgb]{0,0.501961,0}
\begin{verbatim}

I'm not sure how to fix this, also not sure which theorems are the difficult ones.

\end{verbatim}
\color{red}
\begin{verbatim}

The relevant work section is good, but might be
improved by a bit more comparison with average-case complexity.

\end{verbatim}
\color[rgb]{0,0.501961,0}
\begin{verbatim}

I'm not sure what specifically should be added.

\end{verbatim}
\color{black}
\begin{verbatim}

ERRATA

\end{verbatim}
\color{red}
\begin{verbatim}

p3-4:
The paper says "we avoid choosing a specific category of mathematical 
questions", but then chooses questions that are concrete in the sense of 
being about (computable) valuations on the space of bitstrings, as 
opposed to e.g. statements about infinite objects; it might be good to 
clarify this restriction.

\end{verbatim}
\color[rgb]{0,0.501961,0}
\begin{verbatim}

This is somewhat inaccurate, since the formalism is applicable to
uncomputable valuations as well (e.g. we could consider the decision
problem of true sentences in ZFC with some family of distributions on
sentences). Many or most of the interesting examples are indeed
computable, but e.g. Theorem 5.1 doesn't assume computability. To
address the comment, I added footnote 2 which reads "We do require
that these questions can be represented as finite strings of bits."

\end{verbatim}
\color{red}
\begin{verbatim}

p4:
(Specifically, we will consider the resources of time, random and 
advice.)
(Specifically, we will consider the resources of time, randomness
and advice.)

\end{verbatim}
\color[rgb]{0,0.501961,0}
\begin{verbatim}

Fixed as requested.

\end{verbatim}
\color{red}
\begin{verbatim}

p5:
Again we can consider the corresponding weaker condition
Again we can consider the corresponding weaker condition that for all Q

\end{verbatim}
\color[rgb]{0,0.501961,0}
\begin{verbatim}

Fixed as requested.

\end{verbatim}
\color{red}
\begin{verbatim}

p5, above the last displayed eq'n:
<< S : N^n � {0, 1}*}...
S : N2 � {0, 1}*...
p7, last paragraph:
<< the a instance
an instance
p8:
<< determined but the space of functions
determined by the space of functions
<< an F1/2#(?)-optima polynomial-time
an F1/2#(?)-optimal polynomial-time
p9:
<< Unique Games Conjugate
Unique Games Conjecture
p10, end of 1.1:
<< sgn : R ? {0, 1}
sgn : R ? {-1, 1}
p18:
<< 2.2.1 Optimality Relatively to Uniform Families
2.2.1 Optimality Relative to Uniform Families
p19, proof of 2.10:
<< As easy to see
As is easy to see
p20, after 2.12:
<< To relationship to the role
The relationship to the role
p55, bottom:
<< is a fall space
is a fall space.

\end{verbatim}
\uline{\textbf{Report 2}}

TBD

\signature{Vadim Kosoy}
\closing{Sincerely,}

\end{letter}

\end{document}


