%&latex
\documentclass{letter}

\usepackage{csquotes}

%+Make Labels
\makelabels
%-Make Labels
\usepackage[normalem]{ulem}
\usepackage{color}

\begin{document}

% 04/10/03 name and address are moved out from preamble to enjoy TeXWord ...

%+Your Name
\name{Vadim Kosoy} % Your name, used for printing on the envelope together with the return address
%-Your Name

%+Return Address
\address{}
%-Return Address

\begin{letter}{To the Journal of Applied Logic} % Receiver Address

\opening{Dear editors and/or referees,}

Per your request, this is annotated version of the referee reports, explaining how I addressed each point. The relevant paragraphs in the reports are in red, my comments are in green.

\uline{\textbf{Report 1}}
\begin{verbatim}
EVALUATION

The paper gives complexity-theoretic definitions of optimal prediction 
and estimation, with the goal of recovering something resembling a 
theory of probabilistic belief under uncertainty, where the uncertainty 
is due to computational limitations as opposed to information-theoretic 
limitations. That is, there is an intuition that good reasoning under 
uncertainty due to insufficient information may share some structure 
with good reasoning under uncertainty due to insufficient computational 
resources; and this intuition is explored by studying the values 
assigned by optimal computationally bounded algorithms for estimating 
hard-to-compute functions, as though they were probabilities (or 
estimates of random variables).

I regard the main novelties of the paper to be the definition of optimal 
estimators and the non-triviality of the definition (existence); the 
observation that they satisfy basic properties of probability theory 
(Section 3); and the theorems of uniqueness and uniqueness of 
conditionals, which support the view that there are objectively correct 
probabilities to assign to the outcomes of some computations. These seem 
like technically non-trivial and philosophically interesting results, 
with room for significant further work. They also seem potentially 
useful for abstractly modeling intelligent systems; for example, if a 
problem class faced by some intelligent systems admits an optimal 
estimator, then uniqueness potentially gives ways to compare reasoning 
across those systems (because there is one true best set of beliefs to 
hold about that problem class).
\end{verbatim}
\color{red}
\begin{verbatim}
Some of the theorems are hard to parse; it may be helpful to add brief 
English translations of the theorem statements, even if they gloss over 
important details.
\end{verbatim}
\color[rgb]{0,0.501961,0}
\begin{verbatim}
I'm not sure how to fix this, also not sure which theorems are the
difficult ones.
\end{verbatim}
\color{red}
\begin{verbatim}
The relevant work section is good, but might be
improved by a bit more comparison with average-case complexity.
\end{verbatim}
\color[rgb]{0,0.501961,0}
\begin{verbatim}
I'm not sure what specifically should be added.
\end{verbatim}
\color{black}
\begin{verbatim}
ERRATA
\end{verbatim}
\color{red}
\begin{verbatim}
p3-4:
The paper says "we avoid choosing a specific category of mathematical 
questions", but then chooses questions that are concrete in the sense of 
being about (computable) valuations on the space of bitstrings, as 
opposed to e.g. statements about infinite objects; it might be good to 
clarify this restriction.
\end{verbatim}
\color[rgb]{0,0.501961,0}
\begin{verbatim}
This is somewhat inaccurate, since the formalism is applicable to
uncomputable valuations as well (e.g. we could consider the decision
problem of true sentences in ZFC with some family of distributions on
sentences). Many or most of the interesting examples are indeed
computable, but e.g. Theorem 5.1 doesn't assume computability. To
address the comment, I added footnote 2 which reads "We do require
that these questions can be represented as finite strings of bits."
\end{verbatim}
\color{red}
\begin{verbatim}
p4:
(Specifically, we will consider the resources of time, random and 
advice.)
(Specifically, we will consider the resources of time, randomness
and advice.)
p5:
Again we can consider the corresponding weaker condition
Again we can consider the corresponding weaker condition that for all Q
p5, above the last displayed eq'n:
<< S : N^n � {0, 1}*}...
S : N2 � {0, 1}*...
p7, last paragraph:
<< the a instance
an instance
p8:
<< determined but the space of functions
determined by the space of functions
<< an F1/2#(?)-optima polynomial-time
an F1/2#(?)-optimal polynomial-time
p9:
<< Unique Games Conjugate
Unique Games Conjecture
p10, end of 1.1:
<< sgn : R ? {0, 1}
sgn : R ? {-1, 1}
p18:
<< 2.2.1 Optimality Relatively to Uniform Families
2.2.1 Optimality Relative to Uniform Families
p19, proof of 2.10:
<< As easy to see
As is easy to see
p20, after 2.12:
<< To relationship to the role
The relationship to the role
p55, bottom:
<< is a fall space
is a fall space.
\end{verbatim}
\color[rgb]{0,0.501961,0}
\begin{verbatim}
All of the above were fixed as requested.
\end{verbatim}
\color{black}
\uline{\textbf{Report 2}}
\begin{verbatim}
The paper under review addresses the following question. Suppose that I 
have a hard-to-compute decision problem L, or, more generally, a 
hard-to-compute function f(.). Given x, the statement "x in L" is 
definitely either true or false, and, similarly, the value f(x) is a 
fixed value. A computationally bounded party, however, might be unable 
to decide whether x is in L, or what is the value f(x), and would thus 
have a certain amount of "uncertainty" about whether x is in L and about 
the value f(x). Is there a notion of "computationally-bounded Bayesian 
reasoning" that can be used to talk about the "probability"

 Pr [ x in L ]

 or the "expectation"

  E [ f(x) ]

 from the point of such a computationally bounded entity?

The author casts this problem in the framework of average-case 
complexity, which appears to be the right framing of the question. Given 
a function f and a distribution D of inputs, the author calls an 
algorithm A an optimal estimator for a certain complexity bound if A 
runs within such a bound and the error

(1)  E_{x ~ D} [  (A(x) - f(x) )2 ]

of A in computing f is approximately minimized among all algorithms A' 
of similar running time.

One has to work a bit to make sure that the above definition make sense.

As usual in average-case complexity, one computes (1) conditioned on x 
having length n, and  one ignores additive terms that go to zero 
sufficiently fast as n grows. Alternatively, one thinks of D as an 
ensemble of distributions D1,D2,...,Dn,... where Dn is a distribution 
over instances of "size" n (but not necessarily of bit-length n).

Then one has to decide what kind of additive terms going to zero in n 
can be ignored when we say that (1) is *approximately* minimized by A. A 
couple of reasonable options is to ignore *negligible* terms, where a 
function is negligible if it goes to zero faster than any 
inverse-polynomial functions, or to ignore terms smaller than epsilon, 
where epsilon is known to A, and A is allowed to run in time polynomial 
in 1/epsilon.

We will return to these issues in a moment.

Having given the definition of an optimal estimator for a certain 
complexity bound, if A is an optimal estimator for a function f under 
distribution D, the idea is that, given an x sampled from D, we will 
take A(x) to be the "expectation of the value f(x)." Decision problems 
are a special case, in which f(.) is a boolean function, the indicator 
function of membership in the language. Thus, if A is an optimal 
estimator for f, and f is the indicator function of a language L, we 
will take A(x) to be the "probability that x is in L".

The paper shows various ways in which this definition makes sense. If 
f(.) is computable within the complexity bound, then the optimal 
estimator is just an algorithm that computes the function, so there is 
no "uncertainty" and the expectation of f(x) is just the value of f(x) 
and the probability that x is in L is just 0 or 1 depending on whether 
the statement is false or true. At the other extreme, if f(.) is the 
indicator function of a decision problem that is very hard on average 
(meaning that no algorithm with the required running time is able to 
decide the problem with a noticeable advantage over a random guess), 
then an optimal estimator is to output 1/2 on all inputs, capturing the 
intuition that a computationally bounded party has "no information" 
whether a given x is in the language or not.

It's important to note that the second example above would not hold if 
the error was measured as
 E_{x ~ D} [   | A(x) - f(x) |  ]
because in such a case *every* algorithm would be an optimal estimator.

 The author proceeds to show that the definition satisfies several 
desirable properties: in many cases optimal estimators exist, and when 
they exist they have reasonable properties such as "linearity of 
expectation," a "union bound" and additivity for probabilities of 
disjoint events, it is possible to define "conditional probabilities" 
and, most importantly, any two optimal estimators must approximately 
agree, so that the definition is indeed sound.

 Although it does not have any immediate application, the definition 
introduced in this paper is appealing and well motivated, and it is 
likely that it will inspire further work and that applications may come 
later.

 Let me now return to the definitional issues. As mentioned above, to 
instantiate the general idea one has to specify which complexity 
resources are allowed for the algorithms among which we will optimize, 
and one has to define what approximation error is allowed between the 
optimal estimator and all other algorithms; furthermore one may want to 
think of the distribution of inputs as a collection of distributions, 
indexed by a "length" parameter, and one may also want to think of the 
approximation error as an additional parameter.

 This is an annoyance that is common to all work on average-case 
complexity, but the author doubles down with a bizarrely overwrought 
definitional edifice that obscures the elegance of the ideas in the 
paper in order to gain a generality that has no visible advantage in the 
paper. A sign that something has gone astray is that one can see a few 
long proofs in the paper in which only two or three steps in the middle 
are doing the actual work, and all the rest of the proof is boilerplate 
to unfold the definitions used in the assumptions and then fold back the 
result in the form of the definitions used in the conclusions.
\end{verbatim}
\color{red}
\begin{verbatim}
 My advice for the author would be to rewrite this work in the simplest 
setting (corresponding to "heuristic algorithms") without any new 
notation, and give the proofs of the main result just in this setting, 
which can probably be done in 10-15 pages.

 If this is not something that the author is interested in doing, I 
think that the submission is publishable in the current form. 
\end{verbatim}
\color[rgb]{0,0.501961,0}
\begin{verbatim}
I am currently not interested in this simplification.
\end{verbatim}
\color{red}
\begin{verbatim}
In the 
attached file I note some typo and misspelling. In particular note that 
"random" and "pseudorandom" are (in the context in which they are used) 
adjectives and the corresponding nouns are "randomness" and 
"pseudorandomness." 
\end{verbatim}
\color[rgb]{0,0.501961,0}
\begin{verbatim}
All typos were fixed.
\end{verbatim}
\color{red}
\begin{verbatim}
Also the introduction brings up notation that is 
undefined, in the context of defining certain notions: if the goal is to 
just give the spirit of definitions that will be formalized later, then 
at least give forward reference to where the undefined notation will be 
defined.
\end{verbatim}
\color[rgb]{0,0.501961,0}
\begin{verbatim}
Indeed, the goal is to just give the spirit of the definition. To
address this comment, I added forward references to the relevant
definitions.
\end{verbatim}
\color{black}

\signature{Vadim Kosoy}
\closing{Sincerely,}

\end{letter}

\end{document}


