%&latex
\documentclass[11pt]{article}

\usepackage[letterpaper,margin=1in]{geometry}
\usepackage{hyperref}
\usepackage{csquotes}
\usepackage{amsmath,amsthm}
\usepackage{xfrac}

\theoremstyle{definition}
\newtheorem{definition}{Definition}%[section]
\newtheorem{example}{Example}%[section]

\theoremstyle{plain}
\newtheorem{theorem}{Theorem}%[section]
\newtheorem{lemma}{Lemma}%[section]
\newtheorem{proposition}{Proposition}%[section]
\newtheorem{corollary}{Corollary}%[section]

\newcommand{\Comment}[1]{}

% operators that are separated from the operand by a space
\DeclareMathOperator{\Sgn}{sgn}
\DeclareMathOperator{\Supp}{supp}
\DeclareMathOperator{\Dom}{dom}

% autosize delimiters
\newcommand{\AP}[1]{\left(#1\right)}
\newcommand{\AB}[1]{\left[#1\right]}
\newcommand{\AC}[1]{\left\{#1\right\}}
\newcommand{\APM}[2]{\left(#1\;\middle\vert\;#2\right)}
\newcommand{\ABM}[2]{\left[#1\;\middle\vert\;#2\right]}
\newcommand{\ACM}[2]{\left\{#1\;\middle\vert\;#2\right\}}

% operators that require brackets
\newcommand{\Pa}[2]{\underset{#1}{\operatorname{Pr}}\AB{#2}}
\newcommand{\PP}[2]{\underset{\substack{#1 \\ #2}}{\operatorname{Pr}}}
\newcommand{\PPP}[3]{\underset{\substack{#1 \\ #2 \\ #3}}{\operatorname{Pr}}}
\newcommand{\E}[1]{\underset{#1}{\operatorname{E}}}
\newcommand{\Ea}[2]{\underset{#1}{\operatorname{E}}\AB{#2}}
\newcommand{\EE}[2]{\underset{\substack{#1 \\ #2}}{\operatorname{E}}}
\newcommand{\EEE}[3]{\underset{\substack{#1 \\ #2 \\ #3}}{\operatorname{E}}}
\newcommand{\I}[1]{\underset{#1}{\operatorname{I}}}
\newcommand{\Ia}[2]{\underset{#1}{\operatorname{I}}\AB{#2}}
\newcommand{\II}[2]{\underset{\substack{#1 \\ #2}}{\operatorname{I}}}
\newcommand{\III}[3]{\underset{\substack{#1 \\ #2 \\ #3}}{\operatorname{I}}}
\newcommand{\Var}{\operatorname{Var}}

% operators that require parentheses
\newcommand{\En}{\operatorname{H}}
\newcommand{\Ena}[1]{\operatorname{H}\AP{#1}}

\newcommand{\D}{\mathrm{d}}
\newcommand{\KL}[2]{\operatorname{D}_{\mathrm{KL}}(#1 \| #2)}
\newcommand{\Dtv}{\operatorname{d}_{\text{tv}}}
\newcommand{\Dtva}[1]{\operatorname{d}_{\text{tv}}\AP{#1}}

\newcommand{\Argmin}[1]{\underset{#1}{\operatorname{arg\,min}}\,}
\newcommand{\Argmax}[1]{\underset{#1}{\operatorname{arg\,max}}\,}

\newcommand{\Nats}{\mathbb{N}}
\newcommand{\Ints}{\mathbb{Z}}
\newcommand{\Rats}{\mathbb{Q}}
\newcommand{\Reals}{\mathbb{R}}
\newcommand{\Coms}{\mathbb{C}}

\newcommand{\Estr}{\boldsymbol{\lambda}}

\newcommand{\Lim}[1]{\lim_{#1 \rightarrow \infty}}
\newcommand{\LimInf}[1]{\liminf_{#1 \rightarrow \infty}}
\newcommand{\LimSup}[1]{\limsup_{#1 \rightarrow \infty}}

\newcommand{\Abs}[1]{\left\vert #1 \right\vert}
\newcommand{\Norm}[1]{\left\Vert #1 \right\Vert}
\newcommand{\Floor}[1]{\left\lfloor #1 \right\rfloor}
\newcommand{\Ceil}[1]{\left\lceil #1 \right\rceil}
\newcommand{\Chev}[1]{\left\langle #1 \right\rangle}
\newcommand{\Quote}[1]{\left\ulcorner #1 \right\urcorner}

\newcommand{\K}{\xrightarrow{\text{k}}}
\newcommand{\PF}{\xrightarrow{\circ}}

% Paper specific

\newcommand{\A}{\mathcal{A}}
\newcommand{\St}{\mathcal{S}}
\newcommand{\T}{\mathcal{T}}
\newcommand{\R}{\mathcal{R}}
\newcommand{\Ada}{\bar{\A}}
\newcommand{\Hy}{\mathcal{H}}
\newcommand{\UC}{\mathcal{U}}

\newcommand{\Ut}{\operatorname{U}}
\newcommand{\V}{\operatorname{V}}
\newcommand{\Q}{\operatorname{Q}}
\newcommand{\EU}{\operatorname{EU}}

\begin{document}

%+Title
\title{Learning to avoid traps with a little help}
\author{Vadim Kosoy \\ \href{mailto:vadim.kosoy@intelligence.org}{vadim.kosoy@intelligence.org}}
\date{}
\maketitle
%-Title

%+Abstract
\begin{abstract}
    Most known regret bounds for reinforcement learning are either episodic or assume an environment without traps. We derive a regret bound without making either assumption by allowing the algorithm to occasionally delegate an action to an external advisor. We thus arrive at a setting of active one-shot model-based reinforcement learning that we call DRL (delegative reinforcement learning.) The algorithm we construct in order to demonstrate the regret bound is a variant of Thompson sampling supplemented by a subroutine that decides which actions should be delegated. Importantly, the algorithm is not anytime, since the confidence threshold for acting without delegating must be adjusted according to the target time discount. We also demonstrate that this setting can handle situations in which the reward signal and the advisor become unreliable in particular environment states (assuming these states can be avoided.) Currently, our analysis is limited to Markov decision processes with a finite number of states. 
\end{abstract}
%-Abstract

%+Contents
%\tableofcontents
%-Contents

\section{Introduction}

A \emph{reinforcement learning agent} is a system that interacts with an unknown environment in a manner that is designed to maximize the expectation of a utility function that can be written as a sum rewards over time (sometimes weighted by a time-discount function.) A standard metric for evaluating the performance of such an agent is the \emph{regret}: the difference between the expected utility of the agent in a given environment, and the expected utility of an optimal policy for the same environment. This metric allows formalizing the notion of \enquote{the agent \emph{learns} the environment} by requiring that the regret has sublinear growth in the planning horizon (usually assuming the utility function is a finite, undiscounted, sum of rewards.) For example, if we consider stateless environments, reinforcement learning reduces to a multi-armed bandit for which algorithms with guaranteed sublinear regret bounds are well-known \cite{TBD}.

However, the desideratum of sublinear regret is impossible to achieve even for a finite class of environments without making further assumptions, and this is because of the possible presence of \enquote{traps}. A trap is a state which, once reached, forces a linear lower bound on regret. For example, see figure~\ref{TBD}. The agent starts at state $s_1$, and as long as it takes action $a$, it receives a reward of $1$. However, if it ever takes action $b$, it will reach state $s_2$ and remain there, receiving a reward of $0$ forever, whatever it does. Thus, $s_2$ is a trap. On the other hand, it is impossible to design an algorithm which guarantees never entering traps for an arbitrary environment. For example, the environment on figure~\ref{TBD} has the same structure except that taking action $a$ leads into the trap. Thus, if the transition matrix is not known a priori, no algorithm can learn the correct behavior, and every algorithm will have linear regret in at least one of the two environments.

There are two widespread approaches to deriving regret bounds which circumvent this problem. One is simply assuming that the environment contains no traps in some formal sense \cite{TBD}. The other is \enquote{episodic learning \cite{TBD}.} In episodic learning, the timeline is divided into intervals (\enquote{episodes}) and, regret is defined s.t. the contribution of each episode is the difference between following the given policy and following the given policy \emph{during previous episodes} but the optimal policy in the current episode. Such a metric consider entering a trap as a fatal event since in following episodes this even will be considered as \enquote{given.} That is, a policy that enters trap can still achieve sublinear regret in this sense. In fact, algorithms designed to achieve sublinear regret for sufficiently general classes of environments have the property that they eventually enter \emph{every trap they encounter} (such algorithms have a random exploration phase, like e.g. $\epsilon$-exploration in Q-learning.)

In terms of practical applications, it means that most known approaches to reinforcement learning that have theoretical performance guarantees either assume that no mistake is \enquote{fatal}, or that numerous \enquote{fatal} mistakes in the training process are acceptable. These assumptions are unacceptable in applications such as controlling a very expensive, breakable piece of machinery (e.g. spaceship) or performing a task that involves significant risk to human lives (e.g. surgery or rescue,) assuming that the algorithm \emph{cannot} be reliably trained in a simulation since the simulation doesn't reflect all the intricacies of the physical world.

This problem clearly cannot be overcome without using prior knowledge about the environment. In itself, prior knowledge is not such a strong assumption, since at least for any task that can be accomplished by a person, this prior knowledge is already available to us. The challenge is then transferring this knowledge to algorithm. This transfer can be accomplished either by manually transforming the knowledge into a formal mathematical specification, or by establishing a learning protocol that involves a human in the loop. Since human knowledge is often complex, difficult to formalise and partly intuitive, the latter option seems especially attractive.

These idea of using prior knowledge or human intervention to avoid traps has been explored by several authors (see Garcia and Fernandez \cite{Garcia2015} for a survey.) However, to the best of our knowledge, no previous author has established a regret bound in such a setting. In the present work, we derive such a regret bound, specifically for the setting that Clouse \cite{Clouse1997} called \enquote{ask for help} and we call \enquote{delegative reinforcement learning} (DRL), and specifically for a class of environments which consists of some finite number of Markov decision processes with a finite number of states.

In DRL, an agent interacts with an environment during an infinite sequence of \enquote{rounds}. On each round, the agent selects an action and the environment transits to a new state which is observed by the agent. The agent then receives a reward which depends on the state. There are two kinds of actions the agent can take: a \enquote{direct} action $a \in \A$ and the special delegation action $\bot$. If the agent takes action $\bot$, the \emph{advisor} takes some action $b \in \A$ which affects the environment in the same way as if it was taken directly. The agent then observes both $b$ and the new state of the environment. The utility function and regret are defined via geometric time discount with a constant $\gamma$.

The algorithm we construct in order to show the regret bound is a variant of Thompson sampling. Denoting $\alpha:=1-\gamma$, the timeline is divided into intervals of length $O\AP{\alpha^{-1/4}}$. At the start of each interval, the algorithm samples a hypothesis out of its current belief state, and starts carrying out an optimal policy for this hypothesis. On each round, it checks whether the desired action is known to be \enquote{safe} with high probability is a particular formal sense. If it is safe, the action is taken. If it isn't safe, delegation is performed. Moreover, the belief state evolves using all observations, but hypotheses whose probability falls below $O\AP{\alpha^{1/4}}$ are discarded altogether. We then show that (i) given relatively mild assumptions about the advisor (namely, that it only takes safe actions and it takes the optimal action with at least some small probability,) the regret is bounded by $O\AP{\alpha^{-3/4}}$ (in particular it is sublinear in $\alpha^{-1}$) and (ii) the expected number of delegations is bounded by $O\AP{\alpha^{-1/4}}$. Here, we only gave the dependence on $\alpha$, but the expressions we obtain are more detailed and reflect the dependence on the number of hypothesis, the \emph{mixing time} of the hypotheses and the minimal probability with which the advisor takes an optimal action.

The proof is based on the conjunction of four observations. The first is that, with high probability, our algorithm only takes safe actions (since it delegates whenever it is uncertain.) The second is that the number of delegations cannot be large for the \emph{information theoretic} reason that, delegation is only done under conditions in which it is expected to yield a certain minimal \emph{information gain}, whereas, since the number of hypotheses is some finite $N$, the total amount of information is also finite and equal to $\ln N$. The third is that, assuming all actions are safe, regret can be approximated by \enquote{episodic regret} (to show this we use the Taylor expansion of the \emph{value} of states in the parameter $\alpha$.) The fourth is that Thompson sampling (without delegation) satisfies an upper bound on episode regret in terms of the information gain due to observing the reward during this episode (the intuition is, if observing the reward yields no information, then the optimal policy for any hypothesis is optimal for all other hypotheses, so there is no regret,) and the sum of information gains over episodes is again bounded by $\ln N$. Here, both information theoretic steps require the assumption that no hypothesis is assigned a very small but positive probability, which necessitates discarding such hypothesis. This \enquote{distortion} of the belief state has little effect because, obviously, hypotheses with very small probability are likely to be false.

We also consider situations in which there are states of the environment (that we call \enquote{corrupted}) for which the reward signal and/or the advisor become unreliable (i.e. the observed reward in these states might be different from the \enquote{true} reward w.r.t. which regret is defined, and the advisor in these states might fail to satisfy the assumptions we otherwise require from it.) For example, we might imagine a robot that, through its own actions, damages its own input channels or even provides \enquote{deliberately} misleading information to the human operator. Indeed, it has been argued~\cite{TBD} that reinforcement learning agents are incentivized to sabotage themselves in this way (for example \enquote{wirehead} i.e. tamper with itself in order to artificially set the reward to maximum) and therefore sufficiently powerful algorithms (beyond the current state of the art) are almost guaranteed to do so. We show that, assuming corrupted states can be avoided without sacrificing utility, and that the advisor in uncorrupted states never acts so as to enter a corrupted state, DRL can be used to achieve essentially the same regret bound as before (and in particular learn to avoid corruption.) Thus DRL not only combats traps in the external environment but also adversarial incentives inside the agent itself.

The structure of the paper is as follows... TBD

\section{Results}

We start by reviewing some basic definitions and properties of Markov decision processes. 

First, some notation. Given measurable spaces $X$ and $Y$, the notation $K: X \K Y$ means that $K$ is a Markov kernel from $X$ to $Y$. Given $x \in X$, $K(x)$ is the corresponding probability measure on $Y$. Given $A \subseteq Y$ measurable, $K(A \mid x) := K(x)(A)$. Given $y \in Y$, $K(y \mid x):=K\APM{\{y\}}{x}$. Given $J: Y \K Z$, $JK: X \K Z$ is the composition of $J$ and $K$, and when $Y = X$, $K^n$ is the $n$-th composition power.

\begin{samepage}
\begin{definition}

A (finite) \emph{Markov decision process} (MDP) is a tuple

$$M:=\AP{\St_M,\ \A_M,\ \T_M: \St_M \times \A_M \K \St_M,\ \R_M: \St_M \rightarrow [0,1]}$$

Here, $\St_M$ is a finite set (the set of states,) $\A_M$ is a finite set (the section of actions,) $\T_M$ is the transition kernel and $\R_M$ is the reward function\footnote{Sometimes the reward is assumed to depend on the action as well, or on the action and the next state, but these formalisms are easily seen to be equivalent via redefinitions of the state set.}.

Given $\pi: \St_M \rightarrow \A_M$, we define $\T_{M\pi}: \St_M \K \St_M$ by

\begin{equation}
\T_{M\pi}(t \mid s) := \T_M\APM{t}{s,\pi(s)}
\end{equation}

We define $\V_M : \St_M \times [0,1) \rightarrow [0,1]$ and $\Q_M: \St_M \times \A_M \times [0,1) \rightarrow [0,1]$ by

\begin{equation}
\V_M(s,\gamma):=(1-\gamma) \max_{\pi: \St_M \rightarrow \A_M} \sum_{n=0}^\infty \gamma^n \Ea{\T_{M\pi}^n(s)}{\R_M}
\end{equation}

\begin{equation}
\Q_M(s,a,\gamma):=(1-\gamma)\R_M(s)+\gamma\Ea{t \sim \T_{M\pi}(s)}{\V_M(t,\gamma)}
\end{equation}

...

\end{definition}
\end{samepage}

TBD

\section{Proof Outline}

TBD

\section{Proof Details}

TBD

\section{Discussion}

TBD

\section*{Acknowledgements}

TBD

\bibliographystyle{unsrt}
\bibliography{DRL}

\end{document}


