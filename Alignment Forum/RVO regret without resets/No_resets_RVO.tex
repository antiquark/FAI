%&latex
\documentclass[a4paper]{article}

\usepackage[a4paper,margin=1in]{geometry}
\usepackage[affil-it]{authblk}
\usepackage{cite}
\usepackage[unicode]{hyperref}
\usepackage[utf8]{inputenc}
\usepackage[english]{babel}
\usepackage{csquotes}
\usepackage{amsmath,amssymb,amsthm}
\usepackage{enumerate}
\usepackage{commath}

\newcommand{\Co}[1]{}
% operators that are separated from the operand by a space
\DeclareMathOperator{\Sgn}{sgn}
\DeclareMathOperator{\Supp}{supp}
\DeclareMathOperator{\Dom}{dom}
\DeclareMathOperator{\Sp}{span}
% autosize delimiters
\newcommand{\AP}[1]{\left(#1\right)}
\newcommand{\AB}[1]{\left[#1\right]}
\newcommand{\AC}[1]{\left\{#1\right\}}
\newcommand{\APM}[2]{\left(#1\;\middle\vert\;#2\right)}
\newcommand{\ABM}[2]{\left[#1\;\middle\vert\;#2\right]}
\newcommand{\ACM}[2]{\left\{#1\;\middle\vert\;#2\right\}}
% probability theory
\newcommand{\Pa}[2]{\underset{#1}{\operatorname{Pr}}\AB{#2}}
\newcommand{\CP}[3]{\underset{#1}{\operatorname{Pr}}\ABM{#2}{#3}}
\newcommand{\PP}[2]{\underset{\substack{#1 \\ #2}}{\operatorname{Pr}}}
\newcommand{\PPP}[3]{\underset{\substack{#1 \\ #2 \\ #3}}{\operatorname{Pr}}}
\newcommand{\E}[1]{\underset{#1}{\operatorname{E}}}
\newcommand{\Ea}[2]{\underset{#1}{\operatorname{E}}\AB{#2}}
\newcommand{\CE}[3]{\underset{#1}{\operatorname{E}}\ABM{#2}{#3}}
\newcommand{\EE}[2]{\underset{\substack{#1 \\ #2}}{\operatorname{E}}}
\newcommand{\EEE}[3]{\underset{\substack{#1 \\ #2 \\ #3}}{\operatorname{E}}}
\newcommand{\Var}{\operatorname{Var}}
\newcommand{\I}[1]{\underset{#1}{\operatorname{I}}}
\newcommand{\CI}[3]{\underset{#1}{\operatorname{I}}\ABM{#2}{#3}}
\newcommand{\Ia}[2]{\underset{#1}{\operatorname{I}}\AB{#2}}
\newcommand{\II}[2]{\underset{\substack{#1 \\ #2}}{\operatorname{I}}}
\newcommand{\III}[3]{\underset{\substack{#1 \\ #2 \\ #3}}{\operatorname{I}}}
\newcommand{\KL}[2]{\operatorname{D}_{\mathrm{KL}}\AP{#1\middle\vert\middle\vert#2}}
\newcommand{\RD}[3]{\operatorname{D}_{#1}\AP{#2\middle\vert\middle\vert#3}}
\newcommand{\Dtv}{\operatorname{d}_{\text{tv}}}
\newcommand{\Dtva}[1]{\operatorname{d}_{\text{tv}}\AP{#1}}
\newcommand{\En}{\operatorname{H}}
\newcommand{\Ena}[1]{\operatorname{H}\AP{#1}}
% power set
\newcommand{\PS}[1]{\mathcal{P}\AP{#1}}
% differential
\newcommand{\D}{\mathrm{d}}
% arg
\newcommand{\Argmin}[1]{\underset{#1}{\operatorname{arg\,min}}\,}
\newcommand{\Argmax}[1]{\underset{#1}{\operatorname{arg\,max}}\,}
% numbers
\newcommand{\Nats}{\mathbb{N}}
\newcommand{\Ints}{\mathbb{Z}}
\newcommand{\Rats}{\mathbb{Q}}
\newcommand{\Reals}{\mathbb{R}}
\newcommand{\Coms}{\mathbb{C}}
% linear algebra
\newcommand{\PD}{\mathrm{PD}}
\newcommand{\PSD}{\mathrm{PSD}}
% empty string
\newcommand{\Estr}{\boldsymbol{\lambda}}
% limits
\newcommand{\Lim}[1]{\lim_{#1 \rightarrow \infty}}
\newcommand{\LimInf}[1]{\liminf_{#1 \rightarrow \infty}}
\newcommand{\LimSup}[1]{\limsup_{#1 \rightarrow \infty}}
% more delimiters
\newcommand{\Abs}[1]{\left\vert #1 \right\vert}
\newcommand{\Norm}[1]{\left\Vert #1 \right\Vert}
\newcommand{\Floor}[1]{\left\lfloor #1 \right\rfloor}
\newcommand{\Ceil}[1]{\left\lceil #1 \right\rceil}
\newcommand{\Chev}[1]{\left\langle #1 \right\rangle}
\newcommand{\Quote}[1]{\left\ulcorner #1 \right\urcorner}
% arrows
\newcommand{\K}{\xrightarrow{\mathrm{k}}}
\newcommand{\PF}{\xrightarrow{\circ}}
% Paper specific
\newcommand{\B}{B}
\newcommand{\X}{\mathcal{X}}
\newcommand{\Y}{\mathcal{Y}}
\newcommand{\F}{\mathcal{F}}
\newcommand{\St}{\mathcal{S}}
\newcommand{\A}{\mathcal{A}}
\newcommand{\R}{\mathcal{R}}
\newcommand{\T}{\mathcal{T}}
\newcommand{\El}{\mathrm{L}}
\newcommand{\Hy}{\mathcal{H}}
\DeclareMathOperator{\RVO}{\dim_{RVO}}
\DeclareMathOperator{\MB}{\dim_{MB}}
\DeclareMathOperator{\LD}{\dim_{loc}}
\newcommand{\DRVO}{D_{\mathrm{RVO}}}
\newcommand{\DMB}{D_{\mathrm{MB}}}
\newcommand{\DL}{D_{\mathrm{loc}}}
\newcommand{\N}{\mathrm{N}}
\newcommand{\V}{\mathrm{V}}
\newcommand{\Q}{\mathrm{Q}}
\newcommand{\EU}{\mathrm{EU}}
\newcommand{\Reg}{\mathrm{R}}
\newcommand{\PSR}{\text{PS}}
\newcommand{\LS}{\mathrm{LS}}
\newcommand{\CS}{\mathrm{CS}}
\newcommand{\W}{\mathrm{W}}
\newcommand{\AT}{\mathrm{A}}
\newcommand{\THy}{\mathrm{H}_*}
\newcommand{\SHy}{\mathrm{H}}
\newcommand{\De}{\Delta}
\newcommand{\CSE}{G}

\begin{document}

\textbf{TLDR:}\Co{b} I derive a variant of the RL regret bound by [Osband and Van Roy (2014)](https://arxiv.org/abs/1406.1853), that applies to learning without resets of environments without traps. The advantage of this regret bound over those known in the literature, is that it scales with certain learning-theoretic dimensions rather than number of states and actions. My goal is building on this result to derive this type of regret bound for [DRL](https://agentfoundations.org/item?id=1656), and, later on, other settings interesting from an AI\ alignment perspective.

***

[Previously](https://www.alignmentforum.org/posts/zTf946PQwN2AN3X3Y/entropic-regret-i-deterministic-mdps) I derived a regret bound for \textit{deterministic}\Co{i} environments that scales with prior entropy and "prediction dimension". That bound behaves as $O\AP{\sqrt{1-\gamma}}$ in the episodic setting but only as $O\AP{\sqrt[3]{1-\gamma}}$ in the setting without resets. Moreover, my attempts to generalize the result to stochastic environments led to bounds that are \textit{even weaker}\Co{i} (have a lower exponent). Therefore, I decided to put that line of attack on hold, and use Osband and Van Roy's technique instead, leading to an $O\AP{\sqrt{1-\gamma}}$ bound in the stochastic setting without resets. The main disadvantage is, this bound doesn't scale down with the entropy of the prior, but this does not seem as important as dependence on $\gamma$.

\begin{Huge}Results\end{Huge}

[Russo and Van Roy](https://papers.nips.cc/paper/4909-eluder-dimension-and-the-sample-complexity-of-optimistic-exploration) introduced the concept of "eluder dimension"  for the benefit of the multi-armed bandit setting, and Osband and Van Roy extended it to a form suitable for studying reinforcement learning. We will consider the following, slightly modified version of their definition.

Given a real vector space $\Y$, we will use $\PSD(\Y)$ to denote the set of positive semidefinite bilinear forms on $\Y$ and $\PD(\Y)$ to denote the set of positive definite bilinear forms on $\Y$. Given a bilinear form $\B:\Y\times\Y\rightarrow\Reals$, we will slightly abuse notation by also regarding it as a linear functional $\B:\Y\otimes\Y\rightarrow\Reals$. Thereby, we have $\B(v \otimes w) = \B(v, w)$ and $\B v^{\otimes 2}=\B (v,v)$. Also, if $\Y$ is finite-dimensional and $\B$ is non-degenerate, we will denote $\B ^{-1}:\Y^*\times\Y^*\rightarrow\Reals$ the unique bilinear form which satisfies

$$\forall v\in\Y,\alpha\in\Y^*:\AP{\forall\beta\in\Y^*:\B ^{-1}(\alpha,\beta)=\beta v}\iff\AP{\forall w\in\Y:\B (v,w)=\alpha w}$$ 

\textbf{Definition 1}\Co{b}

\textit{Consider a set $\X$, a real vector space $\Y$, some $\F\subseteq\AC{\X\rightarrow\Y}$ and a family $\AC{\B _x\in\PSD(\Y)}_{x\in\X}$. Consider also $n\in\Nats$, a sequence $\AC{x_k\in\X}_{k\in[n]}$ and $x^*\in\X$. $x^*$ is said to be} $\AP{\F,\B }$-dependant on $\AC{x_k}$ \textit{when, for any $f,\tilde{f}\in\F$}\Co{i}

$$\sum_{k=0}^{n-1}\B _{x_k}\AP{f\AP{x_k}-\tilde{f}\AP{x_k}}^{\otimes 2}\leq1\implies \B _{x^*}\AP{f\AP{x^*}-\tilde{f}\AP{x^*}}^{\otimes 2}\leq1$$

\textit{Otherwise, $x^*$ is said to be}\Co{i} $\AP{\F,\B }$-independent of $\AC{x_k}$.

\textbf{Definition 2}\Co{b}

\textit{Consider a set $\X$, a real vector space $\Y$ and some $\F\subseteq\AC{\X\rightarrow\Y}$. The}\Co{i} Russo-Van Roy-Osband dimension (RVO dimension) $\RVO{\F}$ \textit{is the supremum of the set of $n\in\Nats$ for which there is $\AC{x_k\in\X}_{k\in[n]}$ and $\B $ s.t. for all $m\in[n]$, $x_m$ is $\AP{\F,\B }$-independent of $\AC{x_k\in\X}_{k\in[m]}$.}\Co{i}

We have the following basic bounds on RVO dimension.

\textbf{Proposition 1}\Co{b}

\textit{Consider a set $\X$, a real vector space $\Y$ and some $\F\subseteq\AC{\X\rightarrow\Y}$. Then}\Co{i}

$$\RVO{\F}\leq\Abs{\X}$$

\textbf{Proposition 2}\Co{b}

\textit{Consider a set $\X$, a real vector space $\Y$ and some $\F\subseteq\AC{\X\rightarrow\Y}$. Then}\Co{i}

$$\RVO{\F}\leq\frac{\Abs{\F}\AP{\Abs{\F}-1}}{2}$$

Another concept we need to formulate the regret bound is the Minkowski–Bouligand dimension.

\textbf{Definition 3}\Co{b}

\textit{Consider a set $\X$, a real vector space $\Y$, some $\F\subseteq\AC{\X\rightarrow\Y}$ and a family $\AC{\B _x\in\PSD(\Y)}_{x\in\X}$. A set $A\subseteq\F$ is said to be a}\Co{i} $\B $-covering of $\F$ \textit{when}\Co{i}

$$\forall f\in\F\exists\tilde{f}\in A: \sup_{x\in\X}{\B _x\AP{f(x)-\tilde{f}(x)}^{\otimes 2}}<1$$

\textbf{Definition 4}\Co{b}

\textit{Consider a set $\X$, a real vector space $\Y$, some $\F\subseteq\AC{\X\rightarrow\Y}$ and a family $\AC{\B _x\in\PSD(\Y)}_{x\in\X}$. The}\Co{i} covering number $\N(\F,\B )$ \textit{is the infimum of the set of $n\in\Nats$ for which there is a $\B $-covering of $\F$ of size $n$.}\Co{i}

\textbf{Definition 5}\Co{b}

\textit{Consider a finite set $\X$, a finite-dimensional real vector product space $\Y$ and some $\F\subseteq\AC{\X\rightarrow\Y}$. Fix any $\AC{\B \in\PD(\Y)}_{x\in\X}$.  The}\Co{i} Minkowski–Bouligand dimension (MB dimension) of $\F$ \textit{is defined by}\Co{i}

$$\MB{\F}:=\limsup_{\epsilon \rightarrow 0}{\frac{\ln{\N\AP{\F,\epsilon^{-2} \B }}}{\ln\frac{1}{\epsilon}}}$$

It is easy to see the above is indeed well-defined, i.e. doesn't depend on the choice of $\B $. This is because given any $\B ,\tilde{\B }$, there are constants $c_1,c_2\in\Reals^+$ s.t. for all $\F$ and $\epsilon$

$$c_1 \N\AP{\F,\epsilon^{-2} \B } \leq \N\AP{\F,\epsilon^{-2} \tilde{\B }} \leq c_2\N\AP{\F,\epsilon^{-2} \B }$$

For finite $\F$ and $\epsilon\ll1$, it's obvious that $\N\AP{\F,\epsilon^{-2}\B }=\Abs{\F}$, and in particular $\MB\F=0$. It is also possible to show that, for any $\F$, $\MB{\F}\leq\Abs{\X}\dim{\Y}$.

Note that, in general, MB dimension is fractional.

We will need yet another (but rather simple) notion of "dimension".

\textbf{Definition 6}\Co{b}

\textit{Consider a set $\X$, a vector space $\Y$ and some $\F\subseteq\AC{\X\rightarrow\Y}$. The}\Co{i} local dimension of $\F$ \textit{is defined by}\Co{i}

$$\LD{\F}:=\max_{x\in\X}{\dim\Sp\ACM{f(x)}{f\in\F}}$$

Obviously $\LD{\F}\leq\Abs{\F}$ and $\LD{\F}\leq\dim{\Y}$.

Consider finite non-empty sets $\St$ (states) and $\A$ (actions). Observe that $\Delta\St$ can be regarded as a subset of the vector space $\Reals^\St$. This allows us to speak of the RVO, MB and local dimensions of a hypothesis class of transition kernels $\Hy\subseteq\AC{\St\times\A\K\St}$ (that is, in this case $\X=\St\times\A$ and $\Y=\Reals^\St$). 

Finally, there is a certain regularity condition in our main theorem that depends on the following parameters of MDPs.

\textbf{Definition 7}\Co{b}

\textit{Consider finite non-empty sets $\St$ and $\A$, $\R:\St\rightarrow[0,1]$ (reward function) and $\T:\St\times\A\K\St$ (transition kernel). The value function $\V_{\T\R}(s,\gamma)$ is always continuous and piecewise differentiable w.r.t. $\gamma$. Let $\partial_+\V_{\T\R}(s,\gamma)$ and $\partial_-\V_{\T\R}(s,\gamma)$ denote its right and left derivative respectively. We denote}\Co{i}

$$\partial_{\max}\V_{\T\R}(s,\gamma):=\sup_{x\in(\gamma,1)}{\partial_+\V_{\T\R}(s,x)}$$

$$\partial_{\min}\V_{\T\R}(s,\gamma):=\inf_{x\in(\gamma,1)}{\partial_-\V_{\T\R}(s,x)}$$

Note that, since $\V_{\T\R}(s,\gamma)$ is the maximum of a finite set of differentiable functions (the value functions corresponding to the deterministic stationary policies $\St\rightarrow\A$), we have $\partial_+\V_{\T\R}(s,\gamma)\geq\partial_-\V_{\T\R}(s,\gamma)$.

We can now formulate the regret bound.

\textbf{Theorem 1}\Co{b}

\textit{There is some $C\in\Reals^+$ s.t. the following holds.}\Co{i}

\textit{Consider any finite non-empty sets $\St$ and $\A$, $\R:\St\rightarrow[0,1]$, closed set $\Hy\subseteq\AC{\St\times\A\K\St}$ and Borel probability measure $\zeta$ on $\Hy$ (prior). Assume that for any $\T\in\Hy$ and $s,s'\in\St$, $\V_{\T\R}^0(s)=\V_{\T\R}^0\AP{s'}.$}\Co{i} [i.e. there are no traps. This is a stronger condition than the condition $\A^0_{\T\R}(s) = \A$ we used before: not only that long-term value cannot be lost \textit{in expectation}\Co{i}, it cannot be lost at all.] \textit{We define the expected bias span $\tau_{\zeta\R}$ by}\Co{i}

$$\tau_{\zeta\R}:=\Ea{\T\sim\zeta}{\max_{s\in\St}\V^1_{\T\R}(s)-\min_{s\in\St}\V^1_{\T\R}(s)}$$

\textit{Assume that for $1-\gamma\ll1$}\Co{i}

$$\Ea{\T\sim\zeta}{\max_{s\in\St}\partial_{\max}\V_{\T\R}(s,\gamma)-\min_{s\in\St}\partial_{\min}\V_{\T\R}(s,\gamma)}<\infty$$

[When this assumption is true, the left hand side converges to $\tau_{\zeta\R}$ with $\gamma\rightarrow1$.]

\textit{Denote $\DL:=\LD{\Hy}$, $\DMB:=\MB{\Hy}$ and $\DRVO:=\RVO{\Hy}$. Then, there is a family of policies $\AC{\pi^\dagger_\gamma:\St^*\times\St\K\A}_{\gamma\in(0,1)}$ s.t.}\Co{i}

$$\limsup_{\gamma \rightarrow 1}\frac{\Ea{\T\sim\zeta}{\EU^*_{\T\R}(\gamma)-\EU^{\pi^\dagger_\gamma}_{\T\R}(\gamma)}}{\tau_{\zeta\R}\DL\sqrt{\ln{\frac{1}{1-\gamma}}\AP{\DMB+1}\DRVO(1-\gamma)}}\leq C$$

Here (like in previous essays), $\V^1_{\T\R}(s)$ is the first derivative of the value function at $\gamma=1$ for transition kernel $\T$, reward function $\R$ and state $s$; $\EU^{\pi}_{\T\R}(\gamma)$ is the expected utility for policy $\pi$ and geometric time discount parameter $\gamma$; $\EU^*_{\T\R}(\gamma)$ is the maximal expected utility over policies. The expression in the numerator is, thereby, the Bayesian regret.

A few directions for improving on this result:

* It is not hard to see from the proof that it is also possible to write down a concrete bound for fixed $\gamma$ (rather than considering the $\gamma\rightarrow 1$ limit), but its form is somewhat convoluted.

* It is probably possible to get an *anytime* policy with this form of regret, using PSRL with \textit{dynamic}\Co{i} episode duration.

* It is interesting to try to make do with the weaker no-traps condition $\A^0_{\T\R}(s) = \A$, especially since a stronger no-traps conditions would translate to a stronger condition on the advisor in DRL.

* It is interesting to study RVO dimension in more detail. For example, I'm not even sure whether Proposition 2 is the best possible bound in terms of $\Abs{\F}$ or it's e.g. possible to get a linear bound.

* Like I said in the start, I was unable to derive a satisfactory "entropic" regret bound using RVO dimension. However, at the time I did not try to use local dimension: its significance only became apparent to me when working on the current approach.

* It seems tempting to generalize local dimension by allowing the values of $f(x)$ to lie on some \textit{nonlinear} manifold of given dimension for any given $x$. This approach, if workable, might require a substantially more difficult proof.

* The "cellular decision processes" discussed [previously](https://www.alignmentforum.org/posts/zTf946PQwN2AN3X3Y/entropic-regret-i-deterministic-mdps) in "Proposition 3" have exponentially high local dimension, meaning that this regret bound is ineffective for them. We can consider a variant in which, on every time step, only one cell or a very small number of cells (possibly chosen randomly) change. This would have low local dimension. One way to interpret it is, as a continuous time process in which each cell has a certain *rate* of changing its state.

\begin{Huge}Proofs\end{Huge}

\textbf{Proof of Proposition 1}\Co{b}

TBD \textbf{Q.E.D.}\Co{b}

\textbf{Proof of Proposition 2}\Co{b}

TBD \textbf{Q.E.D.}\Co{b} % graph must be a tree

\textbf{Proposition A.N1}\Co{b}

In the setting of Theorem 1, fix $T\in\Nats^+$ and let $\pi_{\zeta\R T}^{\PSR}: \St^*\times\St\K\A$ be the policy implemented by a PSRL algorithm with prior $\zeta$, reward function $\R$ and episode length $T$. Let $(\Omega,P)$ be a probability space governing both the uncertainty about the true hypothesis, the stochastic behavior of the environment and the random sampling inside the algorithm (see the proof of "Lemma 1" in [this](https://www.alignmentforum.org/posts/zTf946PQwN2AN3X3Y/entropic-regret-i-deterministic-mdps) previous essay or the proof of "Theorem 1" in [another](https://agentfoundations.org/item?id=1739) previous essay). Furthermore, let $\THy:\Omega\rightarrow\Hy$ be a random variable representing the true hypothesis, $\AC{\SHy_n:\Omega\rightarrow\Hy}_{n\in\Nats}$ be the random variables s.t. $\SHy_n$ represents the hypothesis sampled at time $n$ (i.e. during episode number $\Floor{n/T}$), $\AC{\Theta_n:\Omega\rightarrow\St}_{n\in\Nats}$ be random variables s.t. $\Theta_n$ represents the state at time $n$ and $\AC{\AT_n:\Omega\rightarrow\A}_{n\in\Nats}$ be random variables s.t. $\AT_n$ represents the action taken at time $n$. Then

$$\Ea{\T\sim\zeta}{\EU^*_{\T\R}(\gamma)-\EU^{\pi_{\zeta\R T}^{\PSR}}_{\T\R}(\gamma)}=\sum_{n=0}^\infty\gamma^{n}\Ea{}{\Ea{s\sim \SHy_n\AP{\Theta_n,\AT_n}}{\V_{\SHy_n\R}(s,\gamma)}-\Ea{s\sim \THy\AP{\Theta_n,\AT_n}}{\V_{\SHy_n\R}(s,\gamma)}}$$

\textbf{Proof of Proposition A.N1}\Co{b}

% Proof uses Proposition B.1 from https://agentfoundations.org/item?id=1723

TBD \textbf{Q.E.D.}\Co{b}

\textbf{Proposition A.N6}\Co{b}

\textit{Consider a real finite-dimensional normed vector space $V$ and a linear subspace $W\subseteq V$. Then, there exists a $\B \in\PSD(V)$ s.t.}\Co{i}

1. \textit{For any $v\in V$, $\B v^{\otimes 2}\leq\Norm{v}^2$}\Co{i}

2. \textit{For any $w\in W$, $\AP{\dim{W}}^2 \B w^{\otimes 2}\geq \Norm{w}^2$}\Co{i}

\textbf{Proof of Proposition A.N6}\Co{b}

% Use the Kadec-Snobar (can also be cited: Koenig-Jaegermann) projection constant bound and John's ellipsoid 
TBD \textbf{Q.E.D.}\Co{b}

\textbf{Proposition A.N4}\Co{b}

\textit{Consider a finite-dimensional real vector space $\Y$, some $\B \in\PD(\Y)$ and a Borel probability measure $\mu\in\Delta\Y$ s.t. $\Pa{y\sim\mu}{\B y^{\otimes 2} \leq 1} = 1$. Let $y_0:=\Ea{y\sim\mu}{y}$ and $\sigma:=\sqrt{e}$. Then, $\mu$ is $\sigma$-sub-Gaussian w.r.t. $\B $, i.e., for any $\alpha\in\Y^*$}\Co{i}

$$\Ea{y\sim\mu}{\exp\AP{\alpha\AP{y-y_0}}} \leq \exp\AP{\frac{\sigma^2\B ^{-1}\alpha^{\otimes 2}}{2}}$$

\textbf{Proof of Proposition A.N4}\Co{b}

TBD \textbf{Q.E.D.}\Co{b}

\textbf{Definition A.N1}\Co{b}

\textit{Consider a set $\X$, a finite-dimensional real vector space $\Y$, some $\F\subseteq\AC{\X\rightarrow\Y}$ and a family $\AC{\B _x\in\PSD(\Y)}_{x\in\X}$. Assume $\F$ is compact w.r.t. the product topology on $\X\rightarrow\Y\cong\prod_{x\in\X}\Y$. Consider also some $n\in\Nats$, $\bold{x}\in\X^n$, $\bold{y}\in\Y^n$ and $r\in\Reals^+$. We then use the notation}\Co{i}

$$\LS^{\F}[\bold{xy},\B]:=\Argmin{f\in\F}{\sum_{m=0}^{n-1}\B _{\bold{x}_m}\AP{f\AP{\bold{x}_m}-\bold{y} }^{\otimes2}}$$

$$\CS^{\F}[\bold{xy},B]:=\ACM{f\in\F}{\sum_{m=0}^{n-1}\B _{\bold{x}_m}\AP{f\AP{\bold{x}_m}-\LS_\B ^\F[\bold{xy}]\AP{\bold{x}_m}}^{\otimes2}\leq 1}$$

I chose the notation $\LS$ as an abbreviation of "least squares" and $\CS$ as an abbreviation of "confidence set". Note that $\LS$ is somewhat ambiguous (and therefore, so is $\CS$) since there might be multiple minima, but this will not be important in the following (i.e. an arbitrary minimum can be chosen).

\textbf{Proposition A.N5}\Co{b}

\textit{There is some $C_{\mathrm{A.N5}}\in\Reals^+$ s.t. the following holds.}\Co{i}

\textit{Consider finite sets $\X,\St$, some $\F\subseteq\AC{\X\K\St}$ and a family $\AC{\B _x\in\PSD\AP{\Reals^\St}}_{x\in\X}$. Let $\AC{\mathfrak{H}_n\subseteq\PS{\X^\omega\times\St^\omega}}_{n\in\Nats}$ be the natural filtration, i.e.}\Co{i}

$$\mathfrak{H}_n:=\ACM{A'\subseteq\X^\omega\times\St^\omega}{A'=\ACM{\bold{xs}}{\bold{xs}_{:n}\in A},\ A\subseteq\X^n\times\St^n}$$

\textit{Consider also $f^*\in\F$ and $\mu\in\Delta\AP{\X^\omega\times\St^\omega}$ s.t. for any $n\in\Nats$, $x\in\X$, and $s\in\St$}\Co{i}

$$\CP{\bold{xs}\sim\mu}{\bold{s}_n=s}{\bold{x}_n=x,\ \mathfrak{H}_n} = f^*(s\mid x)$$

\textit{Fix $\epsilon\in\Reals^+$, $\delta\in(0,1)$. Denote}\Co{i}

$$\beta(t):=C_{\mathrm{A.N5}}\AP{\ln{\frac{\N(\F,\epsilon^{-2}\B )}{\delta}}+\epsilon t\ln{\frac{et}{\delta}}}$$

\textit{Then,}\Co{i}

$$\Pa{\bold{xs}\sim\mu}{f^*\not\in\bigcap_{n=0}^\infty\CS^\F\AB{\bold{xs}_{:n},\frac{\B }{\beta(n+1)}}} \leq \delta$$

\textit{Here, $\N$ and $\CS$ are defined by thinking of $\St$ and $\Delta\St$ as subsets of $\Y=\Reals^\St$.}\Co{i}

\textbf{Proof of Proposition A.N5}\Co{b}

% A.N4 + B.N2
TBD \textbf{Q.E.D.}\Co{b}

\Co{b}

\textbf{Definition A.N2}\Co{b}

\textit{Consider a set $\X$, some $x\in\X$, a real vector space $\Y$, some $\F\subseteq\AC{\X\rightarrow\Y}$ and a family $\AC{\B _x\in\PSD(\Y)}_{x\in\X}$. The}\Co{i} $\B $-width of $\F$ at $x$ \textit{is defined by}\Co{i}

$$\W^\F(x,B):=\sup_{f,\tilde{f}\in\F}\sqrt{\B _x\AP{f(x)-\tilde{f}(x)}^{\otimes2}}$$  

\textbf{Proposition A.N2}\Co{b}

\textit{There is some $C_{\mathrm{A.N2}}\in\Reals^+$ s.t. the following holds.}\Co{i}

\textit{Consider a set $\X$, a real vector space $\Y$, some $\F\subseteq\AC{\X\rightarrow\Y}$ and a family $\AC{\B _x\in\PSD(\Y)}_{x\in\X}$. Consider also some $\bold{x}\in\X^\omega$, $\bold{y}\in\Y^\omega$, $T\in\Nats^+$, $\gamma\in(0,1)$, $\theta\in\Reals^+$, $\eta_0,\eta_1\in\Reals^+$ and $\delta\in(0,1)$. Denote}\Co{i}

$$\beta(t):=\eta_0 + \eta_1t\ln{\frac{et}{\delta}}$$

For any $n\in\Nats$, define $\F_n$ by

$$\F_n:=\CS^\F\AB{\bold{xy}_{:n},\frac{\B }{\beta(n+1)}}$$

Then,

$$\sum_{l=0}^\infty\sum_{m=0}^{T-1}{\gamma^{lT+m}[[\W^{F_{lT}}\AP{\bold{x}_{lT+m},B}>\theta]]} \leq C_{\mathrm{A.N2}}\RVO{\F}\cdot\AP{\theta^{-2}\beta\AP{\frac{1}{1-\gamma}}+T}\ln{\frac{1}{\theta}}$$

\textbf{Proof of Proposition A.N2}\Co{b}

TBD \textbf{Q.E.D.}\Co{b}

\textbf{Proposition A.N3}\Co{b}

\textit{There is some $C_{\mathrm{A.N3}}\in\Reals^+$ s.t. the following holds.}\Co{i}

\textit{Consider a set $\X$, a real vector space $\Y$, some $\F\subseteq\AC{\X\rightarrow\Y}$ and a family $\AC{\B _x\in\PSD(\Y)}_{x\in\X}$. Assume that for any $x\in\X$ and $f\in\F$, $\B _x{f(x)}^{\otimes2}\leq 1$. Consider also some $\bold{x}\in\X^\omega$, $\bold{y}\in\Y^\omega$, $T\in\Nats^+$, $\gamma\in(0,1)$, $\eta_0,\eta_1\in\Reals^+$ and $\delta\in(0,1)$. Define $\beta$ and $\F_n$ the same way as in Proposition A.N2. Denote $D:=\RVO{\F}$. Then,}\Co{i}

$$\sum_{l=0}^\infty\sum_{m=0}^{T-1}{\gamma^{lT+m}\W^{F_{lT}}\AP{\bold{x}_{lT+m},B}} \leq C_{\mathrm{A.N3}}\AP{1+D T\ln\frac{1}{1-\gamma}+\sqrt{D\beta\AP{\frac{1}{1-\gamma}}\frac{1}{1-\gamma}\ln{\frac{1}{1-\gamma}}}}$$

\textbf{Proof of Proposition A.N3}\Co{b}

TBD \textbf{Q.E.D.}\Co{b}

\textbf{Proof of Theorem 1}\Co{b}

We take $\pi^\dagger_\gamma:=\pi^\PSR_{\zeta\R T}$ for some $T\in\Nats^+$ to be specified later. Denote the Bayesian regret by

$$\Reg(\gamma):=\Ea{\T\sim\zeta}{\EU^*_{\T\R}(\gamma)-\EU^{\pi_{\gamma}^{\dagger}}_{\T\R}(\gamma)}$$

By Proposition A.N1

$$\Reg(\gamma)=\sum_{n=0}^\infty\gamma^{n}\Ea{}{\Ea{s\sim \SHy_n\AP{\Theta_n,\AT_n}}{\V_{\SHy_n\R}(s,\gamma)}-\Ea{s\sim \THy\AP{\Theta_n,\AT_n}}{\V_{\SHy_n\R}(s,\gamma)}}$$

$$\Reg(\gamma)\leq\sum_{n=0}^\infty\gamma^{n}\Ea{}{\Abs{\Ea{s\sim \SHy_n\AP{\Theta_n,\AT_n}}{\V_{\SHy_n\R}(s,\gamma)}-\Ea{s\sim \THy\AP{\Theta_n,\AT_n}}{\V_{\SHy_n\R}(s,\gamma)}}}$$

We will use the notation

$$\De\V_{\T\R}(\gamma):=\max_{s\in\St}{\V_{\T\R}(s,\gamma)}-\min_{s\in\St}{\V_{\T\R}(s,\gamma)}$$

It follows that

$$\Reg(\gamma)\leq\sum_{n=0}^\infty\gamma^{n}\Ea{}{\Delta\V_{\SHy_n\R}(\gamma)\Dtva{\SHy_n\AP{\Theta_n,\AT_n},\THy\AP{\Theta_n,\AT_n}} }$$

Consider $\Y=\Reals^\St$ equipped with the $L^1$ norm. Given $s\in\St$ and $a\in\A$, consider also the subspace

$$W_{sa}:=\Sp\ACM{\T(s,a)}{\T\in\Hy}$$

By Proposition A.N6, there is $\B_{sa}\in\PSD\AP{\Y}$ s.t.

1. For any $\mu\in\Delta\St$

$$\B_{sa}\mu^{\otimes2}\leq1$$

2. For any $\T,\tilde{\T}\in\Hy$

$$\frac{1}{4}\DL^2 B_{sa}\AP{\T(s,a)-\tilde{\T}(s,a)}^{\otimes2} \geq \Dtva{\T(s,a),\tilde{\T}(s,a)}^2$$

We now apply Proposition A.N5 with $\delta:=\frac{1}{2}(1-\gamma)^2$ and $\epsilon:=(1-\gamma)^2$. We get

$$\Pa{}{\SHy_*\in\bigcap_{n=0}^\infty\CS^\Hy\AB{\Theta\AT_{:n}\Theta_n,\frac{\B }{\beta(n+1)}}} \geq 1-\frac{1}{2}(1-\gamma)^2$$

Here, $\beta$ was defined in Proposition A.N5.

Since the hypothesis $\SHy_n$ is sampled from the posterior, for any $l\in\Nats$ we also have

$$\Pa{}{\SHy_{lT}\in\CS^\Hy\AB{\Theta\AT_{:lT}\Theta_{lT},\frac{\B }{\beta(lT+1)}}} \geq 1-\frac{1}{2}(1-\gamma)^2$$

$$\Pa{}{\SHy_*,\SHy_{lT}\in\CS^\Hy\AB{\Theta\AT_{:lT}\Theta_{lT},\frac{\B }{\beta(lT+1)}}} \geq 1-(1-\gamma)^2$$

Denote

$$\CSE_n:=\AC{\SHy_*,\SHy_{lT}\in\CS^\Hy\AB{\Theta\AT_{:lT}\Theta_{lT},\frac{\B }{\beta(lT+1)}}}\subseteq\Omega$$

We get

$$\Reg(\gamma)\leq\sum_{n=0}^\infty\gamma^{n}\AP{(1-\gamma)^2+\Ea{}{\Delta\V_{\SHy_n\R}(\gamma)\Dtva{\SHy_n\AP{\Theta_n,\AT_n},\THy\AP{\Theta_n,\AT_n}};G_n }}$$

$$\Reg(\gamma)\leq1-\gamma+\sum_{n=0}^\infty\gamma^{n}\Ea{}{\Delta\V_{\SHy_n\R}(\gamma)\Dtva{\SHy_n\AP{\Theta_n,\AT_n},\THy\AP{\Theta_n,\AT_n}};G_n }$$

Using property 2 of B

$$\Reg(\gamma)\leq1-\gamma+\frac{1}{2}\DL\sum_{n=0}^\infty\gamma^{n}\Ea{}{\Delta\V_{\SHy_n\R}(\gamma)\sqrt{B_{\Theta_n\AT_n}\AP{\SHy_n\AP{\Theta_n,\AT_n}-\THy\AP{\Theta_n,\AT_n}}^{\otimes2}};G_n }$$

Denote

$$\Hy_n:=\CS^\Hy\AB{\Theta\AT_{:n}\Theta_n,\frac{\B }{\beta(n+1)}}$$

Clearly

$$\Pa{}{\sqrt{B_{\Theta_n\AT_n}\AP{\SHy_n\AP{\Theta_n,\AT_n}-\THy\AP{\Theta_n,\AT_n}}^{\otimes2}}\leq\W^{\Hy_{\Floor{n/T}}}\AP{\Theta_n\AT_n,B};G_n}=1$$

$$\Reg(\gamma)\leq1-\gamma+\frac{1}{2}\DL\sum_{n=0}^\infty\gamma^{n}\Ea{}{\Delta\V_{\SHy_n\R}(\gamma)\W^{\Hy_{\Floor{n/T}}}\AP{\Theta_n\AT_n,B};G_n }$$

Dropping $;G_n$ (since it can only make the right hand side smaller) and moving the sum inside the expected value, we get

$$\Reg(\gamma)\leq1-\gamma+\frac{1}{2}\DL\Ea{}{\Delta\V_{\SHy_n\R}(\gamma)\sum_{n=0}^\infty\gamma^{n}\W^{\Hy_{\Floor{n/T}}}\AP{\Theta_n\AT_n,B} }$$
 
% Apply A.N3 (widths)

...

It is easy to see that, for each $\T\in\Hy$, $s\in\St$ and $\gamma\in(0,1)$ we have

$$\V^0_{\T\R}\AP{s}-(1-\gamma)\partial_{\max}\V_{\T\R}(s,\gamma)\leq\V_{\T\R}(s,\gamma)\leq\V^0_{\T\R}\AP{s}-(1-\gamma)\partial_{\min}\V_{\T\R}(s,\gamma)$$

We will use the notation 

$$\partial_{\Delta}\V_{\T\R}(\gamma):=\max_{s\in\St}{\partial_{\max}\V_{\T\R}(s,\gamma)}-\min_{s\in\St}{\partial_{\min}\V_{\T\R}(s,\gamma)}$$

By the no-traps condition, $\V^0_{\T\R}(s)$ doesn't depend on $s$ and we get

...
 
% Use the dominated convergence theorem to recover \tau_{\zeta\R}

TBD \textbf{Q.E.D.}\Co{b}

\begin{Huge}Appendix\end{Huge}

The proposition below appeared in Osband and Van Roy as "Lemma 1", so we state it without proof.

\textbf{Proposition B.N1 (Osband-Van Roy)}\Co{b}

\textit{TBD}\Co{i}

% episode decoupling seems to be redundant, it is already taken care of using the "Lipschitz constant"
\Co{...

\textbf{Proposition B.N3}\Co{b}

Consider some $\gamma\in(0,1)$, $\tau\in(0,\infty)$, $T\in\Nats^+$, a universe..., some $\pi^*: ? \rightarrow \A$ and some $\pi^0: ? \K \A$. Assume that $\gamma \geq \gamma_M$. For any $n \in \Nats$, let $\pi^*_n$ be a policy s.t. for any $h \in ?$

$$\pi^*_n(h):=\begin{cases} \pi^0(h) \text{ if } \Abs{h} < nT \\ \pi^*(h) \text{ otherwise} \end{cases}$$

Assume that for any $h \in ?$

i. $$\pi^*(s) \in \A_{M}^\omega\AP{S(h)}$$

ii. $$\Supp{\pi^0(h)} \subseteq \A_{M}^0\AP{S(h)}$$

iii. For any $\theta\in(\gamma,1)$ $$\Abs{\frac{\D\V_{M}\AP{S(h),\theta}}{\D\theta}} \leq \tau$$

Then

$$\EU^{*}_\upsilon(\gamma)-\EU^{\pi^0}_\upsilon(\gamma) \leq (1-\gamma)\sum_{n=0}^\infty \sum_{m=0}^{T-1} \gamma^{nT+m}\left(\E{x\sim\mu\bowtie\pi^*_n}\left[r\left(x_{:nT+m}\right)\right]-\E{x\sim\mu\bowtie\pi^0}\left[r\left(x_{:nT+m}\right)\right]\right) + \frac{2\tau\gamma^T(1-\gamma)}{1-\gamma^T}$$}

The next proposition appeared (in slightly greater generality) in Osband and Van Roy as "Proposition 5".

\textbf{Proposition B.N2 (Osband-Van Roy)}\Co{b}

\textit{There is some $C_{\mathrm{B.N2}}\in\Reals^+$ s.t. the following holds.}\Co{i}

\textit{Consider a finite set $\X$, the vector space $\Y:=\Reals^d$ for some $d\in\Nats$ and some $\F\subseteq\AC{\X\rightarrow\Y}$. Let $\AC{\mathfrak{H}_n\subseteq\PS{\X^\omega\times\Y^\omega}}_{n\in\Nats}$ be the natural filtration, i.e.}\Co{i}

$$\mathfrak{H}_n:=\ACM{A'\subseteq\X^\omega\times\Y^\omega}{A'=\ACM{\bold{xy}}{\bold{xy}_{:n}\in A},\ A\subseteq\X^n\times\Y^n\text{ Borel}}$$

\textit{Consider also $f^*\in\F$ and $\mu\in\Delta\AP{\X^\omega\times\Y^\omega}$ s.t. for any $n\in\Nats$ and $x\in\X$}\Co{i}

$$\CE{\bold{xy}\sim\mu}{\bold{y}_n}{\bold{x}_n=x,\ \mathfrak{H}_n} = f^*(x)$$

\textit{Assume that $M\in\Reals^+$ is s.t. $\bold{y}_n\cdot\bold{y}_n\leq M^2$ with $\mu$-probability $1$ for all $n\in\Nats$. Assume also that $\sigma\in\Reals^+$ is s.t. $\mu$ is $\sigma$-sub-Gaussian, i.e., for any $\alpha\in\Y$, $n\in\Nats$ and $x\in\X$}\Co{i}

$$\CE{\bold{xy}\sim\mu}{\exp\AP{\alpha\cdot\AP{\bold{y}_n-f^*(x)}}}{\bold{x}_n=x,\ \mathfrak{H}_n} \leq \exp\AP{\frac{\sigma^2\AP{\alpha\cdot\alpha}}{2}}$$

\textit{Fix $\epsilon\in\Reals^+$, $\delta\in(0,1)$. Define $\beta:\Reals^+\rightarrow\Reals$ and, for each $n\in\Nats$, $Q_n\in\PD(\Y)$ by}\Co{i}

$$\beta(t):=C_{\mathrm{B.N2}}\AP{\sigma^2 \ln{\frac{\N(\F,\epsilon)}{\delta}}+\epsilon t\AP{M+\sigma\ln{\frac{et}{\delta}}}}$$

$$Q_n(v,w):=\frac{v\cdot w}{\beta(n+1)}$$

\textit{Then,}\Co{i}

$$\Pa{\bold{xy}\sim\mu}{f^*\not\in\bigcap_{n=0}^\infty\CS^\F\AB{\bold{xy}_{:n},Q_n}} \leq \delta$$

Note that we removed a square root in the definition of $\beta$ compared to equation (7) in Osband and Van Roy. This only makes $\beta$ larger (up to a constant factor) and therefore, only makes the claim weaker.

\end{document}


