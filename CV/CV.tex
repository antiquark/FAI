%&latex
%% Derived from: `cvctan.tex'

\documentclass[a4paper]{article}

\usepackage{tabularx}

%\usepackage{doublespace}
%\setstretch{1.2}

%\usepackage{ae}
\usepackage[T1]{fontenc}
\usepackage{CV}
\usepackage{hyperref}

\begin{document}

\pagestyle{empty}

%Ueberschrift
\begin{center}
\huge{\textsc{Curriculum Vitae}}
\vspace{\baselineskip}

\Large{\textsc{Vanessa Kosoy}}
\end{center}
\vspace{1.5\baselineskip}

\section{Contact}

\begin{flushleft}
  Email: vanessa.kosoy@intelligence.org \\
  Phone: TBD
\end{flushleft}

\section{About}

Trying to save the world from existential catastrophes caused by Artificial Superintelligence. Autodidact. 

Math/Compsci background: statistical learning theory, computational complexity theory, algorithmic information theory, functional analysis, probability theory, differential geometry, algebraic geometry, category theory. 

Physics background: quantum field theory, general relativity, mathematical physics.

\section{Publications}

"Delegative Reinforcement Learning: learning to avoid traps with a little help", V. Kosoy, International Conference on Learning Representations 2019, {SafeML workshop}\\

"Forecasting using incomplete models", V. Kosoy ({preprint}, 2017) \\

"Optimal Polynomial-Time Estimators: A Bayesian Notion of Approximation Algorithm", V. Kosoy and A. Appel, Journal of Applied Logics ({forthcoming})

\section{Employment}

\begin{CV}
\item[June 2015--present] Research Associate, Machine Intelligence Research Institute

My research aims at mathematical formalization of general intelligence and value alignment, using tools from computational learning theory and algorithmic information theory. Such mathematical models serve to elucidate the potential failure modes of AGI, clarify confusing conceptual questions, and lead to AI algorithms satisfying theoretical guarantees that imply safety and effectiveness under clear and (ultimately) realistic assumptions.

\item[Jan 2014--June 2015] Start-up founder and freelance consultant, specializing in algorithm engineering and computer vision.

\item[Dec 2012--Dec 2013] Software Product/Group Manager, Mantis Vision

Managing of the software R\&D of the company: software and algorithm engineers and a test engineer. The group developed a new software product (small form factor 3D scanner) virtually from scratch. I was fully responsible for the group and personally responsible for software specification. Additionally I operated several subcontractors supplementing the work done inside the company.

\item[2004--Oct 2012] Team Leader, VisionMap

One of the 1st workers in what was initially a start-up. Managing a team of up to 7 people, comprised of algorithm developers and software engineers. Developing algorithms in the fields of photogrammetry, computer vision and image processing.

I worked on a project which received the Israel Defense Prize in 2009.

\end{CV}

\section{Formal Education}

\begin{CV}

\item[2004--2007] BSc, Pure Mathematics, cum laude (Tel Aviv University)

\end{CV}

\end{document}

%Tabellen
\begin{table}[htbp] \centering%
\begin{tabular}{lll}\hline\hline
1 & 2 & 3 \\ \hline
1 & \multicolumn{2}{c}{2} \\
\hline
\end{tabular}
\caption{Titel\label{Tabelle: Label}}
\end{table}







