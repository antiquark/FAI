%&latex
\documentclass[a4paper]{article}

\usepackage[a4paper,margin=1in]{geometry}
\usepackage[affil-it]{authblk}
\usepackage{cite}
\usepackage[unicode]{hyperref}
\usepackage[utf8]{inputenc}
\usepackage[english]{babel}
\usepackage{csquotes}
\usepackage{amsmath,amssymb,amsthm}
\usepackage{enumerate}
\usepackage{commath}
\usepackage{bm}

\newcommand{\Bool}{\{0,1\}}
\newcommand{\Words}{{\Bool^*}}
\newcommand{\WordsLen}[1]{{\Bool^{#1}}}

% operators that are separated from the operand by a space
\DeclareMathOperator{\Sgn}{sgn}
\DeclareMathOperator{\Supp}{supp}
\DeclareMathOperator{\Stab}{stab}
\DeclareMathOperator{\Img}{Im}

% operators that require brackets
\DeclareMathOperator{\Prb}{Pr}
\DeclareMathOperator{\E}{E}
\newcommand{\EE}[2]{\operatorname{E}_{\substack{#1 \\ #2}}}
\newcommand{\EEE}[3]{\operatorname{E}_{\substack{#1 \\ #2 \\ #3}}}
\DeclareMathOperator{\Var}{Var}

\newcommand{\FOO}[2]{\operatorname{E}_{\substack{#1 \\ #2}}}

% operators that require parentheses
\DeclareMathOperator{\Ent}{H}
\DeclareMathOperator{\Hom}{Hom}
\DeclareMathOperator{\End}{End}
\DeclareMathOperator{\Sym}{Sym}
\DeclareMathOperator{\Ev}{ev}

% special symbols that are not really operators
\DeclareMathOperator{\T}{T}
\DeclareMathOperator{\R}{r}
\DeclareMathOperator{\A}{a}
\DeclareMathOperator{\M}{M}
\DeclareMathOperator{\UM}{UM}
\DeclareMathOperator{\Un}{U}
\DeclareMathOperator{\En}{c}

\newcommand{\KL}[2]{\operatorname{D}_{\mathrm{KL}}(#1 \| #2)}
\newcommand{\Dtv}{\operatorname{d}_{\textnormal{tv}}}

\newcommand{\Argmin}[1]{\underset{#1}{\operatorname{arg\,min}}\,}
\newcommand{\Argmax}[1]{\underset{#1}{\operatorname{arg\,max}}\,}

\newcommand{\Nats}{\mathbb{N}}
\newcommand{\Ints}{\mathbb{Z}}
\newcommand{\Rats}{\mathbb{Q}}
\newcommand{\Reals}{\mathbb{R}}
\newcommand{\Coms}{\mathbb{C}}

\newcommand{\NatPoly}{\Nats[K_0, K_1 \ldots K_{n-1}]}
\newcommand{\NatPolyJ}{\Nats[J_0, J_1 \ldots J_{n-2}]}
\newcommand{\NatFun}{\Nats^n \rightarrow}

\newcommand{\Estr}{\bm{\lambda}}
\newcommand{\LLU}{\mathbf{LLU}}

\newcommand{\Lim}[1]{\lim_{#1 \rightarrow \infty}}
\newcommand{\LimInf}[1]{\liminf_{#1 \rightarrow \infty}}
\newcommand{\LimSup}[1]{\limsup_{#1 \rightarrow \infty}}

\newcommand{\Abs}[1]{\lvert #1 \rvert}
\newcommand{\Norm}[1]{\lVert #1 \rVert}
\newcommand{\Floor}[1]{\lfloor #1 \rfloor}
\newcommand{\Ceil}[1]{\lceil #1 \rceil}
\newcommand{\Chev}[1]{\langle #1 \rangle}
\newcommand{\Quote}[1]{\ulcorner #1 \urcorner}

\newcommand{\Dist}{\mathcal{D}}
\newcommand{\GrowR}{\Gamma_{\mathfrak{R}}}
\newcommand{\GrowA}{\Gamma_{\mathfrak{A}}}
\newcommand{\Grow}{\Gamma:=(\GrowR,\GrowA)}
\newcommand{\MGrow}{\mathrm{M}\Gamma}
\newcommand{\Fall}{\mathcal{F}}
\newcommand{\EG}{\Fall(\Gamma)}
\newcommand{\ESG}{\Fall^\sharp(\Gamma)}
\newcommand{\EMG}{\Fall(\MGrow)}
\newcommand{\ESMG}{\Fall^\sharp(\MGrow)}
\newcommand{\BoolR}[1]{\Bool^{\R_{#1}(K)}}

\newcommand{\GammaPoly}{\Gamma_{\textnormal{poly}}}
\newcommand{\GammaLog}{\Gamma_{\textnormal{log}}}
\newcommand{\FallU}{{\Fall_{\textnormal{uni}}^{(n)}}}
\newcommand{\FallUt}[1]{{\Fall_{\textnormal{uni}}^{(#1)}}}
\newcommand{\FallM}{\Fall_{\textnormal{mon}}^{(n)}}

\newcommand{\Alg}{\xrightarrow{\textnormal{alg}}}
\newcommand{\Markov}{\xrightarrow{\textnormal{mk}}}
\newcommand{\Scheme}{\xrightarrow{\Gamma}}
\newcommand{\MScheme}{\xrightarrow{\MGrow}}
\newcommand{\ORC}{\xrightarrow{\textnormal{orc}}}

\newcommand{\Base}{\mathcal{B}}

\begin{document}

*The notation of this post is based on VK16 rather than the notation of previous posts. The concept of "quasi-optimal estimator" introduced here is completely unrelated to the previously introduced "quasi-optimal predictors". The latter term should be considered obsolete (it can be expressed different in the language of VK16).*

Inspired by (presently unpublished) work by Garrabrant, we introduce a generalization of the concept of optimal polynomial-time estimators which we named "quasi-optimal polynomial-time estimators." As agnostic PAC learning (with a quadratic loss function) gives rise to optimal estimators, online convex optimisation gives rise to quasi-optimal estimators. At the same time, most properties of optimal estimators carry over to quasi-optimal estimators.

%##Results
\section{Results}

%#Definition 1
\subsection{Definition 1}

An *ordered ring circuit* is a circuit with 3 types of binary gates: ${+}$, ${\times}$ and ${\max}$. Given ${A,B}$ finite sets, ${\Rats^A \ORC \Rats^B}$ is the set of ordered ring circuits with $\Abs{A}$ normal inputs labeled by ${A}$, ${\Abs{B}}$ outputs labeled by ${B}$ and a finite set of auxiliary inputs that receive constants in ${\Rats}$ (we can think of them as 0-nary gates). The notation ${\phi: \Rats^A \ORC \Rats^B}$ means ${\phi \in (\Rats^A \ORC \Rats^B)}$. Any ${\phi: \Rats^A \ORC \Rats^B}$ can be regarded as a function from ${\Rats^A}$ to ${\Rats^B}$ in the obvious way (circuit evaluation). 

Given ${\phi: \Rats^A \ORC \Rats^B}$ with ${N}$ gates and ${M}$ auxiliary inputs ${\{c_i \in \Rats\}_{i \in [M]}}$, the *norm* of ${\phi}$ is ${\Norm{\phi}:=N+\sum_{i \in [M]} \Abs{c_i}}$.

***

For the rest of the post, we fix ${n \in \Nats^{>0}}$ and ${\Gamma}$ a pair of growth space of rank ${n}$.

%#Definition 2
\subsection{Definition 2}

Consider a finite set ${\Base}$, ${\{(\Dist_i,f_i)\}_{i \in \Base}}$ a collection of distributional estimation problems and ${\{\Fall_i\}_{i \in \Base}}$ a collection of fall spaces. ${\{P_i: \Words \Scheme \Rats\}_{i \in \Base}}$ is called a *family of ${\EG}$-quasi-optimal polynomial-time estimators for ${(\Dist,f)}$* when for any ${Q: \Words \Scheme (\Rats^{\Base} \ORC \Rats^{\Base})}$ s.t. ${\Norm{Q^K(x,y)}}$ is bounded by a constant, there are ${p \in \NatPoly}$ and ${\{\varepsilon_i \in \Fall_i\}_{i \in \Base}}$ s.t.

$$\forall i \in \Base, J \in \Nats^{n-1}, k \in \Nats, l \geq p(J,k): \EE{x \sim \Dist^{Jk}}{y \sim \Un_P^{Jl}}[(P_i^{Jl}(x,y)-f_i(x))^2] \leq \EEE{x \sim \Dist^{Jk}}{y \sim \Un_P^{Jl}}{z \sim \Un_Q^{Jk}}[(Q^{Jk}(x,z;P^{Jl}(x,y))_i-f_i(x))^2]+\varepsilon_i(J,l)$$

Here, the notation ${Q^{Jk}(x,z;P^{Jl}(x,y))_i}$ means that the circuit ${Q^{Jk}(x,z)}$ is evaluated on ${\{P^{Jl}(x,y)_j\}_{j \in \Base}}$ and component ${i}$ of the result is used.

***

Moo

%##Appendix
\section{Appendix A}

Foo

%##Appendix
\section{Appendix B}

Foo

%##Appendix
\section{Appendix C}

Foo

\end{document}


