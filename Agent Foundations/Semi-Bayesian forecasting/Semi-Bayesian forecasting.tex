%&latex
\documentclass[a4paper]{article}

\usepackage[a4paper,margin=1in]{geometry}
\usepackage[affil-it]{authblk}
\usepackage{cite}
\usepackage[unicode]{hyperref}
\usepackage[utf8]{inputenc}
\usepackage[english]{babel}
\usepackage{csquotes}
\usepackage{amsmath,amssymb,amsthm}
\usepackage{enumerate}
\usepackage{commath}
\usepackage{bm}

\newcommand{\Bool}{\{0,1\}}
\newcommand{\Words}{{\Bool^*}}

% operators that are separated from the operand by a space
\DeclareMathOperator{\Sgn}{sgn}
\DeclareMathOperator{\Supp}{supp}
\DeclareMathOperator{\Stab}{stab}
\DeclareMathOperator{\Img}{Im}

% operators that require brackets
\DeclareMathOperator{\Prb}{Pr}
\DeclareMathOperator{\E}{E}
\newcommand{\EE}[2]{\operatorname{E}_{\substack{#1 \\ #2}}}
\newcommand{\EEE}[3]{\operatorname{E}_{\substack{#1 \\ #2 \\ #3}}}
\DeclareMathOperator{\Var}{Var}

% operators that require parentheses
\DeclareMathOperator{\Ent}{H}
\DeclareMathOperator{\Hom}{Hom}
\DeclareMathOperator{\End}{End}
\DeclareMathOperator{\Sym}{Sym}

\DeclareMathOperator{\Prj}{pr}

\newcommand{\KL}[2]{\operatorname{D}_{\mathrm{KL}}(#1 \| #2)}
\newcommand{\Dtv}{\operatorname{d}_{\textnormal{tv}}}

\newcommand{\Argmin}[1]{\underset{#1}{\operatorname{arg\,min}}\,}
\newcommand{\Argmax}[1]{\underset{#1}{\operatorname{arg\,max}}\,}

\newcommand{\Nats}{\mathbb{N}}
\newcommand{\Ints}{\mathbb{Z}}
\newcommand{\Rats}{\mathbb{Q}}
\newcommand{\Reals}{\mathbb{R}}
\newcommand{\Coms}{\mathbb{C}}

\newcommand{\Estr}{\bm{\lambda}}

\newcommand{\Lim}[1]{\lim_{#1 \rightarrow \infty}}
\newcommand{\LimInf}[1]{\liminf_{#1 \rightarrow \infty}}
\newcommand{\LimSup}[1]{\limsup_{#1 \rightarrow \infty}}

\newcommand{\Abs}[1]{\lvert #1 \rvert}
\newcommand{\Norm}[1]{\lVert #1 \rVert}
\newcommand{\Floor}[1]{\lfloor #1 \rfloor}
\newcommand{\Ceil}[1]{\lceil #1 \rceil}
\newcommand{\Chev}[1]{\langle #1 \rangle}
\newcommand{\Quote}[1]{\ulcorner #1 \urcorner}

\newcommand{\Alg}{\xrightarrow{\textnormal{alg}}}
\newcommand{\Markov}{\xrightarrow{\textnormal{mk}}}

\newcommand{\Prob}{\mathcal{P}}

% Paper specific

\newcommand{\Act}{\mathcal{A}}
\newcommand{\Obs}{\mathcal{O}}

\begin{document}

*This post continues the research programme of attacking the grain of truth problem by departure from the Bayesian paradigm. In the [previous post](https://agentfoundations.org/item?id=1046), I suggested using Savage's minimax regret decision rule, but here I fall back to the simple minimax decision rule. This is because I failed to prove a satisfactory asymptotic optimality theorem for minimax regret (it appears like a problem of principle rather than merely technical).*

We consider "semi-Bayesian" agents following the minimax expected utility decision rule, in oblivious environments with full monitoring (a setting that we will refer to as "forecasting"). This setting is considered in order to avoid the need to enforce exploration, as a preparation for analysis of general environments. We show that such agents satisfy a certain asymptotic optimality theorem. Intuitively, this theorem means that whenever the environment satisfies an incomplete model that is included in the prior, the agent will eventually learn this model i.e. extract at least as much utility as can be guaranteed for this model.

The proofs are in the Appendix.

\section{Notation}

Given ${X}$ a topological space, ${\Prob(X)}$ will denote the space of Borel probability measures on ${X}$. We regard it as a topological space using the weak${^*}$ topology.

\section{Elementary properties of minimax}

Fix ${S}$ and ${E}$ compact Polish spaces, ${S}$ representing the agent's pure policies and ${E}$ representing the pure environments. Let ${u: S \times E \rightarrow \Reals}$ be a continuous utility function.

\subsection{Proposition 1}

Consider ${\Phi \subseteq \Prob(E)}$. There exists 

$${\pi^* \in \Argmax{\pi \in \Prob(S)} \inf_{\mu \in \Phi} \E_{\rho \times \mu}[u]}$$

Moreover, let ${\bar{\Phi}}$ be the closure of the convex hull of ${\Phi}$. Then, the same ${\rho^*}$ satisfies

$${\pi^* \in \Argmax{\pi \in \Prob(S)} \min_{\mu \in \bar{\Phi}} \E_{\rho \times \mu}[u]}$$

We will say that such a ${\pi^*}$ is a *minimax policy for ${\Phi}$*.

\subsection{Proposition 2}

Consider ${\Phi, \Phi' \subseteq \Prob(E)}$ convex closed and ${p \in [0,1]}$. Then, there is ${\nu^* \in \Phi'}$ s.t. any minimax policy for ${p\Phi+(1-p)\Phi'}$ is a minimax policy for ${p\Phi+(1-p)\nu^*}$.

***

In particular, Proposition 2 implies that a minimax policy for ${\Phi}$ is the optimal policy for some ${\mu^* \in \Phi}$.

Now we ask what policy is implemented by "subagents" created by a minimax policy. Suppose ${S = S_1 \times S_2}$, where ${S_2}$ represents the pure policies of the subagent. Assume that there is a Borel set ${A \subseteq E}$ (the condition for the creation of the subagent), a finite set ${T}$ (the part of the policy executed before the creation of the subagent), a Borel measurable mapping ${\alpha: S_1 \rightarrow T}$ and continuous functions ${u_1: S_1 \times E \rightarrow \Reals}$ and ${u_2: S_2 \times T \times E \rightarrow \Reals}$ s.t. 

$${u(s_1,s_2,e)=u_1(s_1,e)+\chi_A(e) u_2(s_2,\alpha(s_1),e)}$$

Consider ${\Phi \subseteq \Prob(E)}$ convex closed and ${\pi^* \in \Prob(S)}$ minimax for ${\Phi}$. Denote ${\Prj_{1,2}: S \rightarrow S_{1,2}}$ the projection mappings. Define ${\pi_1^* \in \Prob(S_1)}$ by

$$\pi_1^* := \Prj_{1*}\pi^*$$

Define ${\pi_2^*: T \rightarrow \Prob(S_2)}$ by

$$\pi_2^*(t) := \Prj_{2*} (\pi^* \mid \alpha^{-1}(t) \times S_2)$$

\subsection{Proposition 3}

$$\pi_2^* \in \Argmax{\pi_2: T \rightarrow \Prob(S_2)} \min_{\mu \in \Phi} (\E_{\pi_1^* \times \mu}[u_1] + \mu(A) \E_{t \sim \alpha_*\pi_1^*}[\E_{\pi_2^*(t) \times \mu}[u_2(t)]])$$

***

At this point, it should be possible to make do without ${T}$ and the associated decomposition of ${u}$ by instead decomposing ${\pi^*}$ into ${\pi_1^*}$ and a Markov kernel from ${S_1}$ to ${S_2}$. However, we won't need the general case.

In general, ${\pi_2^*}$ is not a minimax policy for any natural model i.e. the minimax decision rule is not "dynamically consistent". However, if we assume a certain factorization condition, it is. Specifically, assume ${E_1, E_2}$ are compact Polish, ${f: E_1 \times E_2 \rightarrow E}$ is Borel measurable, ${\bar{u}_1: S_1 \times E_1 \rightarrow \Reals}$ and ${\bar{u}_2: S_2 \times T \times E_2 \rightarrow \Reals}$ are continuous s.t. 

$${u_1(s_1,f(e_1,e_2))=\bar{u}_1(s_1,e_1)}$$

$${u_2(s_2,t,f(e_1,e_2))=\bar{u}_1(s_2,t,e_2)}$$

Further assume that ${A_1 \subseteq E_1}$ is Borel s.t. ${f^{-1}(A) = A_1 \times E_2}$ and ${\Phi_{1,2} \subseteq E_{1,2}}$ are convex closed s.t. ${\Phi}$ is the closure of the convex hull of ${f(\Phi_1 \times \Phi_2)}$.

\subsection{Proposition 4}

$$\pi_2^* \in \Argmax{\pi_2: T \rightarrow \Prob(S_2)} \min_{\mu \in \Phi_2} \E_{t \sim \alpha_*\pi_1^*}[\E_{\pi_2^*(t) \times \mu}[\bar{u}_2(t)]]$$

Moreover, define ${\bar{S}_2:=T \rightarrow S_2}$ equipped with the product topology, define ${\bar{\pi}_2^* \in \Prob(\bar{S}_2)}$ by

$${\bar{\pi}_2^* := \prod_{t \in T} \pi_2^*(t)}$$

Define ${\bar{E}_2:=T \times E_2}$ and define ${\bar{\Phi}_2 \subseteq \bar{E}_2}$ by

$$\bar{\Phi}_2:=\{r_* \pi_1^* \times \mu \mid \mu \in \Phi_2\}$$

Finally, define ${\hat{u}_2: \bar{S}_2 \times \bar{E}_2 \rightarrow \Reals}$ by

$$\hat{u}_2(s,t,e):=\bar{u}_2(s(t),t,e)$$

Then, ${\bar{\pi}_2^*}$ is a minimax policy for ${\bar{\Phi}_2}$ w.r.t. the utility function ${\hat{u}_2}$.

\section{Asymptotic optimality}

Fix finite sets ${\Act}$ and ${\Obs}$. We now assume that ${E=\Obs^\omega}$ and ${S=\Obs^* \rightarrow \Act}$ with the product topology. We fix ${\gamma: \Nats \rightarrow \Reals^{\geq 0}}$ a time discount function satisfying ${\sum_n \gamma(n) < \infty}$ and ${r: (\Act \times \Obs)^* \rightarrow \Reals}$ a bounded reward function. Given ${e \in E}$ and ${s \in S}$, we define ${e^s \in (\Act \times \Obs)^\omega}$ by

$${e^s_n:=(s(e_{<n}),e_n)}$$

The utility function is then given by

$$u(s,e)=\sum_{n \in \Nats} \gamma(n) r(e^s_{<n})$$

Consider ${\Phi,\Phi' \subseteq \Prob(E)}$ and ${p \in [0,1]}$. Denote ${\Psi = p \Phi + (1-p) \Phi'}$. We think of ${\Psi}$ as the prior, ${\Phi}$ as one of the models in the prior (assigned probability ${p}$) and ${\Phi'}$ as the convex combination of all other models. Let ${\pi^* \in \Prob(S)}$ be a minimax policy for ${\Psi}$. Define ${v: \Obs^* \rightarrow \Reals}$ by

$$v(x):=\max_{\pi_2: T \rightarrow \Prob(S_2)} \min_{\mu \in \Phi_2} \E_{t \sim \alpha_*\pi_1^*}[\E_{\pi_2^*(t) \times \mu}[\bar{u}_2(t)]]???$$

%##Appendix
\section{Appendix}

Bar

\end{document}


